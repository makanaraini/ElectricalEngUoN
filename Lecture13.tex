
\section{Lecture 13: 04/11/2024}

\subsection{Introduction to Analog Multipliers}

Today, we will continue our discussion on analog multipliers, a topic we began yesterday. In telecommunications, analog multipliers are often represented by a symbol consisting of a square with a triangle on top, typically marked as \( x \) and \( y \). Here, the signal \( V_x \) is connected to one input, \( V_y \) to the other, and the output is \( V_{out} \). All these signals are referenced with respect to the Earth.

In the introduction, we explored the concept of an ID multiplier, its limitations in practical applications, and how we can realize analog multipliers effectively.

\subsection{Log-Antilog Method}

The first scheme we discussed is the \textbf{Log-Antilog} method. This approach utilizes log amplifiers and anti-log amplifiers. The fundamental equation governing this method is:

\[
XY = \text{antilog}(\log X + \log Y)
\]

To implement this circuit, two log amplifiers are required, followed by an anti-log amplifier to obtain the product \( XY \). However, upon studying high-quality log and anti-log amplifiers, we observe that they are highly complicated, containing numerous operational amplifiers (op-amps). These op-amps suffer from offsets, necessitating multiple setting circuits to compensate. Additionally, log and anti-log circuits that handle unipolar signals are not unidirectional, making the design even more complex. Due to these challenges, this implementation is seldom used in practice and is primarily considered academic.

\subsection{Quarter Square Method}

The next implementation technique is the \textbf{Quarter Square} method, which is the focus of today's lecture. If time permits, we will also introduce the \textbf{Transconductance Method}, sometimes referred to as the \textbf{Variable Transconductance Multiplier} circuit.

\subsubsection{Mathematical Basis}

In the Quarter Square method, we use the identity:

\[
\frac{1}{4}[(x + y)^2 - (x - y)^2] = XY
\]

Expanding both terms:

\[
(x + y)^2 = x^2 + 2xy + y^2
\]
\[
(x - y)^2 = x^2 - 2xy + y^2
\]

Subtracting the two:

\[
(x + y)^2 - (x - y)^2 = 4xy \implies \frac{1}{4}(x + y)^2 - \frac{1}{4}(x - y)^2 = XY
\]

\subsubsection{Circuit Implementation}

To implement this identity, the circuit requires:

\begin{itemize}
    \item Adders for \( x + y \) and \( x - y \)
    \item Squaring circuits for both \( (x + y) \) and \( (x - y) \)
    \item A subtractor to compute \( \frac{1}{4}[(x + y)^2 - (x - y)^2] \)
\end{itemize}

The block diagram of the circuit is as follows:

\begin{figure}[h]
    \centering
    \includegraphics[width=0.8\textwidth]{quarter_square_method.png}
    \caption{Quarter Square Multiplier Circuit}
    \label{fig:quarter_square}
\end{figure}

In this circuit:

\begin{itemize}
    \item \( V_x \) is connected to the inverting input of amplifier \( A_1 \), resulting in an output of \( -x \).
    \item Similarly, \( V_y \) is connected to the inverting input of amplifier \( A_2 \), resulting in an output of \( -y \).
    \item These signals pass through diodes \( D_2 \), \( D_3 \), and \( D_4 \) to perform the necessary additions and subtractions.
\end{itemize}

\subsection{Simplified Analysis}

To simplify the analysis, consider a simplified circuit with four diodes \( D_1, D_2, D_3, D_4 \). The currents through these diodes are denoted as \( I_A \) and \( I_B \).

Let:

\[
Z = X + Y
\]

Then:

\[
I_B = I_{N} \left(e^{KZ} - 1\right)
\]
\[
I_A = I_{N} \left(-KZ + \frac{(KZ)^2}{2}\right)
\]

Summing the currents:

\[
I_1 = I_A + I_B = I_{N} \left(2KZ^2\right)
\]

\subsection{Output Voltage Calculation}

The output voltage \( V_{out} \) can be expressed as:

\[
V_{out} = -I_3 R_F = -I_{N} K^2 (X^2 + Y^2 - (X - Y)^2)
\]

To achieve multiplication, we set:

\[
R_F I_{N} K^2 = \frac{1}{4}
\]

\subsection{Limitations of the Quarter Square Method}

While the Quarter Square method simplifies the multiplier design, the accuracy is limited due to:

\begin{itemize}
    \item Approximate squaring using diodes
    \item Offset errors from operational amplifiers
    \item Complexity in setting up the necessary resistor and amplifier configurations
\end{itemize}

Therefore, this implementation is rarely used in practical applications.

\subsection{Variable Transconductance Method}

The next scheme is the \textbf{Variable Transconductance Method}, also known as the \textbf{Transconductance Multiplier}. This method varies the transconductance of a differential amplifier using a second signal to achieve multiplication.

\subsubsection{Basic Two Quadrant Multiplier}

The simplest implementation of this method is the \textbf{Basic Two Quadrant Multiplier}, which provides multiplication results in only two quadrants. Here, one variable can take positive values, while the other can take both positive and negative values.

The output voltage \( V_{out} \) is given by:

\[
V_{out} = -G_m V_1 R_L
\]

where \( G_m \) is the transconductance, which varies with input \( V_2 \):

\[
G_m = K V_2
\]

Substituting:

\[
V_{out} = -K V_1 V_2 R_L
\]

\subsubsection{Circuit Implementation Challenges}

This method faces several challenges:

\begin{itemize}
    \item Difficulty in measuring inputs and outputs simultaneously due to different ground references
    \item Limitations in handling DC voltages unless \( V_2 \) overcomes the \( V_{BE} \) threshold
    \item Floating output requiring additional amplification to reference ground
\end{itemize}

These limitations make the Basic Two Quadrant Multiplier less practical for certain applications.

\subsection{Gilbert Multiplier Cell}

To overcome the limitations of the Basic Two Quadrant Multiplier, the \textbf{Gilbert Multiplier Cell} was developed. This configuration connects two differential amplifiers in a parallel and specialized manner, enhancing performance and allowing for high-frequency operations.

\subsubsection{Circuit Overview}

\begin{figure}[h]
    \centering
    \includegraphics[width=0.8\textwidth]{gilbert_multiplier_cell.png}
    \caption{Gilbert Multiplier Cell}
    \label{fig:gilbert_multiplier}
\end{figure}

In this circuit:

\begin{itemize}
    \item \( Q_1 \) and \( Q_2 \) form the first differential amplifier.
    \item \( Q_3 \) and \( Q_4 \) form the second differential amplifier.
    \item \( Q_5 \) and \( Q_6 \) are used to vary the transconductance based on the input \( V_2 \).
    \item A constant current source \( I_{total} \) ensures stable operation.
\end{itemize}

\subsubsection{Assumptions for Analysis}

\begin{itemize}
    \item Transistors \( Q_1, Q_2, Q_3, Q_4, Q_5, Q_6 \) are matched for optimal performance.
    \item Base currents are negligible due to high-beta transistors.
    \item The circuit is properly biased, ignoring static currents for dynamic analysis.
\end{itemize}

\subsubsection{Operation Principle}

The Gilbert Multiplier Cell operates by shifting currents between transistors based on the input voltages \( V_1 \) and \( V_2 \). The dynamic currents, resulting from signal variations, are analyzed to determine the output multiplication.

\subsection{Conclusion}

Analog multiplier circuits, while theoretically sound, present significant practical challenges in terms of complexity, accuracy, and biasing requirements. The Quarter Square method, although simpler, suffers from inaccuracies due to diode approximations and op-amp offsets. The Variable Transconductance method offers better performance but introduces its own set of challenges, particularly regarding ground references and DC operation.

The Gilbert Multiplier Cell emerges as a robust solution, addressing many of these limitations by providing high-frequency operation and better accuracy, making it suitable for applications in communication receivers and signal processing circuits.

\subsection{Applications}

Analog multipliers are essential in various applications, including:

\begin{itemize}
    \item \textbf{Communication Receivers:} Used in modulators and demodulators for processes like Double Side Band (DSB) modulation, Frequency Modulation (FM), and Phase Modulation (PM).
    \item \textbf{Frequency Translation:} Multiplying two cosine waves yields sum and difference frequencies, useful for signal processing.
    \item \textbf{Modulation Techniques:} Employed in both modulation and demodulation processes within modems.
\end{itemize}

Despite their complexities, analog multipliers remain a cornerstone in high-frequency and signal processing applications, particularly where precision and speed are paramount.

\subsection{Further Improvements}

To enhance the basic differential multiplier:

\begin{itemize}
    \item Introduce additional amplification stages to reference the output to the ground.
    \item Employ robust biasing techniques to minimize offset errors.
    \item Utilize matched transistors to maintain symmetry and improve accuracy.
\end{itemize}

These improvements pave the way for more reliable and accurate analog multipliers, expanding their applicability in advanced electronic systems.

