00:00:02	Hello everyone, I hope you can see and hear me clearly. Moses, you can switch on your mic on your computer. Can you read the text on the screen? Yes? Okay, great. We are now applying Electronics. I just wanted to confirm that you can read it well. Some students are still trying to log in, and I usually use full screen presentation to maximize the space on the screen. I can hear them trying to get in, but I can't see them. Please inform the students that if they are late, I will not be able to admit them as it interferes with the class.

00:03:03	Let's begin our course. This is what we are expected to cover. The first topic is nonlinear analog systems. Moses, can you hear me? Yes? Okay. We are supposed to study chopper techniques, lock-in amplifiers, logarithmic and anti-logarithmic amplifiers, multipliers, dividers, function generators, relaxation oscillators, and ramp oscillators. We usually cover relaxation oscillators and ramp oscillators in the third year. The next topic is power supplies, focusing mainly on DC power supplies. We need to know about capacitor filters, rectifiers, Zener regulators, emitter followers, closed-loop series and shunt regulators, protection circuits, and short circuit limits. The first part, which is capacitor input filters, is covered in your circuits course. We will focus on regular rectifiers and Zener diodes.

00:05:46	These topics are extensive, and I usually assign them as homework. Students studying Power Electronics will have an advantage. You need to know the basics, but we may not have time to cover everything in detail. Read about forced and natural commutation, di/dt, and dv/dt limitations. These are the rates at which current changes in commutating circuits, which are usually specified for devices like SCRs. The limitations include how fast the current can be turned on and the maximum rate of change of voltage across the switches. We will also discuss how to connect them in series and parallel. For example, if you have a diode rated at 50 volts, putting two in series will allow them to block 100 volts. Parallel connection is used to increase the current going to the load. If one switching device can handle 10 amps, putting two in parallel will allow them to handle 20 amps.

00:07:14	This applies to transistors as well. In a power output stage, you might use one transistor to switch the load to the upper voltage and another to switch to the lower voltage. If one transistor can handle 10 amps, putting two in parallel will allow them to handle 20 amps. This is known as parallel connection, which increases the current capability of the system. Series connection increases the voltage across the devices if the supply voltage is higher than what one device can handle. We will also cover the basic principles of inverters, converters, and cycloconverters. Some of these topics are outdated, but inverters are still relevant, especially in solar energy systems. Solar panels produce DC, but most appliances require AC, so we use inverters to convert DC to AC.

00:10:02	A converter does the opposite, converting DC to DC. For example, your laptop power supply takes AC voltage from the wall (usually 240 volts RMS) and converts it to DC using a full bridge rectifier. This rectifier converts AC to DC, which is then smoothed using a capacitor. The resulting DC voltage is around 340 volts. Inside your laptop power supply, a DC to DC converter steps down this voltage to around 19 volts DC. DC to DC converters have many applications, such as powering laptops, desktop computers, and other devices that require different voltages.

00:13:12	Inside a desktop computer, the power supply provides various voltages, such as 12 volts, 3.3 volts, and others. These voltages are generated using DC to DC converters. A step-down DC to DC converter takes a high input voltage (e.g., 340 volts) and outputs a lower voltage. There are also step-up DC to DC converters, which increase the input voltage. For example, an electric car like the Nissan Leaf or the first-generation Tesla requires a higher voltage to charge its battery. The house supply voltage (240 volts AC) is rectified and smoothed to around 340 volts DC, which is then stepped up to 400 volts or more to charge the battery.

00:15:48	Some electric cars, like the Porsche, use 800-volt batteries. To charge these batteries, the 240-volt house supply is rectified and stepped up to 800 volts using a DC to DC converter. This process involves using a high-frequency transformer to step up or step down the voltage. DC to DC converters are used in various applications, such as welding machines, car chargers, laptop chargers, and televisions. Modern televisions are similar to computers, with operating systems and various DC voltages required for different circuits.

00:18:05	Cycloconverters were used to drive electric motors, especially three-phase motors, by changing the input frequency to control the motor speed. Nowadays, we use variable frequency drives (VFDs) for this purpose. VFDs are more efficient and are used in electric bikes, cars, and other applications. They convert DC to three-phase AC and control the output frequency to adjust the motor speed. This is what we are expected to cover in this course. For those not specializing in Power Electronics, you need to know about inverters, converters, and the limitations of switching devices.

00:20:25	We will also cover chopper-stabilized amplifiers. In your analog electronics course, you studied the behavior of transistors in various configurations, such as common emitter, common base, and common collector. You also learned about cascode configurations and differential amplifiers. The cascode configuration provides high current gain and high-frequency response by combining the benefits of common emitter and common base configurations. The common collector configuration is useful for buffering, as it has a high input impedance and a voltage gain of less than one.

00:23:20	A chopper-stabilized amplifier is used to amplify very slowly varying, low-amplitude signals, which can be considered DC. The input signal is such that the rate of change with respect to time is approximately zero. Most amplifiers you studied in class are not capable of amplifying such signals because the frequency is too low. For example, if a signal changes its amplitude from positive to negative over a period of one day, the frequency is extremely low, and a typical common emitter amplifier with capacitive coupling will not work.

00:27:00	The reactance of the capacitors at the input and output of the common emitter amplifier will be too high, causing the signal to be lost. The only amplifier that can work in this case is an operational amplifier (op-amp). Op-amps are designed to amplify signals from DC to high frequencies. However, op-amps have problems with input offset voltage and current, which vary with temperature. These offsets can cause errors in the amplified signal, making op-amps unsuitable for amplifying very low-level, slowly varying signals.

00:34:34	To overcome these issues, we use chopper-stabilized amplifiers. These amplifiers consist of a modulator, an AC amplifier, and a demodulator. The modulator converts the low-frequency input signal to a higher frequency, which is then amplified by the AC amplifier. The demodulator converts the amplified signal back to the original low frequency. This process eliminates the problems associated with input offset voltage and current, providing a more accurate amplification of low-level, slowly varying signals.

