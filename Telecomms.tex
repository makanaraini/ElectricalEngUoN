\section*{Sample IV\footnotetext{Sample III: \textbf{22/02/2018}}}
\textbf{1.} \textbf{(a)} Briefly explain the problems associated with low level slowly varying signal amplification hence state how the problem is solved practically and the principle of operation of such a system. List the typical noises sources in a low noise DC amplifiers.
\textit{(5 marks)}

\textbf{(b)} Compare the performance of a shunt and series mosfet switch using analysis hence state the preferred application of each.
\textit{(3 marks)}

\textbf{(c)} A chopper stabilized amplifier system is designed from a modulator and demodulator both of which constitute a series-shunt mosfet choppers. The input of CSA is \textbf{10 mV} dc. The OPAMP used has a drift voltage of \textbf{0.05 mV} when referred to its input and assume sinusoidal at \textbf{0.001 Hz}. The ac amplifier has \textbf{10 nF} capacitor both at its input and output for de blocking. The feedback resistor is \textbf{1 MΩ} and input resistor to opamp is \textbf{10 kΩ}. Determine the maximum percentage error of the amplified signal at the output of the system. Take the MOSFET parameters to be R\textsubscript{dsON}=\textbf{25 Ω}, R\textsubscript{dsOFF}=\textbf{10\textsuperscript{10} Ω}. The chopping frequency is \textbf{50 kHz} square wave. Op-amp may be assumed ideal apart from suffering from offsets. Also determine signal to noise ratio at the output of the CSA.
\textit{(12 marks)}

\begin{center}\rule{0.5\linewidth}{0.5pt}\end{center}

\textbf{2.} \textbf{(a)} Compare the performances quarter square method to variable transconductance method of Multiplication hence state giving reasons which offers better performance.
\textit{(4 marks)}

\textbf{(b)} For the circuit of Fig.2, assuming all diodes and transistors are of silicon type and pin 1 is connected to +V\textsubscript{cc} =\textbf{15 V} pin 2 and 14 are connected each via \textbf{3 kΩ} to +V\textsubscript{cc} of +\textbf{15 V} and pin 3 and 13 are each connected via \textbf{12 kΩ} to \textbf{0 V}(ground) and pin 7 connected -V\textsubscript{cc}=\textbf{-15 V}, obtain the quiescent currents of the constant current sources. Also evaluate the currents through Q\textsubscript{5}, Q\textsubscript{6}, Q\textsubscript{7} \& Q\textsubscript{8} with no signal to Y and X input.
\textit{(6 marks)}

\textbf{(c)} Derive the transfer function of the pre-conditioning and the gilberts cell circuit of Fig.2 and hence determine the overall transfer function clearly stating any assumptions used.
\textit{(10 marks)}
\begin{figure}[h!]
    \centering
    \includegraphics[width=0.4\textwidth]{Matthew.jpg}
\end{figure}
\begin{center}\rule{0.5\linewidth}{0.5pt}\end{center}

\textbf{3.} \textbf{(a)} With aid of a block diagram clearly explain the principle of operation of series and shunt voltage Regulator hence state with application each is preferred.
\textit{(8 marks)}

\textbf{(b)} The regulator of Fig.3 is required to supply a nominal load of \textbf{5 A} from an input that varies from \textbf{15-40 V}. Evaluate the output voltage given that the opamp has an open loop gain of \textbf{10000} and the transistors have a V\textsubscript{be} active of \textbf{0.6 V}.
\textit{(4 marks)}

\textbf{(c)} Given that the output transistor Q\textsubscript{1} has h\textsubscript{fe}=\textbf{400}, design requirements of Q\textsubscript{1} if the opamp has a maximum output drive capability of \textbf{15 mA}. Calculate the dissipation requirements of the Zener and transistor. What is the minimum efficiency of the regulator at the maximum load current? Modify the circuit of fig.3 to incorporate short circuit protection and a variable output voltage.
\textit{(8 marks)}
\begin{figure}[h!]
    \centering
    \includegraphics[width=0.4\textwidth]{Matthew1.jpg}
\end{figure}
\begin{center}\rule{0.5\linewidth}{0.5pt}\end{center}

\textbf{4.} \textbf{(a)} Clearly explain the principle of operation of the circuit in Fig.4. Hence state whether the circuit is a boost or buck converter giving reasons.
\textit{(5 marks)}

\textbf{(b)} Why do flat screen televisions and flat screen computer monitors universally use switches mode Power supplies?
\textit{(5 marks)}
\begin{figure}[h!]
    \centering
    \includegraphics[width=0.4\textwidth]{Matthew2.jpg}
\end{figure}
\textbf{(c)} The step-down switching regulator of Fig.4 has a control unit that operates at \textbf{10 kHz}. The circuit converts \textbf{20 V} into \textbf{5 V} at a nominal current of \textbf{5 A} and a peak ripple voltage of \textbf{100 mV}. The peak-to-peak ripple current of the inductor is \textbf{0.5 A}. Assuming ideal devices determine the values of inductor L and capacitance C.
\textit{(10 marks)}