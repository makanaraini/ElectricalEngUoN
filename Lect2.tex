
\section{Lecture 2: 19/09/2024}
Low-level signals that vary very slowly with respect to time can be considered DC signals or very slowly varying DC signals. These signals operate within the DC frequency range, necessitating the use of amplifiers capable of functioning up to DC frequencies. As discussed in the third-year amplifier course, only an emitter-coupled amplifier, also known as a differential amplifier, can amplify signals all the way down to DC. In the previous class, it was demonstrated that the best example of a good DC amplifier is the operational amplifier.

\section{Operational and Instrumentation Amplifiers}
Operational amplifiers are so named because they are used to perform mathematical operations and are integral components in computers. A modified version, known as the instrumentation amplifier, is optimized for very low noise. This optimization is achieved by minimizing the noise generated by the transistors, especially those in the differential pair.

\subsection{Noise in Amplifiers}
In these types of amplifiers, various components contribute to noise generation:
\begin{itemize}
    \item \textbf{Resistors:} The resistors used to bias and couple the transistors generate noise. The type of resistor affects the noise characteristics:
    \begin{itemize}
        \item \textit{Metal Oxide Resistors:} High-quality resistors made from metal oxide generate minimal noise.
        \item \textit{Carbon Resistors:} Commonly found in shops and laboratories, these resistors generate significant noise.
    \end{itemize}
\end{itemize}
Even with the best instrumentation amplifiers, noise in a much higher frequency band than the input signal can still pose challenges.

\section{Drift in Instrumentation Amplifiers}
Instrumentation amplifiers suffer from \textit{V\textsubscript{IO}} (input offset voltage) and \textit{I\textsubscript{IO}} (input offset current) drift. These drifts are on the same order of magnitude as the signal to be amplified and are temperature-dependent. For example, in Nairobi, laboratory temperatures can range from 15°C in the morning to 22–25°C in the afternoon, affecting the output of transistors and the overall amplifier. This drift complicates the differentiation between the useful signal and noise, as the drift voltages and currents are comparable to the amplified signal.

\section{Chopper-Stabilized Amplifiers}
To address the drift issues in DC amplifiers, chopper-stabilized amplifiers offer superior temperature and drift performance. These amplifiers consist of three main modules:
\begin{enumerate}
    \item \textbf{Modulator:} Converts the slowly varying DC signal into a higher frequency amplitude-modulated signal by chopping the input signal using a switch controlled by a square or rectangular wave.
    \item \textbf{AC Amplifier (Demodulator):} Similar to a high-quality operational amplifier with capacitors that block DC, ensuring only the AC component is amplified.
    \item \textbf{Demodulator:} Converts the amplified AC signal back into a DC signal.
\end{enumerate}

\subsection{Modulation Techniques}
Common modulation techniques include:
\begin{itemize}
    \item \textit{Pulse Amplitude Modulation (PAM)}
    \item \textit{Double-Sideband (DSB) Modulation}
    \item \textit{Frequency Shift Keying (FSK)}
    \item \textit{Quadrature Phase Shift Keying (QPSK)}
\end{itemize}
For chopper amplifiers, pulse amplitude modulation is primarily used.

\subsection{Chopper Switches}
The chopper switch is critical in converting the DC signal to a modulated signal. Ideally, a chopper switch should have:
\begin{itemize}
    \item \textbf{Zero On Resistance:} To prevent loss of input voltage.
    \item \textbf{Infinite Off Resistance:} To ensure no signal passes when the switch is open.
    \item \textbf{High-Speed Switching:} To handle rapid modulation requirements.
    \item \textbf{Good Isolation:} Prevents control voltage from appearing at the output.
    \item \textbf{No Offset Voltage:} To avoid introducing errors into the output signal.
    \item \textbf{Zero Control Power:} Ideally, no power is required to activate the switch.
    \item \textbf{Infinite Current and Voltage Handling:} Can pass infinite current when on and block infinite voltage when off.
\end{itemize}
In practice, achieving these ideal characteristics is challenging, leading to the use of different types of switches.

\subsection{Types of Chopper Switches}
\subsubsection{Electromechanical Relays}
Used primarily in the early 1960s, electromechanical relays come in two types:
\begin{itemize}
    \item \textbf{Normal Relays:} Larger size, require more power, slower switching speeds (typically below 60 Hz), and prone to contact bounce, causing signal distortion.
    \item \textbf{R Relays:} Smaller, faster, require less power, but were rarely used beyond the mid-1960s due to their diminishing advantages over time.
\end{itemize}
Advantages of electromechanical relays include:
\begin{itemize}
    \item On/off characteristics close to ideal switches.
    \item No drift in output terminals.
    \item Good isolation between control voltage and output.
\end{itemize}
Disadvantages:
\begin{itemize}
    \item Low switching speeds.
    \item Contact bounce leading to signal distortion.
    \item Bulky construction and high power consumption.
    \item Not suitable for remote applications due to high energy requirements.
\end{itemize}

\subsubsection{Semiconductor Switches}
In modern applications, semiconductor switches have largely replaced electromechanical relays. Types include:
\begin{itemize}
    \item \textbf{BJT Transistors:} High switching speeds but suffer from large offset voltages (approximately 0.1 V for germanium and 0.2 V for silicon transistors), which can negate the amplified signal.
    \item \textbf{Field Effect Transistors (FETs):} Provide better control and lower noise compared to BJTs.
    \item \textbf{Metal Oxide FETs:} Offer enhanced performance characteristics.
\end{itemize}
\subsection{Challenges with BJT Switches}
Although BJTs have high switching speeds, their large offset voltages can severely impact signal integrity. For example, a 0.2 V drop across a BJT switch can nullify a 50 mV signal. Additionally, BJTs are unidirectional; they cannot effectively switch signals when polarity reverses, limiting their applicability in bidirectional signal processing.

\section{Conclusion}
Chopper-stabilized amplifiers provide a robust solution to drift and noise issues in DC amplifiers. However, selecting the appropriate switching mechanism is crucial to maintain signal integrity and performance. While electromechanical relays offered a near-ideal switching characteristic in the past, semiconductor switches have become the preferred choice despite their inherent challenges, such as offset voltages and unidirectional switching limitations.

