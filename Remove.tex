\documentclass[a4paper,9pt,twoside,openany,twocolumn]{memoir}

% Essential packages
\usepackage{amsmath,amssymb,graphicx,xcolor,tikz}
\usepackage[
    top=0.5cm,    % Even smaller top margin
    bottom=0.6cm, % Even smaller bottom margin
    left=0.8cm,   % Reduced left margin
    right=0.8cm   % Reduced right margin
]{geometry}
\usepackage{microtype}
\usepackage{caption}
\usepackage{subcaption}
\usepackage{float}
\usepackage{tabularx}
\usepackage[utf8]{inputenc}
\usepackage[T1]{fontenc}
\usepackage[hidelinks]{hyperref}
\usepackage{enumitem} % For compact list control
\usepackage[most]{tcolorbox} % For tcolorbox

% Colors
\definecolor{chaptercolor}{RGB}{0,75,150}

% Chapter Title Style
\chapterstyle{section} % Predefined style in memoir

% Headers and footers
\usepackage{fancyhdr}
\pagestyle{fancy}
\fancyhf{}
\fancyhead[LE,RO]{\thepage}
\fancyhead[RE]{\nouppercase{\leftmark}}
\fancyhead[LO]{\nouppercase{\rightmark}}
\renewcommand{\headrulewidth}{0.1pt} % Thinner header rule
% Reduced header/footer space
\setlength{\headheight}{10pt}
\setlength{\headsep}{8pt}
\setlength{\footskip}{8pt}

% Spacing Optimizations
\linespread{0.92} % More compact line spacing (9pt font size)

% Math Spacing Optimization
\setlength{\abovedisplayskip}{1.5pt}    % Reduced math spacing
\setlength{\belowdisplayskip}{1.5pt}    % Reduced math spacing
\setlength{\abovedisplayshortskip}{1pt} % Reduced math spacing
\setlength{\belowdisplayshortskip}{1pt} % Reduced math spacing
\setlength{\jot}{1pt}                  % Even smaller spacing between math lines
\setlength{\mathsurround}{0.5pt}       % Tightened math symbols

% Section and subsection spacing
\setlength{\beforechapskip}{2pt} % Reduced space before chapter title
\setlength{\afterchapskip}{2pt}  % Reduced space after chapter title
\setlength{\beforesecskip}{2pt}  % Reduced space before section title
\setlength{\aftersecskip}{1pt}   % Reduced space after section title

% Tighter lists
\setlist[itemize]{leftmargin=4mm, itemsep=0pt, topsep=1pt, parsep=0pt}
\setlist[enumerate]{leftmargin=4mm, itemsep=0pt, topsep=1pt, parsep=0pt}

% Tighten table spacing
\renewcommand{\arraystretch}{0.9}  % Reduced table row height

% Tcolorbox settings for compactness
\newtcolorbox[auto counter, number within=section]{compactbox}[2][]{colframe=blue!40!black, 
    colback=blue!5!white, 
    coltitle=black, 
    boxrule=0.5pt, 
    arc=2mm, 
    outer arc=2mm, 
    top=3pt, 
    bottom=3pt, 
    left=4pt, 
    right=4pt, 
    fonttitle=\bfseries\small, 
    font=\small, 
    #1}

% Document
\begin{document}


\chapter{Chopper Stabilized Amplifier}
This is an amplifier used to amplify very slowly varying low amplitude signals, which, to most extents and purposes, can be considered to be D.C.
Let $V_i$ such that 
\[
\frac{\partial V_i}{\partial t} \approx 0
\]

There are many applications that require amplification of extremely low-level input signals that vary very slowly with respect to time. For example:
\begin{itemize}
    \item Stress and deformation measurements using strain gauges
    \item Monitoring temperature in closed-loop control systems using thermocouples
    \item Seismic measurements
    \item Analog computers
    \item Biomedical Amplifiers
\end{itemize}

Because these signals vary very slowly with respect to time, they can be considered to be D.C. signals and hence require an amplifier whose response is down to D.C. to provide satisfactory amplification.

Practical D.C. amplifiers, such as the OPAMP, however, are not suitable since they suffer from drift, i.e. 
\[
\frac{\partial V_{io}}{\partial t} \quad \text{and} \quad \frac{\partial I_{io}}{\partial t}
\]
which are of the order of the signal to be amplified when referred to the input. The implication of this is that a processing circuit at the output of the amplifier cannot differentiate between the input signal and the signal from the offset voltage or current. $V_{io}$ is due to the imbalance of the $V_{be}$'s of the emitter-coupled transistors (Q3 and Q4), and $I_{io}$ is due to the difference in $\beta$ of the two transistors of the long tail pair.

The above two effects have been decreased substantially by careful matching of the long tail pair and by temperature control of the substrate. However, the latter technique is expensive and achieves drift voltages of $1 \, \mu V/^\circ C$ referred to the input.
A cheaper and higher-performance alternative is to use a chopper-stabilized amplifier, which provides superior temperature stability and drift perfomance.

The Chopper-Stabilized Amplifier consists of the following:
\begin{itemize}
    \item Modulator
    \item A.C. amplifier
    \item Demodulator
\end{itemize}

\subsection*{Modulator}
This converts the slowly varying D.C. input signal into a higher frequency that is amplitude-modulated, which is then amplified by the A.C. amplifier.

A simple amplitude modulator is effected by chopping the signal with a switch controlled by a square wave at the desired carrier frequency. Typically, this should be greater than the Nyquist rate. However, for reconstruction of the signal with minimum distortion, this should be 10 times greater than the Nyquist rate.

A chopped or sampled output of the signal appears at the output of the switch, hence the name "Chopper." A high-pass filter (HPF) ensures that only A.C. passes to the A.C. amplifier.

\subsection*{Chopper Switch}
The ideal chopper switch should have the following characteristics:
\begin{itemize}
    \item Zero ON resistance
    \item Infinite OFF resistance
    \item High switching speed
    \item Good isolation between the control voltage $V_c$ and the output terminals of the switch
    \item Should have no offset voltages at the output
    \item The output terminals of the switch should not suffer from drift problems
    \item Zero control power (power for activating the switch)
    \item Switch should pass an infinite current when ON and block an infinite voltage when OFF.
\end{itemize}

Practical switches are non-ideal and comprise of the following:
\subsubsection{Electromechanical relays}
\begin{itemize}
    \item Normal electromechanical relays
    \item Reed relays
\end{itemize}

\uline{Advantages:}
\begin{itemize}
    \item ON and OFF switching characteristics are close to those of the ideal
    \item Do not suffer from drift at the output terminals
    \item They do not have output offsets
    \item They have good isolation between the control voltage and the output terminals of the switch
\end{itemize}

\uline{Disadvantages:}
\begin{itemize}
    \item Due to the fact that the switches have moving parts with mass, the switching speeds are low ($< 60$ Hz for relays and $< 400$ Hz for reed relays)
    \item Slight distortion in the reconstructed signal at the demodulator may occur unless these switches are finely tuned, due to contact bounce
    \item Because of their size, it results in a bulky chopper construction
    \item Require high power to drive them, leading to high power consumption, and are not suited for applications such as meteorological stations where the power source is solar
\end{itemize}

\subsubsection*{Semiconductor Switches}
\begin{itemize}
    \item BJTs
    \item FET
    \item MOSFETs
\end{itemize}

\subsubsection*{BJTs}
Although BJTs have a high switching speed, they have large offset voltages at the output terminals of the switch on the order of 0.1V for Germanium and 0.2V for Silicon transistors. This offset is much greater than the signal to be passed through the switch (about 50mV). Transistors are not bidirectional (can only chop one way).

\subsubsection*{FETs}
FETs are the most commonly used switch due to the following advantages:
\begin{itemize}
    \item No offset voltage between drain and source
    \item High switching speed
    \item They are bidirectional, so can chop both positive and negative voltages
    \item Output drift characteristics are the best of the semiconductor devices
    \item Low ON resistance (typically 25 $\Omega$ for small signal FETs) and high OFF resistance (typically $10^{10} \Omega$ at 25°C)
    \item Small in size
    \item Low power control requirements
\end{itemize}

\uline{Disadvantages:}
\begin{itemize}
    \item Have non-ideal values for ON and OFF resistances
    \item Switching spikes may appear at the output due to incomplete isolation between the output terminal and the control voltage
\end{itemize}

\subsubsection*{MOSFETs}
MOSFETs have a metal oxide layer between the gate and channel, with a high impedance typically $10^{12} \Omega$. Drain-source characteristics are identical to FETs. MOSFETs require much less power than FETs because the gate is insulated from the channel. Of the switching devices, MOSFET implementation gives rise to the lowest power consumption in chopper-stabilized amplifiers.

\subsection*{Examples of Electromechanical Choppers and Modulators}

\begin{figure}[h]
    \centering
    \includegraphics[width=0.3\textwidth]{Electromechanical.png}
    \caption{Example of electromechanical chopper modulator}
\end{figure}

S1 and S2 are arranged such that when one is on, the other is off, and the output $V_O$ is the amplitude modulated signal.

\subsection*{A.C. Amplifier}
This is usually a D.C. amplifier (op-amp) with low drift characteristics to which a D.C. blocking capacitor has been added so as not to pass the D.C. amplifier drift to the signal. It should have adequate bandwidth to pass all the sidebands of the modulated signal.

Typical parameters of a monolithic amplifier at 25°C (e.g., 741):
\begin{itemize}
    \item Input offset voltage: 5mV
    \item Input offset current: 20nA
    \item Input bias current $I_B$: 100nA
    \item CMRR: 100dB
    \item PSRR: 20$\mu$V/V
    \item $I_{io}$ drift: 0.1nA/$^\circ$C
    \item $V_{io}$ drift: 5$\mu$V/$^\circ$C
    \item Slow rate: 1V/$\mu$S
    \item Unity gain frequency: 1MHz
    \item Full power bandwidth: 50kHz
    \item Open loop difference gain $A_d$: 100,000
    \item Open loop output resistance: 100$\Omega$
    \item Open loop input resistance: 1M$\Omega$
    \item $R_i$ for a JFET input stage: $10^{12}$ $\Omega$
\end{itemize}
\subsection*{A block diagram of a chopper stabilized amplifier system}

\begin{figure}[h]
    \centering
    \includegraphics[width=0.4\textwidth]{Abidha.png}
    \caption{A block diagram of a chopper stabilized amplifier system}
\end{figure}

The above diagram represents a complete chopper DC amplifier which may be regarded as a carrier system. The chopping frequency corresponds to the carrier frequency while the small varying DC corresponds to the chopping signal.

\section{Questions}
\begin{enumerate}
    \item Demonstrate with the aid of analysis that the circuit below can be used to implement a carrier-based chopper amplifier.
\begin{figure}[h]
    \centering
    \includegraphics[width=0.4\textwidth]{Abidha.png}
    \caption{A block diagram of a chopper stabilized amplifier system}
\end{figure}
    \item Find the gain of the chopper amplifier stabilized system.
    \item Explain how to recover the amplified signal.
\end{enumerate}



\subsection*{I. Implementation of a Simple Carrier-Based Chopper Stabilized Amplifier System}

At point (a), the signal is given by:
\[
m(t) = A \cos \omega_c t
\]
Thus, at point (b), the signal becomes:
\[
m(t) = A m(t) \cos \omega_c t
\]
which can be written as:
\[
m(t) = K \cdot A m(t) \cos \omega_c t
\]
At the output (c), the voltage $V_o$ is:
\[
V_o = A K m(t) \cos \omega_c t
\]
Expanding further, we get:
\[
V_o = A^2 K m(t) \cos^2 \omega_c t
\]
Using the trigonometric identity $\cos^2 \omega_c t = \frac{1}{2} (1 - \cos 2 \omega_c t)$, we have:
\[
V_o = A^2 K m(t) \cdot \frac{1}{2} (1 - \cos 2 \omega_c t)
\]
Thus, the output voltage becomes:
\[
V_o = \frac{A^2 K}{2} m(t) - \frac{A^2 K}{2} m(t) \cos 2 \omega_c t
\]

\subsection{Gain of the Chopper Amplifier Stabilized System}
The overall gain of the chopper stabilized amplifier system is:
\[
\text{Gain} = \frac{A^2 K}{2}
\]
To recover the output signal from the chopper stabilized system, a low-pass filter (LPF) with a cutoff frequency slightly greater than the highest component of the signal $m(t)$ should be employed at the output. The recovered signal is then:
\[
V_o(t) = \frac{A^2 K}{2} m(t) = K' m(t)
\]
where \( K' = \frac{A^2 K}{2} \).

\subsection{Note}
The above system implementation illustrates the principle of operation of a chopper stabilized amplifier. However, in practice, the analog modulator and demodulator must be designed properly to minimize drift problems inherent in multiplier implementation.

\subsection*{II. Implementation of a Chopper Stabilized Amplifier Using a Positive Gain, Negative Gain Amplifier}
\begin{figure}[h]
    \centering
    \includegraphics[width=0.4\textwidth]{Stab1.png}
    \caption{Chopper stabilized Amplifier}
\end{figure}
When the switch S is closed by a control voltage Vc (rectangular waveform) of peak amplitude 0V, the equivalent circuit is as follows
\begin{figure}[h]
    \centering
    \includegraphics[width=0.4\textwidth]{Stab2.png}
    \caption{Switch S is closed}
\end{figure}
\begin{figure}[h]
    \centering
    \includegraphics[width=0.4\textwidth]{Stab3.png}
    \caption{Switch S is closed}
\end{figure}
Given the equation for the output voltage \(V_o(t)\):
\[
V_o(t) = -R_f V_i = -R \cdot V_m(t)
\]
where \(R_i\) and \(R\) are resistances, the output voltage becomes:
\[
V_o(t) = -V_m(t)
\]

\textit{When switch \(S\) is open, the equivalent circuit is...}
\begin{figure}[h]
    \centering
    \includegraphics[width=0.4\textwidth]{Stab4.png}
    \caption{Switch S is open}
\end{figure}

Since the operational amplifier (op-amp) is operated in its linear region, the response of the output is due to the response of the input connected to its negative terminal and the input connected to its positive terminal.

The output voltage \( V_o(t) \) can be expressed as the sum of two components:
\[
V_o(t) = V_{o1}(t) + V_{o2}(t) \tag{1}
\]
where:
\[
V_{o1}(t) \quad \text{is due to} \quad V_{\text{only}} \tag{2}
\]
and
\[
V_{o2}(t) \quad \text{is due to} \quad V_{+} \ \text{only} \tag{3}
\]
\textbf{Output due to \( V_{-} \) only.}

Thus, the output due to \( V^{-} \) only can be modeled by the equivalent circuit as shown below:


\begin{figure}[h]
    \centering
    \includegraphics[width=0.3\textwidth]{Stab5.png}
    \caption{output due to $V^{-}$}
\end{figure}
\[
V_{o1}(t) = -R_f V_i = -R \cdot V_m(t)
\]
\[
= -V_m(t) \tag{4}
\]

\textbf{Output due to \( V_{+} \) only.}
\begin{figure}[h]
    \centering
    \includegraphics[width=0.3\textwidth]{Stab6.png}
    \caption{output due to $V^{+}$}
\end{figure}
\[
V_{o2}(t) = V_m(t) + I R_f
\]
\[
= V_m(t) + V_m(t) R_f
\]
\[
= V_m(t) + V_m(t) R
\]
\[
= 2 V_m(t) \tag{5a}
\]

Therefore, the output due to the switch being open is:
\[
V_o(t) = - V_m(t) + 2 V_m(t)
\]
\[
= V_m(t) \tag{5b}
\]
From equation (4), it is seen that when the switch was closed, the amplifier system multiplies the output by -1, and from equation (5b), when the switch is open, the amplifier multiplies the system by +1. Thus, this is referred to as a positive gain, negative gain amplifier.

Graphical Solution to Output Waveform of the Positive Gain, Negative Gain Amplifier
\begin{figure}[h]
    \centering
    \includegraphics[width=1.075\linewidth]{Stab7.png}
    \caption{Reconstruction}
\end{figure} 

The above circuit reconstructs the signal when switches are closed.

\[
V_{o2}(t) = -V_m(t)
\]

\[
V_{o1} = V_{o1}(t) = - R \cdot V_m(t) = V_m(t) \quad \text{(when switch is closed)}
\]

When switches are open:

\[
V_{o2}(t) = V_m(t)
\]

\[
V_o(t) = -R \cdot V_m(t) = - V_m(t)
\]

\[
V_o''(t) = V_m(t) + IR = V_m(t) + V_m(t)R = 2 V_m(t) \quad V_{o2}(t)
\]

\[
V_o''(t) = V_m(t)
\]
\section{FET Chopper: Common Source Characteristics}

The above is the characteristics that make the FET suitable as a shunt or series chopper.

As the voltages and current levels to be chopped are small, the FET operates in a region well below its pinch-off region, and this implies:
\begin{enumerate}
    \item The channel behaves as a pure resistance (current and voltage are linearly related), and the value depends on the gate-to-source voltage.
    \item If \(V_{DS}\) is reversed, the chopper still behaves as required, i.e., it can chop both positive and negative signals, provided the gate-channel junction does not become forward biased.
    \item All the lines pass through the origin, meaning the FET does not suffer from offsets.
\end{enumerate}

\subsection*{Error Analysis of FET Switch}

Three possible configurations are available for using FET as switches in a chopper. These are:
\begin{enumerate}
    \item Shunt Chopper
    \item Series Chopper
    \item Series Shunt Chopper
\end{enumerate}

\subsection*{i) Shunt Chopper}
\begin{figure}[h]
    \centering
    \includegraphics[width=0.3\textwidth]{Square1.png}
    \caption{}
\end{figure}
\textbf{Circuit:}  
In this configuration, the switching device is in parallel with the load and alternately switched ON and OFF by a gate drive voltage. When the device is in the OFF state, the generator voltage is applied to potential divider \( R_1 \) in series with \( r_{ds(\text{OFF})} \), which is in parallel with \( R_L \). When the device is in the ON state, generator voltage \( V_g \) divides between \( R_1 \) and \( r_{ds(\text{ON})} \), which is in parallel with \( R_L \). \( R_1 \) is included to protect the device from low impedances.

\subsubsection*{ON State Error for Shunt Switch}

\textbf{Equivalent Circuit during ON State:}  
\begin{figure}[h]
    \centering
    \includegraphics[width=0.3\textwidth]{Square2.png}
    \caption{}
\end{figure}
With the device ON, the output should ideally be 0, but because of finite ON-resistance \( (25 \Omega) \), the output does not fall to zero, thus generating an error at the output of the switch.

\[
\text{Error} = \text{Indicated} - \text{True (ideal)} = V_L - 0 = \varepsilon_r = \frac{R_{\text{eq}} V_g}{R_1 + R_g + R_{\text{eq}}}
\]

Where
\[
R_{\text{eq}} = \frac{r_{ds(\text{ON})} R_L}{R_L + r_{ds(\text{ON})}}
\]
and
\[
r_{ds(\text{ON})} = 25 \, \Omega
\]

Thus,
\[
\varepsilon_r = \frac{r_{ds(\text{ON})} V_g}{R_g + R_1 + r_{ds(\text{ON})}} \quad \text{(Equation 4)}
\]

For a given \( r_{ds(\text{ON})} \), the error in the ON state can be minimized by using large values of \( R_g \) and \( R_1 \). This requirement may contradict the requirement for minimum error in the OFF state.

\subsubsection*{OFF State Error}
\begin{figure}[h]
    \centering
    \includegraphics[width=0.3\textwidth]{Square3.png}
    \caption{}
\end{figure}
\textbf{Equivalent Circuit:}  
When the device is in the OFF state, the output voltage is \( V_g \). Due to series resistance \( R_g \) and \( R_1 \), not all the voltage appears at the load, generating an OFF state error.

\[
\text{Error (OFF)} = V_L - V_g = \varepsilon_r = \frac{V_g R_{\text{eq}}}{R_1 + R_g + R_{\text{eq}}}
\]

Where
\[
R_{\text{eq}} = \frac{r_{ds(\text{OFF})} R_L}{r_{ds(\text{OFF})} + R_L}
\]
and
\[
r_{ds(\text{OFF})} = 10^{10} \, \Omega
\]

Thus, the OFF state error becomes:
\[
\varepsilon_r = -V_g \frac{1 - \frac{R_{\text{eq}}}{R_g + R_1 + R_{\text{eq}}}}{1 + \frac{R_{\text{eq}}}{R_g + R_1}}
\]

For practical cases, the error is minimized by choosing values of \( R_1 \) and \( R_L \) that balance the ON and OFF state requirements:
\[
R_1^{\text{opt}} = \sqrt{r_{ds(\text{ON})} R_L}
\]

\subsection*{Other Non-quantifiable Errors: Feed-through Spikes}
\begin{figure}[h]
    \centering
    \includegraphics[width=0.4\textwidth]{Square4.png}
    \caption{}
\end{figure}
\textbf{Equivalent Circuit:}  
Feed-through spikes are caused by inter-electrode capacitances and can be minimized by:
\begin{enumerate}
    \item Reducing the rate of change of control voltage
    \item Reducing load and generator resistances
    \item Using sinusoidal pulse drive
\end{enumerate}

The effect of feed-through spikes is to generate a new offset error and may even overload the AC amplifier if their magnitude is large. The amplitude of the ON and OFF state spikes is generally similar, but the energy in the OFF spike is larger, contributing more significantly to the offset voltage at the output.

\subsection*{ii) Series Chopper}
\begin{figure}[h]
    \centering
    \begin{minipage}{0.4\textwidth}
        \centering
        \includegraphics[width=0.7\textwidth]{Square5.png}
        \caption{}
    \end{minipage}%
    \hfill
    \begin{minipage}{0.4\textwidth}
        \centering
        \includegraphics[width=0.7\textwidth]{Square6.png}
        \caption{}
    \end{minipage}
\end{figure}
The switching device is connected in series with the signal and load, thereby connecting and disconnecting \( R_L \) to the generator. The gate drive switches the gate from ground to a negative potential greater than the pinch-off voltage (when the device is OFF).

\subsubsection*{Analysis of OFF State Error}
\begin{figure}[h]
    \centering
    \includegraphics[width=0.3\textwidth]{Square7.png}
    \caption{}
\end{figure}
When the device is OFF, the output voltage should ideally be 0, but due to finite \( r_{ds(\text{OFF})} \), the output error is given by:
\[
\varepsilon_r = \frac{V_L}{R_g + r_{ds(\text{OFF})} + R_L}
\]

For \( r_{ds(\text{OFF})} \gg R_L \), the error is large for low \( R_L \) values. This is especially true if the FET drives a source follower or buffer circuit.

\subsubsection*{ON State Error Analysis}

The error is given by:
\[
\varepsilon_r = \frac{V_L}{R_g + r_{ds(\text{ON})} + R_L}
\]

For minimal error, \( r_{ds(\text{ON})} \) should be small, and \( R_L \) should be large, but this conflicts with the OFF state error requirements.

\subsection*{Effects of Feed-through Spikes in the Series Chopper}
\begin{figure}[h]
    \centering
    \includegraphics[width=0.4\textwidth]{Square8.png}
    \caption{}
\end{figure}
These effects depend mainly on the capacitances between the gate and the output, the load resistance, the generator resistance, and the amplitude of the drive signal. When the FET is used as a series chopper, the amplitude of feed-through spikes is influenced mainly by the generator resistance.

\subsection*{Practical Values of Error for the Series Chopper}

For various values of \( R_L \), the errors are calculated as follows:

\[
\varepsilon_r = \frac{V_g R_L}{R_g + r_{ds(\text{OFF})}}
\]
\[
\varepsilon_r = \frac{-V_g (R_g + r_{ds(\text{ON})})}{R_g + r_{ds(\text{ON})} + R_L}
\]

\textbf{Example:}

For \( R_g = 0 \) (low resistance source), \( R_L = 2 \, k\Omega \), \( r_{ds(\text{OFF})} = 10^{10} \, \Omega \), and \( r_{ds(\text{ON})} = 25 \, \Omega \):

\[
\varepsilon_r = -\frac{V_g (R_g + r_{ds(\text{ON})})}{R_g + r_{ds(\text{ON})} + R_L} = -\frac{V_g (25)}{2025} = -1.23\%
\]

For other values of \( R_L \), similar calculations can be made. The results are tabulated below:

\[
\begin{array}{|c|c|c|}
\hline
R_L (\Omega) & \text{ON Error (\%)} & \text{OFF Error (\%)} \\
\hline
2K & 1.23 & 2 \times 10^{-5} \\
10K & 0.249 & 1 \times 10^{-4} \\
50K & 0.0498 & 5 \times 10^{-3} \\
100K & 0.0249 & 1 \times 10^{-3} \\
\hline
\end{array}
\]
\section*{Exercise}

Solve for $R_L$ when the two errors are equal.

\subsection*{Practical values of errors in a shunt switch.}

\begin{itemize}
    \item $R_g = 0$, $r_{ds(\text{ON})} = 25\Omega$, $R_1 = 158\Omega$, $r_{ds(\text{OFF})} = 10\,10\Omega$
\end{itemize}

\subsubsection*{OFF state error:}
\[
\epsilon_r \% = \frac{\Delta V}{V_g} \times 100\% = \frac{-V_g}{158 + 2000} \times 100\% = -7.3\%
\]

For various $R_L$ values:
\begin{itemize}
    \item $R_L = 10k\Omega$, $\epsilon_r \% = 1.55\%$
    \item $R_L = 50k\Omega$, $\epsilon_r \% = 0.315\%$
    \item $R_L = 100k\Omega$, $\epsilon_r \% = 0.157\%$
\end{itemize}

\subsubsection*{ON state error:}
\[
\epsilon_r \% = \frac{r_{ds(\text{ON})}}{R_1 + R_g + r_{ds(\text{ON})}} \times \frac{V_g}{V_g} = \frac{25}{158 + 25} \times 100\% = 13.66\%
\]

For various $R_L$ values:
\begin{itemize}
    \item $R_L = 10k\Omega$, $\epsilon_r \% = 13.66\%$
    \item $R_L = 50k\Omega$, $\epsilon_r \% = 13.66\%$
\end{itemize}

\subsection*{Table 2: Errors for combined series \& shunt choppers}

\begin{table}[h!]
\centering
\setlength{\tabcolsep}{2pt} % Adjust column separation
\renewcommand{\arraystretch}{1.2} % Adjust row separation for better readability
\begin{tabularx}{\columnwidth}{|c|X|X|X|X|}
\hline
$R_L$ (\(\Omega\)) & $2K$ & $10K$ & $50K$ & $100K$ \\
\hline
\multicolumn{5}{|c|}{Shunt ON state error \%} \\
\hline
ON state error \% & 13.66 & 13.66 & 13.66 & 13.66 \\
OFF state error \% & 7.3  & 1.55 & 0.315 & 0.157 \\
\hline
\multicolumn{5}{|c|}{Series ON state error \%} \\
\hline
ON state error \% & 1.23 & 0.249 & 0.0498 & 0.0249 \\
OFF state error \% & $2 \times 10^{-5}$ & $1 \times 10^{-4}$ & $5 \times 10^{-3}$ & $1 \times 10^{-3}$ \\
\hline
\end{tabularx}
\caption{Error percentages for different $R_L$ values.}
\label{tab:error_percentages}
\end{table}


From the comparison of the two tables, it is seen that the errors in the shunt chopper circuit when driven from a low impedance source ($R_g = 0$) are much higher as compared to those from the series chopper.

\subsubsection*{The series-shunt chopper}
\begin{figure}[h]
    \centering
    \includegraphics[width=0.3\textwidth]{Square9.png}
    \caption{}
\end{figure}

In both the series and shunt chopper circuits, a compromise has to be made on the circuit values in order to minimize the errors due to the ON \& OFF states.

For the shunt chopper, $R_1$ (protection resistance) should be high when the device is in the ON state and vice versa for minimum error. For the series chopper, $R_L$ should be low when the device is in the OFF state and high when it’s in the ON state. A simple way to do away with the above requirements is to use a series-shunt chopper.

In the series-shunt chopper, two FETs are used, one in series and the other in shunt with the signal, and they are alternately switched (i.e., when one is ON, the other is OFF).

\subsection*{Error due to series-shunt chopper}

\subsubsection*{Case 1: When series device is ON \& shunt device OFF}
\begin{figure}[h]
    \centering
    \includegraphics[width=0.5\linewidth]{Square10.png}
    \caption{}
\end{figure}
For this case, ideally, all the voltage from the generator should appear at the load, but

\[
\epsilon_r = \text{Indicated} - \text{True} = V_L - V_g
\]

\[
= V_g \frac{R_{\text{eq}}}{R_g + r_{ds(\text{ON})} + R_{\text{eq}}} - V_g
\]

\[
= -V_g \left( \frac{R_g + r_{ds(\text{ON})}}{R_g + r_{ds(\text{ON})} + R_{\text{eq}}} \right)
\]

Since $R_{\text{eq}} = R_L \parallel r_{ds(\text{OFF})} \approx R_L$, we have

\[
\epsilon_r = -V_g \left( \frac{1}{1 + \frac{R_L}{R_g + r_{ds(\text{ON})}}} \right)
\]

To minimize error, $R_g$ and $r_{ds(\text{ON})}$ should be small and $R_L$ large.

\subsubsection*{Case 2: When series device is OFF \& shunt device ON}
For this case, ideally, $V_L = 0$, but due to non-ideal switching devices, an error voltage appears at the output given by
\begin{figure}[h]
    \centering
    \includegraphics[width=0.3\textwidth]{Square11.png}
    \caption{}
\end{figure}
\[
\epsilon_r = V_L = V_g \frac{r_{ds(\text{ON})}}{R_g + r_{ds(\text{OFF})} + r_{ds(\text{ON})}}
\]

The error is minimum when $R_{\text{eq}} \to 0$, i.e., $r_{ds(\text{OFF})}$ must be high, and $R_L \parallel r_{ds(\text{ON})}$ must be low. Therefore, for the best performance, $r_{ds(\text{ON})}$ should be low, and $R_L$ should be high.

\subsection*{Problems}

\subsubsection*{i) Compute the errors for the below 3 configurations:}
Given that $R_g = 1K\Omega$, $R_L = 1K\Omega$, $r_{ds(\text{OFF})} = 10\,10\Omega$, and $r_{ds(\text{ON})} = 25\Omega$.

\subsubsection*{ii) Repeat for $R_g = 100K\Omega$, $R_L = 1K\Omega$, $r_{ds(\text{OFF})} = 10\,10\Omega$, and $r_{ds(\text{ON})} = 25\Omega$.}

\subsubsection*{iii) Repeat for $R_g = 100K\Omega$, $R_L = 100K\Omega$, $r_{ds(\text{OFF})} = 10\,10\Omega$, and $r_{ds(\text{ON})} = 25\Omega$.}

\subsubsection*{Practical series-shunt chopper}
\begin{figure}[h]
    \centering
    \includegraphics[width=1.1\linewidth]{Square12.png}
    \caption{}
\end{figure}
\begin{figure}[h]
    \centering
    \includegraphics[width=0.49\paperwidth]{Square13.png}
    \caption{}
\end{figure}

Compute the effective error due to switching for the 2 circuits where $f_c = 400\,\text{Hz}$ and $t_r = 200\,\text{ns}$.

Another implementation of the series-shunt chopper is to use a complementary series-shunt chopper where complementary devices are used. This is useful when antiphase voltages for driving series-shunt chopper are not available.

\begin{figure}[h]
    \centering
    \includegraphics[width=\linewidth]{Square14.png}
    \caption{}
\end{figure}

From the results of the performance of the 3 configurations, it is seen that:
\begin{enumerate}
    \item When load resistance is low and switching speed is not important, the most convenient configuration is the series chopper because of its low net error and due to small components used, hence low cost.
    \item When load resistance is high and switching speeds and feed-through spikes are important for low input signals, then the \textbf{series-shunt} chopper offers the best performance.
    \item The shunt chopper produces relatively large errors although the spike performance is less than that of the series chopper. It has a higher switching capability compared to the series chopper (series chopper not suitable for driving capacitive loads).
\end{enumerate}

\section*{Questions}

\begin{enumerate}
    \item Define a chopper stabilized amplifier stating its method of operation and the application it is well suited for, giving examples.
    \item Distinguish between the series, series-shunt, and shunt chopper, and state the criterion used to select a particular configuration for a particular operation.
    \item The figure below(figure 1.25) is a chopper stabilized amplifier with an input signal $V_i = 20mV$ d.c. The maximum drift input voltage $V_{io} = 0.001 \cos(2\pi f t)$ volts assumed to be sinusoidal when referred to the input. For the circuit, 


\begin{enumerate}
    \item[(i)] Determine the drift voltage magnitude at the output point \( Y \) if the input to the A.C. shunt amplifying circuit is \( 10 \, \text{dB} \).
    
    \item[(ii)] The maximum error of the amplitude of the signal at the output of the chopper-stabilized amplifier system, given the following parameters:
    \begin{itemize}
        \item \( r_{\text{ds(ON)}} = 25 \, \Omega \)
        \item \( r_{\text{ds(OFF)}} = 10^{10} \, \Omega \)
        \item Chopping frequency \( f_c = 100 \, \text{kHz} \)
    \end{itemize}
    (6 Marks)

    \item[(iii)] State what can be done to improve the performance of the circuit.
\end{enumerate}

\end{enumerate}
\begin{figure}[h]
    \centering
    \includegraphics[width=\textwidth]{Square15.png}
    \caption{}
\end{figure} 


\chapter{Logarithmic Amplifiers}

The output voltage \( V_o \) of a logarithmic amplifier is related to the input voltage \( V_{in} \) as:

\[
V_o = K \log V_{in}
\]

Where \( K \) is a constant.

Alternatively, the input voltage and output voltage are related by:

\[
e^{\frac{V_o}{K}} = V_{in}
\]

\subsection*{Practical Applications}
\begin{enumerate}
    \item \textbf{Analog computation:} Logarithmic amplifiers are widely used in analog computation due to their ability to handle exponential relationships.
    \item \textbf{Compressors:} In audio signal processing, logarithmic amplifiers are used in compressors to manage dynamic range in applications such as recording studios and public addressing systems.
    \item \textbf{Driving display devices:} Logarithmic amplifiers are often used to drive display devices that show logarithmic scales, such as sound level meters.
    \item \textbf{Log display indicators:} These amplifiers can be used in devices that need to display logarithmic quantities, such as light meters and radiation detectors.
\end{enumerate}

\subsection*{Theory of the Basic Diode Log Amplifier}
\begin{figure}[H]
    \centering
    \includegraphics[width=0.25\textwidth]{BasicLog.png}
    \caption{Basic diode Log Amp}
    \label{fig:question_image}
\end{figure}

For the diode, the current \( I_f \) is given by:

\[
I_f = I_o \left( e^{\frac{V_f}{\eta V_T}} - 1 \right) \tag{1}
\]

Where:
- \( I_o \) is the saturation current.
- \( V_f \) is the forward voltage across the diode.
- \( \eta \) is the ideality factor (typically around 1).
- \( V_T \) is the thermal voltage.

Assuming \( \frac{V_f}{\eta V_T} \gg 1 \), we can approximate:

\[
I_f = I_o e^{\frac{V_f}{\eta V_T}} \tag{2}
\]

For the circuit, the current \( I_f \) is also related to the voltage \( V_s \) and resistance \( R_i \) by:

\[
I_f = \frac{V_s}{R_i} \tag{3}
\]

Substituting equation (3) into equation (2):

\[
\frac{V_s}{R_i} = I_o e^{\frac{V_f}{\eta V_T}} \tag{4}
\]

Relating the forward voltage \( V_f \) to the output voltage \( V_o \), we have:

\[
V_f = -V_o \tag{5}
\]

Substituting equation (5) into equation (4):

\[
\frac{V_s}{R_i} = I_o e^{-\frac{V_o}{\eta V_T}} \tag{6}
\]

Finally, solving for \( V_s \), we get:

\[
V_s = I_o R_i e^{-\frac{V_o}{\eta V_T}} \tag{7}
\]
Taking the logarithm of both sides of equation (7), we get:

\[
\ln(V_s) = -\frac{V_o}{\eta V_T} \tag{8}
\]

Substituting the expression for \( V_s \), we have:

\[
V_o = -\eta V_T \ln \left( \frac{V_s}{I_o R_i} \right) \tag{9}
\]

Thus, the output voltage \( V_o \) is:

\[
V_o = K_1 \ln(K_2 V_s) \tag{10}
\]

Where:
- \( K_1 = -\eta V_T \)
- \( K_2 = \frac{1}{I_o R_i} \)

From equation (10), the basic diode log amplifier functions as a logarithmic amplifier. However, the accuracy of this circuit depends on the relationship between \( V_f \) and \( I_f \). It is found that this accuracy diminishes as \( I_f \) increases. To improve performance, the circuit must be used with small values of \( I_f \).

The practical realization of the basic log amplifier is achieved with a transistor log amplifier, where the base-emitter junction is used as the diode.
\begin{figure}[H]
    \centering
    \includegraphics[width=0.25\textwidth]{Shegone.png}
    \caption{Basic transistor Log Amp}
    \label{fig:question_image}
\end{figure}
Using Eber's Moll relations:

\[
I_e = I_o \left( e^{\frac{V_{be}}{\eta V_T}} - 1 \right) \tag{1}
\]

For \( \frac{V_{be}}{\eta V_T} \gg 1 \), we have:

\[
I_e = I_o e^{\frac{V_{be}}{\eta V_T}} \tag{2}
\]

For the transistor, \( I_c \approx I_e \), and from the circuit:

\[
I_c = I_1 \tag{3}
\]

From the equation \( I_1 = \frac{V_s}{R_i} \), we substitute into equation (2):

\[
I_1 = I_c = I_e = I_o e^{\frac{V_{be}}{\eta V_T}} \tag{5}
\]

Relating \( V_{be} \) to \( V_o \), we have:

\[
V_{be} = -V_o \tag{6}
\]

Substituting equation (6) into equation (5), we get:

\[
V_s = I_o e^{-\frac{V_o}{\eta V_T}} \tag{7}
\]
\[
R_i
\]
\[
V_s = \frac{e^{-V_o / \eta V_T}}{I_o R_i} \tag{8}
\]

Taking the natural logarithm of both sides:

\[
\ln V_s = -\frac{V_o}{\eta V_T} \tag{9}
\]
\[
I_o R_i
\]

Rearranging for \( V_o \):

\[
V_o = -\eta V_T \ln \left( \frac{V_s}{I_o R_i} \right) \tag{10}
\]

This can be rewritten as:

\[
V_o = K_1 \ln K V_s \tag{11}
\]

Where:
- \( \eta = 1 \) for Germanium
- \( \eta = 2 \) for Silicon
- \( V_T = \frac{kT}{q} \approx 0.0257 \, \text{V} \approx 26 \, \text{mV} \)

Thus,

\[
V_o = -0.026 \ln \left( \frac{V_s}{I_o R_i} \right) \tag{12}
\]

Using \( \ln(x) = 2.303 \log_{10}(x) \), we can rewrite this as:

\[
V_o = -0.026 \times 2.303 \ln V_s \tag{13}
\]
\[
I_o R_i
\]

\[
V_o = -0.05986 \ln \left( \frac{V_s}{I_o R_i} \right) \tag{13}
\]


\subsection*{Explanation:}

\begin{itemize}
    \item Equation (8) relates the input voltage \( V_s \) to the output voltage \( V_o \) using the reverse saturation current \( I_o \) and resistance \( R_i \).
    \item In equation (9), the natural logarithm is applied to both sides.
    \item Equation (10) expresses the output voltage \( V_o \) in terms of the logarithm of \( V_s \), \( I_o \), and \( R_i \).
    \item Equation (11) is a rearranged form of the previous equation.
    \item Equation (12) gives a more specific expression for \( V_o \) using a typical value for \( V_T \).
    \item Equation (13) introduces a conversion from the natural logarithm to the base-10 logarithm for practical computation.
\end{itemize}

\[
V_o = -0.05986 \ln \left( \frac{V_s}{I_o R_i} \right) \tag{13}
\]

\subsection*{Discussion on Temperature Dependency:}

\begin{itemize}
    \item From equation (12) or (13), we see that the output \( V_o \) depends on \( I_o \), the reverse saturation current, which is temperature-dependent. Typically, \( I_o \) doubles for every 10°C increase in temperature, which can cause large errors if the ambient temperature increases.
    \item A solution to this problem is to design a matched transistor or matched diode log amplifier, which compensates for temperature variations and reduces the influence of \( I_o \).
\end{itemize}
\begin{figure}[H]
    \centering
    \includegraphics[width=\linewidth]{MatchedAmp.png}
    \caption{Matched Transistor Log Amp}
    \label{fig:question_image}
\end{figure}
Monolithic fabrication of $Q_1$, $Q_2$ for same temperature coupling.
\begin{figure}[H]
    \centering
    \includegraphics[width=0.25\textwidth]{DiffAmp.png}
    \caption{Considering first the difference amplifie}
    \label{fig:question_image}
\end{figure}
Since A3 is operated in linear region, we apply superposition theorem i.e. net response due to sum of individual responses
\begin{figure}[H]
    \centering
    \includegraphics[width=0.25\textwidth]{DueAmp.png}
    \caption{Output due to $V_a$ only}
    \label{fig:question_image}
\end{figure}


\[
V_{o1} = -\frac{R_f}{R_i} V_i \tag{1}
\]
\[
V_o = A_R V_a \tag{2}
\]
\[
V_o = - A V_a \tag{3}
\]
\begin{figure}[H]
    \centering
    \includegraphics[width=0.4\linewidth]{MarkAmp.png}
    \caption{Output due to $V_b$ only}
    \label{fig:question_image}
\end{figure}
\[
V_{o2} = V_P + \frac{V_P A_R}{R} \tag{4}
\]
\[
V_{o2} = V_P \left( 1 + \frac{A_R}{R} \right) \tag{4a}
\]
\[
V_{o2} = V_P \left( 1 + A \right) \tag{5}
\]
\[
V_P = \frac{V_b A_R}{R + A_R} = \frac{V_b A}{R(1 + A)} \tag{6}
\]

Substituting equation (6) in equation (5):

\[
V_{o2} = \frac{V_b A (1 + A)}{1 + A} = A V_b \tag{7}
\]

Therefore, the total output response is:

\[
V_{oT} = V_{o1} + V_{o2} = - A V_a + A V_b = A (V_a - V_b) \tag{8}
\]

The output \( V_a \) in equation (8) is the output of the basic transistor log amplifier comprising of \( A_1 \), while \( V_b \) is the output of \( A_2 \).

From earlier derived results for the basic transistor log amplifier:

\[
V_a = - \eta V_T \ln \left( \frac{V_{s1}}{I_{o1} R_s} \right) \tag{9}
\]

\[
V_b = - \eta V_T \ln \left( \frac{V_{s2}}{I_{o2} R_s} \right) \tag{10}
\]

Substituting equations (9) and (10) into equation (8):

\[
V_{oT} = A \left( - \eta V_T \ln \frac{V_{s2}}{I_{o2} R_s} + \eta V_T \ln \frac{V_{s1}}{I_{o1} R_s} \right) \tag{11}
\]

\[
V_{oT} = A \eta V_T \ln \left( \frac{V_{s1} I_{o2} R_s}{V_{s2} I_{o1} R_s} \right) \tag{12}
\]

\[
V_{oT} = A \eta V_T \ln \left( \frac{V_{s1}}{V_{s2}} \right) \tag{13}
\]

From equation (13), if \( Q_1 \) and \( Q_2 \) are matched, the output is independent of the reverse saturation current \( I_o \), and hence gives a better performance than the basic transistor log amplifier. However, the output is still dependent on \( V_T \). So long as \( V_T \) is constant, the output will be correct.
\begin{figure}[H]
    \centering
    \includegraphics[width=1.1\linewidth]{TransitLife.png}
    \caption{Practical logarithmic amplifier using matched transistors}
    \label{fig:question_image}
\end{figure}


\[
V_o = V + \frac{V R_4}{R_3} \tag{1}
\]
\[
V_o = V \left( 1 + \frac{R_4}{R_3} \right) \tag{2}
\]

\[
V_o = V \frac{R_3 + R_4}{R_3} \tag{3}
\]

To determine \( V \), apply Kirchhoff's Voltage Law (KVL) along loop \( a \) \( b \) \( c \):

\[
-V_{be1} + V_{be2} = V \tag{1a}
\]

Also, from Eber's Moll relation for Q1 and Q2:

\[
I_c = I_o \left( e^{\frac{V_{be}}{\eta V_T}} - 1 \right) \tag{2}
\]

For \( \frac{V_{be}}{\eta V_T} \gg 1 \), we approximate:

\[
I_c \approx I_o e^{\frac{V_{be}}{\eta V_T}} \tag{3}
\]

\[
I_{c1} = I_{o1} e^{\frac{V_{be1}}{\eta V_T}} \tag{4}
\]

Taking the natural logarithm:

\[
\ln I_{c1} = \frac{V_{be1}}{\eta V_T} \tag{5}
\]

Thus:

\[
V_{be1} = \eta V_T \ln \left( \frac{I_{c1}}{I_{o1}} \right) \tag{6}
\]

Similarly:

\[
V_{be2} = \eta V_T \ln \left( \frac{I_{c2}}{I_{o2}} \right) \tag{7}
\]

Substituting equations (6) and (7) into equation (1a):

\[
-\eta V_T \ln \left( \frac{I_{c1}}{I_{o1}} \right) + \eta V_T \ln \left( \frac{I_{c2}}{I_{o2}} \right) \tag{8}
\]

\[
\eta V_T \ln \left( \frac{I_{c2}}{I_{o2}} \right) \cdot \frac{I_{o1}}{I_{o2}} = V \tag{9}
\]

For matched transistors \( Q1 \) and \( Q2 \), \( I_{o1} = I_{o2} \):

\[
V = \eta V_T \ln \left( \frac{I_{c2}}{I_{c1}} \right) \tag{10}
\]

\[
V = -\eta V_T \ln \left( \frac{I_{c1}}{I_{c2}} \right) \tag{11}
\]

Now, for the collector currents:

\[
I_{c1} = \frac{V_s}{R_1} \tag{12}
\]

\[
I_{c2} = \frac{V_R}{R_2} \tag{13}
\]

Thus:

\[
V = -\eta V_T \ln \left( \frac{V_s R_2}{V_R R_1} \right) \tag{14}
\]

From equation (1):

\[
V_o = V \frac{R_3 + R_4}{R_3} \tag{15}
\]

Substituting for \( V \):

\[
V_o = -\frac{R_3 + R_4}{R_3} \eta V_T \ln \left( \frac{V_s R_2}{V_R R_1} \right) \tag{16}
\]

Let \( K_1 = \eta V_T \) and \( K_2 = \frac{R_2}{R_1 V_R} \):

\[
V_o = -K_1 \ln \left( K_2 V_s \right) \tag{17}
\]

Substituting the values for the resistances:

\[
\eta = 1, \quad R_3 + R_4 = 0.5 + 29.5 = 60, \quad R_3 = 0.5
\]

\[
V_o = -0.0259 \times 60 \ln \left( V_s \times 20k \right) \tag{18}
\]

\[
V_o = -1.554 \ln \left( 0.2 V_s \right) \tag{19}
\]

\[
V_o = -3.5788 \log_{10} \left( 0.2 V_s \right) \tag{20}
\]

For the above practical circuit, the input dynamic range is typically from 2 mV to 20 V, and the output is between -12 V and -2.5 V.

For the circuit to be used practically, it must be calibrated. The following steps are recommended:

\begin{enumerate}
    \item Set \( V_s = 0 \), and adjust the potentiometer \( P_1 \) until \( V' = 0 \), as \( V' = A_d (V_d) \).
    \item Set \( V_s = V_R \frac{R_1}{R_2} \), implying:

    \[
    V_o = -V_T \frac{R_3 + R_4}{R_3} \ln \left( \frac{V_R R_1 R_2}{R_3 R_2 R_1 V_R} \right)
    \]

    If \( V_o \neq 0 \), adjust \( P_2 \) until \( V_o = 0 \).
\end{enumerate}

From equation (16), it is seen that \( V_o \) is temperature-dependent (dependent on \( V_T \)). If \( \frac{R_4}{R_3} \) can be made to decrease with temperature by using a positive temperature coefficient for \( R_3 \) (so that \( R_3 \) increases with temperature), then by suitable selection of values for \( R_3 \) and \( R_4 \), the entire expression for \( V_o \) can be made temperature-independent.

\[
V_o = -K_T \left( 1 + \frac{R_4}{R_3 (1 + \alpha_T T)} \right) \ln \left( \frac{V_s}{R_1 V_R} \right) \tag{21}
\]

\textbf{Question:} Design for \( V_o \) such that \( V_o \) is independent of \( T \). 

\[
\frac{\partial V_o}{\partial T} = 0 \tag{22}
\]

\begin{figure}[H]
    \centering
    \includegraphics[width=1.08\linewidth]{Psalm.png}
    \caption{An improved Matched Transistor Log Amp with constant current source}
    \label{fig:question_image}
\end{figure}

D1 and D2 are protection diodes against large reverse bias of the B-E junction of \( Q_1 \) and \( Q_2 \). 

If \( V^+ \) of \( A_2 \) is equal to \( V \), this implies that \( V^- \) of \( A_2 \) is also \( V \). Therefore, the output voltage \( V_o \) is:

\[
V_o = V + I R_3 \tag{1}
\]

\[
V_o = V + \frac{V R_3}{R_T + R_4} \tag{2}
\]

\[
V_o = V \left( 1 + \frac{R_3}{R_T + R_4} \right) \tag{3}
\]

\[
V_o = V \frac{R_T + R_4 + R_3}{R_T + R_4} \tag{4}
\]

Considering the reference current \( I_{\text{ref}} \):

\[
I_{\text{ref}} = \frac{V_z}{R_5} \tag{5}
\]

If \( h_{FE}(Q_2) \gg 1 \), i.e., \( I_{b2} \approx 0 \) and zero bias current for \( A_2 \), then:

\[
I_{c2} \approx I_{\text{ref}} \tag{6}
\]

Apply Kirchhoff’s Voltage Law (KVL) to loop \( abc \):

\[
-V_{be1} + V_{be2} = V \tag{7}
\]

From Eber’s Moll relation for \( Q_1 \) and \( Q_2 \):

\[
I_{c1} = I_{o1} \left( e^{\frac{V_{be1}}{\eta V_T}} - 1 \right) \tag{8}
\]

For large \( \frac{V_{be1}}{\eta V_T} \), we approximate:

\[
I_{c1} \approx I_{o1} e^{\frac{V_{be1}}{\eta V_T}} \tag{9}
\]

Thus:

\[
\ln I_{c1} = \frac{V_{be1}}{\eta V_T} \tag{10}
\]

\[
\eta V_T \ln I_{c1} = V_{be1} \tag{11}
\]

Similarly:

\[
\eta V_T \ln I_{c2} = V_{be2} \tag{12}
\]

Substituting equations (11) and (12) into equation (7):

\[
\eta V_T \ln I_{c1} + \eta V_T \ln I_{c2} = V \tag{13}
\]

\[
\eta V_T \ln I_{c2} \times \frac{I_{o1}}{I_{o2}} = V \tag{14}
\]

For matched transistors \( Q_1 \) and \( Q_2 \), \( I_{o1} = I_{o2} \), so:

\[
V = \eta V_T \ln \left( \frac{I_{c2}}{I_{c1}} \right) \tag{15}
\]

\[
V = -\eta V_T \ln \left( \frac{I_{c1}}{I_{c2}} \right) \tag{16}
\]

From equation (4) and substituting into equation (16):

\[
V_o = V \frac{R_T + R_4 + R_3}{R_T + R_4} \tag{17}
\]

\[
V_o = -\eta V_T \ln \left( \frac{I_{c1} \left( R_T + R_4 + R_3 \right)}{I_{c2} \left( R_T + R_4 \right)} \right) \tag{18}
\]

Substitute for \( I_{c1} = \frac{V_s}{R_1} \) and \( I_{c2} = I_{\text{ref}} = \frac{V_z}{R_5} \):

\[
V_o = -\frac{R_T + R_4 + R_3}{R_T + R_4} \eta V_T \ln \left( \frac{V_s}{R_1} \times \frac{R_5}{V_z} \right) \tag{19}
\]

This concludes the calculation for \( V_o \) in terms of the reference voltage \( V_z \) and the various circuit components.

\chapter{ANTILOG AMPLIFIER (EXPONENTIAL AMPLIFIER)}
This is an amplifier whose output is of the form:

\[
V_o = K_1 e^{K_2 V_s}
\]

\[
V_o = e^{K_2 V_s} K_1
\]

\[
K_3 V_o = e^{K_2 V_s}
\]

\[
\ln(K_3 V_o) = K_2 V_s
\]

\[
K_3 V_o = \ln^{-1}(K_2 V_s)
\]

\[
V_o = K_4 \ln^{-1}(K_2 V_s)
\]

\subsection*{Applications}

\begin{enumerate}
    \item Expansion of compressed signals
    \item Analog computation
\end{enumerate}
\begin{figure}[H]
    \centering
    \includegraphics[width=0.25\textwidth]{Accurate.png}
    \caption{Basic Diode Antilog Amp}
    \label{fig:question_image}
\end{figure}

The output voltage of the amplifier is given by:

\[
V_o = - I_1 R_f \tag{1}
\]

The diode current is given by:

\[
I_D = I_o \left( e^{\frac{V_f}{\eta V_T}} - 1 \right) \tag{2}
\]

For large \( V_f \), we approximate as:

\[
I_D \approx I_o e^{\frac{V_f}{\eta V_T}} \quad \text{where} \quad \frac{V_f}{\eta V_T} \gg 1 \tag{3}
\]

Substituting equation (3) into equation (1), we get:

\[
V_o = - I_o R_f e^{\frac{V_f}{\eta V_T}}
\]

Since \( V_s = V_f \), we substitute to obtain:

\[
V_o = - I_o R_f \frac{e^{\frac{V_s}{\eta V_T}}}{\eta V_T}
\]

Finally, we express the output voltage in terms of the general antilog form:

\[
V_o = K_1 e^{K_2 V_s}
\]

For proper operation of the circuit, \( V_s \) must be zero, i.e., the circuit computes the antilog for positive values of \( V_s \). The maximum signal should be \( V_{Dsat} \).

\section*{Improved Antilog Amplifier Circuit}
\begin{figure}[H]
    \centering
    \includegraphics[width=\linewidth]{Antilog.png}
    \caption{Matched diode antilog amplifier}
    \label{fig:question_image}
\end{figure}
The above circuit is an improvement on the basic diode antilog amplifier as it eliminates most of the limitations of the former, i.e., variation of \( V_o \) with \( I_o \) and the limitation of \( V_s < V_{Dsat} \).

Considering loop abc and applying Kirchhoff's Voltage Law (KVL):

\[
- V_{f2} + V_{f1} = V_1 \tag{1}
\]

The diode current is given by:

\[
I_D = I_o \left( e^{\frac{V_f}{\eta V_T}} - 1 \right) \tag{2}
\]

For large \( V_f \), we approximate as:

\[
I_D \approx I_o e^{\frac{V_f}{\eta V_T}} \tag{3}
\]

Rearranging for \( V_f \), we get:

\[
\frac{I_D}{I_o} = e^{\frac{V_f}{\eta V_T}} \tag{4}
\]

Taking the natural logarithm of both sides:

\[
\frac{\eta V_T}{I_o} \ln I_D = V_f \tag{5}
\]

For diode \( D_1 \), we have:

\[
\frac{\eta V_T}{I_o} \ln I_{f} = V_{f1} \quad \text{for} \quad D_1 \tag{6}
\]

Similarly, for diode \( D_2 \):

\[
\frac{\eta V_T}{I_o} \ln I_{2} = V_{f2} \quad \text{for} \quad D_2 \tag{7}
\]

Substituting equations (6) and (7) into equation (1):

\[
- \frac{\eta V_T}{I_o} \ln I_2 + \frac{\eta V_T}{I_o} \ln I_f = V_1 \tag{8}
\]

Simplifying:

\[
\frac{\eta V_T}{I_o} \left( \ln I_f - \ln I_2 \right) = V_1 \tag{9}
\]

This becomes:

\[
\frac{\eta V_T}{I_o} \ln \left( \frac{I_f}{I_2} \right) = V_1 \tag{10}
\]

For matched diodes \( D_1 \) and \( D_2 \), where \( I_{o1} = I_{o2} \), we have:

\[
\frac{\eta V_T}{I_o} \ln I_f = V_1 \tag{11}
\]

Finally:

\[
- \frac{\eta V_T}{I_o} \ln I_2 = V_1 \tag{12}
\]
\subsection*{Improvement of the Antilog Amplifier}

Starting with:

\[
\ln I_2 = - V_1 \tag{10}
\]

This gives:

\[
I_f e^{-\frac{V_1}{\eta V_T}} = I_2 \tag{11}
\]

Rearranging:

\[
I_f e^{-\frac{V_1}{\eta V_T}} = I_2 \tag{12}
\]

Now, using:

\[
V_1 = V_s \frac{R_1}{R_1 + R_2} \tag{13}
\]

And:

\[
I_2 = \frac{V_0}{R_3} \tag{14}
\]

Substitute equation (13) into equation (12):

\[
I_f e^{-\frac{V_s R_1}{\eta V_T (R_1 + R_2)}} = \frac{V_0}{R_3} \tag{15}
\]

Finally, we get:

\[
V_0 = K_1 e^{K_2 V_s} \tag{16}
\]

\textit{Issues with the Circuit:}
\begin{itemize}
\item  \( V_0 \) still depends on the thermal voltage \( V_T \), which is not ideal.
\item For practical op-amps, the circuit still suffers from offset voltages, which can be corrected using null offsetting potentiometers.


\end{itemize}
An improvement to this is the matched transistor antilog amplifier.
\begin{figure}[H]
    \centering
    \includegraphics[width=\linewidth]{MatchedAntilog.png}
    \caption{Matched transistor antilog amplifier}
    \label{fig:question_image}
\end{figure}

Applying Kirchhoff's Voltage Law (KVL) on loop ABC:

\[
-V_{be2} + V_{be1} = V_1 \tag{1}
\]

Determining \( V_{be1} \) and \( V_{be2} \) from Eber's Moll relation:

\[
V_{be1} = \eta V_T \ln \left( \frac{I_{c1}}{I_{o1}} \right) \tag{2}
\]

\[
V_{be2} = \eta V_T \ln \left( \frac{I_{c2}}{I_{o2}} \right) \tag{3}
\]

Substituting equations (3) and (2) into equation (1):

\[
-\eta V_T \ln \left( \frac{I_{c2}}{I_{o2}} \right) + \eta V_T \ln \left( \frac{I_{c1}}{I_{o1}} \right) = V_1 \tag{4}
\]

\[
\eta V_T \ln \left( \frac{I_{c1}}{I_{o1}} \right) \frac{I_{o2}}{I_{c2}} = V_1 \tag{5}
\]

For matched transistors \( Q_1 \) and \( Q_2 \), we get:

\[
V_1 = \eta V_T \ln \left( \frac{I_{c1}}{I_{c2}} \right) \tag{6}
\]

\[
V_1 = -\eta V_T \ln \left( \frac{I_{c2}}{I_{c1}} \right) \tag{7}
\]

Taking the logarithm:

\[
\ln \left( \frac{I_{c2}}{I_{c1}} \right) = - \frac{V_1}{\eta V_T} \tag{8}
\]

\[
\frac{I_{c2}}{I_{c1}} = e^{-\frac{V_1}{\eta V_T}} \tag{9}
\]

Substituting for \( V_1 \):

\[
V_1 = V_s \frac{R_3}{R_3 + R_4} \tag{11}
\]

Thus, the collector current for transistor 1 is:

\[
I_1 = I_{c1} = \frac{V_R - V_1}{R_2} \approx \frac{V_R}{R_2} \tag{12} 
\]
(since \( R_4 \gg R_3 \)).

For transistor 2, we have:

\[
I_{c2} = \frac{V_0}{R_1} \tag{13}
\]

Substituting equations (11), (12), and (13) into equation (10):

\[
V_0 = V_R \exp \left( -\frac{V_s R_3}{\eta V_T (R_3 + R_4)} \right) \tag{14}
\]

Finally, the output voltage is:

\[
V_0 = \frac{R_1}{R_2} V_R \exp \left( -\frac{V_s R_3}{\eta V_T (R_3 + R_4)} \right) \tag{15}
\]

The output voltage relationship is given by:

\[
V_0 = K_1 e^{K_2 V_S} \tag{16}
\]

\subsection*{Calibration Procedure}
\begin{enumerate}
    \item Set \( V_S = 0 \) and monitor the output of \( A_1 \), which should be zero. If not, vary \( P_1 \) until the output of \( A_1 = 0 \).
    \item Set \( V_S = 0 \) and monitor the output of \( A_2 \) until \( V_0 = \frac{R_1}{R_2} V_R \). If not, vary \( P_2 \) until \( V_0 = \frac{R_1}{R_2} V_R \).
\end{enumerate}

\subsection*{Advantages}
\begin{enumerate}
    \item \( V_S \) can be greater than \( V_D \) of a diode.
    \item The diodes used are those due to the base-emitter junction of the transistor, which operate at a much lower current, ensuring conformity to the exponential relationship of the Eber’s Moll model.
    \item The output is independent of \( I_o \).
\end{enumerate}

\subsection*{Disadvantages}
\begin{enumerate}
    \item The output is still dependent on \( V_T \) (thermal voltage).
\end{enumerate}
\begin{figure}[H]
    \centering
    \includegraphics[width=\linewidth]{Junior.png}
    \caption{}
    \label{fig:question_image}
\end{figure}

\textbf{D1 \& D2} are protection diodes to prevent excessive reverse biasing of the base-emitter junction. \\
\textbf{RT} is a temperature coefficient resistor which compensates for the fall of \( V_{be} \) with temperature.

\subsection*{Applying KVL to Loop abc}

\[
-V_{be2} + V_{be1} = V \tag{1}
\]

\[
V_{be1} - V_{be2} = V
\]

\[
\eta V_T \ln \left(\frac{I_{c1}}{I_{o1}}\right) - \eta V_T \ln \left(\frac{I_{c2}}{I_{o2}}\right) = V \tag{2}
\]

\[
V = \eta V_T \ln \left(\frac{I_{c1}}{I_{o1}}\right) \frac{I_{o2}}{I_{c2}} \tag{3}
\]

For matched \( Q_1 \) and \( Q_2 \):

\[
V = \eta V_T \ln \left(\frac{I_{c1}}{I_{c2}}\right) \tag{4}
\]

\[
V = -\eta V_T \ln \left(\frac{I_{c2}}{I_{c1}}\right) \tag{5}
\]

\[
\ln \left(\frac{I_{c2}}{I_{c1}}\right) = - \frac{V}{\eta V_T} \tag{6}
\]

\[
e^{-\frac{V}{\eta V_T}} = \frac{I_{c2}}{I_{c1}} \tag{7}
\]

But:

\[
V = V_s R_T \tag{8}
\]

\[
I_1 = I_{c1} = \frac{V_R}{R_2}, \quad I_2 = \frac{V_o}{R_L}
\]

Substituting (8) into (7):

\[
V_o = \frac{V_R}{R_L} \exp\left(- \frac{V_s R_T}{\eta V_T (R_T + R_3)}\right) \tag{9}
\]

\[
V_o = R_L V_R \exp\left(- \frac{V_s R_T}{\eta V_T (R_T + R_3)}\right) \tag{10}
\]

\[
V_o = K_1 e^{K_2 V_s} \tag{11}
\]

\section*{Problems}

\subsection*{Problem 1}
For the basic transistor log amplifier (input resistance \( R_i = 10 \, \text{k}\Omega \) and \( I_o = 10^{-13} \, \text{A} \)), calculate \( V_o \) for each of the following input voltages:
\begin{enumerate}
    \item (a) \( 10 \, \text{mV} \)
    \item (b) \( 100 \, \text{mV} \)
    \item (c) \( 1 \, \text{V} \)
    \item (d) \( 10 \, \text{V} \)
\end{enumerate}
Comment on the computed results.

\subsection*{Problem 2}
For the basic transistor antilog amplifier, assume \( R_f = 10 \, \text{k}\Omega \) and \( I_o = 10^{-13} \, \text{A} \). Calculate \( V_o \) for each of the following input voltages \( V_i \):
\begin{enumerate}
    \item (a) \( 0.4 \, \text{V} \)
    \item (b) \( 0.5 \, \text{V} \)
    \item (c) \( 0.6 \, \text{mV} \)
\end{enumerate}
Comment on these results.

\subsection*{Problem 3}
Determine the transfer functions of the circuits given below.
\begin{figure}[H]
    \centering
    \includegraphics[width=\linewidth]{page1.png}
    \caption{(a)}
    \label{fig:question_image}
\end{figure}
\begin{figure}[H]
    \centering
    \includegraphics[width=\linewidth]{page2.png}
    \caption{(b)}
    \label{fig:question_image}
\end{figure}
\begin{figure}[H]
    \centering
    \includegraphics[width=\linewidth]{page3.png}
    \caption{(c)}
    \label{fig:question_image} 
\end{figure} 

\chapter{Analog Multipliers \& Dividers}

\section{Multipliers}
An ideal analog multiplier is a circuit that provides an output \( V \) as a function of \( V_x \) and \( V_y \) such that:
\[
V = K V_x V_y
\]
where \( K \) is a constant.

An ideal multiplier has the following characteristics:
\begin{itemize}
    \item Infinite input impedance for the signals \( V_x \) and \( V_y \)
    \item Zero output impedance
    \item Infinite bandwidth
    \item Linear for all ranges of input
\end{itemize}

\subsection{Practical Multipliers}
Practical multipliers have the following limitations:
\begin{enumerate}
    \item Since the power supply voltages are usually \( \pm 15V \), the theoretical maximum output signal swing is limited to \( \pm 15V \), but normal operation is restricted to \( \pm 10V \).
    \item The input signal is also limited to \( \pm 10V \).
    \item Limited output current capability.
    \item Output impedance is not zero (the load resistance should be large).
    \item Finite input resistance, typically between \( 10k\Omega \) and \( 100k\Omega \).
    \item Tendency to suffer from offset errors.
    \item Finite bandwidth.
\end{enumerate}

\subsection{Practical Realizations (Schemes)}
The following are common practical realizations:
\begin{itemize}
    \item Log-antilog
    \item Quarter square
    \item Transconductance (variable) multiplier
\end{itemize}

\subsubsection{Log-Antilog Method (Log Sum)}
This method uses the relation:
\[
xy = \text{antilog}(\log x + \log y)
\]

\textbf{Limitations:}
\begin{itemize}
    \item High cost as many operational amplifiers (OPAMPs) are required, making it complex.
    \item Log and antilog amplifiers are usually unidirectional, and unidirectional OPAMPs are expensive due to the large component count.
    \item The dynamic range of the input signal is usually small in the region where the log is computed with high accuracy.
    \item Bandwidth is small and dependent on the OPAMP used.
    \item Suffers from large offsets due to the cascaded devices.
\end{itemize}

This circuit is rarely used due to these limitations.

\subsubsection{Quarter Square Method}
This method uses the identity:
\[
\frac{1}{4} \left[(x + y)^2 - (x - y)^2\right] = xy
\]
Based on this identity, the elements required are squarers, adders, and subtractors. Diodes are used to perform the squaring function in this circuit.

\paragraph{General Block Diagram}
The general block diagram for this configuration is shown in Figure \ref{fig:quarter_square_diagram}. The diagram illustrates:
\begin{itemize}
    \item Input signals \(x\) and \(y\) being combined through adders and subtractors
    \item Squaring blocks that compute \((x + y)^2\) and \((x - y)^2\)
    \item A final division by 4 to achieve the multiplication result \(xy\)
\end{itemize}

\begin{figure}[H]
\centering
\begin{tikzpicture}[
    block/.style={rectangle, draw, minimum width=1cm, minimum height=0.5cm},
    sum/.style={circle, draw, minimum size=0.4cm, inner sep=0pt},
    arrow/.style={->, thick}
]

% Input nodes
\node[block] (x) at (0,0) {Input $x$};
\node[block] (y) at (0,-2) {Input $y$};

% Adder and subtractor nodes  
\node[sum] (add) at (2,0) {$+$};
\node[sum] (sub) at (2,-2) {$-$};

% Squaring blocks
\node[block] (sqr1) at (4,0) {$(x+y)^2$};
\node[block] (sqr2) at (4,-2) {$(x-y)^2$};

% Output stage
\node[block] (div) at (6,-1) {$\div 4$};
\node[block] (out) at (8,-1) {Output $xy$};

% Connections
\draw[arrow] (x) -- (add);
\draw[arrow] (y) -- (add);
\draw[arrow] (x) -- (sub);
\draw[arrow] (y) -- (sub);
\draw[arrow] (add) -- (sqr1);
\draw[arrow] (sub) -- (sqr2);
\draw[arrow] (sqr1) -- (div);
\draw[arrow] (sqr2) -- (div);
\draw[arrow] (div) -- (out);

\end{tikzpicture}
\caption{Block diagram of the quarter square multiplier showing signal flow and processing stages.}
\label{fig:quarter_square_diagram}
\end{figure}



\paragraph{Circuit Implementation}
In the circuit implementation, components are arranged according to the quarter square method identity. This involves:
\begin{itemize}
    \item Squarers: Diodes are arranged to implement the squaring function.
    \item Adders and subtractors: Combined to satisfy the identity \(\frac{1}{4} \left[(x + y)^2 - (x - y)^2\right] = xy\).
\end{itemize}

\begin{figure}[h]
    \centering
    \includegraphics[width=\linewidth]{CircuitImplementation.png} % Adjust width as necessary
    \caption{Quarter Square Circuit Implementation}
    \label{fig:quarter_square_circuit}
\end{figure}


\paragraph{Considering the Simplified Circuit}
In the simplified circuit, focus is placed on the core components: 

\begin{figure}[h]
    \centering
    \includegraphics[width=0.25\textwidth]{Simplified.png} % Adjust width as necessary
    \caption{Simplified Circuit Diagram}
    \label{fig:simplified_circuit}
\end{figure}

\begin{itemize}
    \item Diodes \( D_1 \) and \( D_2 \) are used to perform the squaring operations.
    \item Output voltage \( V_o \) represents the multiplied result.
\end{itemize}





\textbf{Squarer Components:} Consider the squarer comprising diodes \( D_1 \) and \( D_2 \), which are arranged to achieve the desired squaring function.

\begin{figure}[h]
    \centering
    \includegraphics[width=0.25\textwidth]{Squarer.png} % Adjust width as necessary
    \caption{Squarer}
    \label{fig:squarer}
\end{figure}

\textbf{Consider the Squarer Comprising \( D_1 \) and \( D_2 \)}


Let:
\[
x + y = z
\]

The diode current \( I_D \) is given by:
\[
I_D = I_o \left(e^{\frac{V_f}{\eta V_T}} - 1\right) \tag{1}
\]

Thus, for \( D_2 \):
\[
I_b = I_o \left(e^{\frac{Z}{\eta V_T}} - 1\right) \tag{2}
\]
\[
I_b = I_o \left(e^{KZ} - 1\right), \quad K = \frac{1}{\eta V_T} \tag{3}
\]

Expanding \( I_b \) using a Taylor series:
\[
I_b = I_o \left(1 + KZ + \frac{(KZ)^2}{2!} + \frac{(KZ)^3}{3!} + \cdots - 1\right) \tag{4}
\]
\[
I_b \approx I_o \left(KZ + \frac{(KZ)^2}{2!}\right) \tag{5}
\]

Similarly,
\[
I_a = I_o \left(e^{-KZ} - 1\right) \tag{6}
\]
\[
I_a \approx I_o \left(-KZ + \frac{(-KZ)^2}{2!}\right) \tag{7}
\]

Thus, we have:
\[
I_1 = I_a + I_b = I_o \frac{2(KZ)^2}{2!} \tag{8}
\]
\[
= I_o (KZ)^2 \tag{9}
\]

\textbf{Squarer Components:} Consider the squarer comprising diodes \( D_3 \) and \( D_4 \), which are arranged to achieve the desired squaring function.

\begin{figure}[h]
    \centering
    \includegraphics[width=0.15\textwidth]{Squarer.png} % Adjust width as necessary
    \caption{Squarer}
    \label{fig:squarer}
\end{figure}

\textbf{Consider the Squarer Comprising \( D_3 \) and \( D_4 \)}


Let:
\[
p = x - y
\]

Then:
\[
I_d = I_o \left(e^{\frac{P}{\eta V_T}} - 1\right) \tag{10}
\]
\[
I_d = I_o \left(e^{KP} - 1\right) \tag{11}
\]

Expanding \( I_d \) similarly:
\[
I_d = I_o \left(1 + KP + \frac{(KP)^2}{2!} + \frac{(KP)^3}{3!} + \cdots - 1\right) \tag{12}
\]
\[
I_d \approx I_o \left(KP + \frac{(KP)^2}{2!}\right) \tag{13}
\]

Similarly:
\[
I_c = I_o \left(-KP + \frac{(-KP)^2}{2!}\right) \tag{14}
\]

We define:
\[
I_2 = I_o (KP)^2 \tag{15}
\]
\[
I_3 = I_1 - I_2 = I_o (KZ)^2 - I_o (KP)^2 \tag{16}
\]
\[
= I_o \left[(KZ)^2 - (KP)^2\right] \tag{17}
\]

The output voltage \( V_o \) is given by:
\[
V_o = -I_3 R_f \tag{18}
\]
\[
= -R_f I_o \left[(KZ)^2 - (KP)^2\right] \tag{19}
\]
\[
= -R_f I_o K^2 (Z^2 - P^2) \tag{20}
\]
\[
= -R_f I_o K^2 \left[(x + y)^2 - (x - y)^2\right] \tag{21}
\]

For the circuit to multiply, we set:
\[
-R_f I_o K^2 = \frac{1}{4} \quad \text{or} \quad R_f = \frac{1}{4 I_o K^2} \tag{22}
\]

Since the squaring is performed by using diodes (which is approximate), this realization results in multipliers that are fairly inaccurate; hence, this implementation is rarely used.

\subsubsection{Variable Transconductance Method}
In this scheme, the transconductance of a differential amplifier is varied to achieve multiplication.

\paragraph{Basic 2-Quadrant Multiplier}
For the general differential amplifier:
\[
V_o = -g_m V_1 R_L \tag{23}
\]
If \( g_m \) can be varied by a signal \( V_2 \), i.e., \( g_m = K V_2 \), then:
\[
V_o = -K V_1 V_2 R_L \tag{24}
\]
\[
= -K' V_1 V_2 \tag{25}
\]
where \( K' = KR_L \).
\begin{figure}[h]
    \centering
    \includegraphics[width=0.2\textwidth]{Transconductance.png} % Adjust width as necessary
    \caption{Transconductance Representation}
    \label{fig:transconductance}
\end{figure}
For the differential amplifier, the output voltage is given by:
\[
V_o = -g_m V_1 R_L \tag{26}
\]
where the transconductance \( g_m \) can be expressed as:
\[
g_m = \frac{I_{EE}}{2V_T} \tag{27}
\]
Substituting for \( I_{EE} \):
\[
I_{EE} = \frac{V_2 - V_{BE}}{R_E} \approx \frac{V_2 - 0.6}{R_E} \tag{28}
\]
Thus, we have:
\[
g_m = \frac{V_2 - 0.6}{2V_T R_E} \tag{29}
\]

Inserting this back into the equation for \( V_o \):
\[
V_o = -\frac{(V_2 - 0.6)V_1 R_L}{2V_T R_E} \tag{30}
\]
\begin{figure}[h]
    \centering
    \includegraphics[width=0.3\textwidth]{Transconductance.png} % Adjust width as necessary
    \caption{Transconductance Representation}
    \label{fig:transconductance}
\end{figure}

\paragraph{Limitations}
\begin{enumerate}
    \item The output is floating, meaning there is no common reference point.
    \item The signal \( V_1 \) should be much less than \( 4V_T \) to prevent transistors \( Q_1 \) and \( Q_2 \) from operating as switches.
    \item The signals \( V_1 \) and \( V_2 \) do not share a common zero reference.
\end{enumerate}

\paragraph{Advantages}
\begin{enumerate}
    \item Low cost due to a reduced number of components.
    \item Suitable for IC fabrication.
    \item High-frequency capability.
\end{enumerate}

Because of these advantages, this circuit is commonly used in small-signal multiplication applications, such as those found in small-signal processing circuits, communication receivers, transceivers, and transmitting circuits. It is also employed in the generation of double sideband (DSB), frequency modulation (FM), amplitude modulation (AM) modulators and demodulators, and phase modulation systems.

\begin{figure}[h]
    \centering
    \includegraphics[width=0.66\textwidth]{Improvement.png} % Adjust width as necessary
    \caption{Improved Transconductance Circuit Representation}
    \label{fig:improved_transconductance}
\end{figure}

Figure (b) shows an improvement over Figure (a), where the output is no longer floating but is now referenced with respect to ground. This is achieved by connecting an additional amplifier to the output of transistors \( Q_1 \) and \( Q_2 \), which stabilizes the output reference.

\[
I_{EE4} = \frac{V_2 - V_{BE4}}{R_E} \approx \frac{V_2}{R_E}
\]

Here, \( I_{EE4} \) is defined as the emitter current derived from \( V_2 \) and grounded through \( R_E \), allowing for improved performance in signal grounding and amplification.

\subsubsection{Gilbert's Multiplier Cell}
This is an improvement on the basic differential transconductance multiplier as it eliminates most of the former's limitations while retaining its advantages. In its implementation, two sets of differential amplifiers are connected in parallel with the load.
\begin{figure}[h]
    \centering
    \includegraphics[width=0.35\textwidth]{CELL.png} % Adjust width as necessary
    \caption{Gilbert's Multiplier Cell}
    \label{fig:CELL}
\end{figure}
\paragraph{Assumptions}
\begin{itemize}
    \item \( Q_1, Q_2, Q_3 \) and \( Q_4 \) are matched; \( Q_5 \) and \( Q_6 \) are matched.
    \item \( \beta_1, \beta_2, \beta_3, \beta_4, \beta_5, \beta_6 \gg 1 \), i.e., \( I_C \gg I_B \) (effects of \( I_B \) can be neglected).
\end{itemize}

I. The dynamic current relations are:
\begin{align}
I_1 + I_2 &= I_5 \tag{31} \\
I_3 + I_4 &= I_6 \tag{32} \\
I_5 + I_6 &= I_T \tag{33}
\end{align}

II. Consider the current relations in any general differential amplifier. The output voltage is given by:
\begin{figure}[h]
    \centering
    \includegraphics[width=0.2\textwidth]{II.png} % Adjust width as necessary
    \caption{General differential amplifier}
    \label{fig:II}
\end{figure}
\begin{align}
V_o &= -g_{m12} V_1 R_L \tag{34} \\
V_o &= (V_{CC} - I_1 R_L) - (V_{CC} - I_2 R_L) \tag{35} \\
V_o &= V_{CC} - I_1 R_L - V_{CC} + I_2 R_L \tag{36} \\
V_o &= (I_2 - I_1) R_L \tag{37} \\
V_o &= -(I_1 - I_2) R_L \tag{38}
\end{align}

Equating (38) with (34):
\begin{align}
-(I_1 - I_2) R_L &= -g_{m12} V_1 R_L \tag{39} \\
g_{m12} V_1 &= I_1 - I_2 \tag{40}
\end{align}

III. Applying (40) to \( Q_1, Q_2 \) and \( Q_3, Q_4 \):
\begin{align}
I_1 - I_2 &= g_{m12} V_1 \quad \text{for } Q_1 \text{ \& } Q_2 \tag{41} \\
I_4 - I_3 &= g_{m34} V_1 \quad \text{for } Q_3 \text{ \& } Q_4 \tag{42}
\end{align}

IV. Consider the output voltage of the cell:
\begin{align}
V_o &= [V_{CC} - (I_1 + I_3)R_L] - [V_{CC} - (I_2 + I_4)R_L] \tag{43} \\
&= V_{CC} - (I_1 + I_3)R_L - V_{CC} + (I_2 + I_4)R_L \tag{44} \\
&= [-(I_1 + I_3) + (I_2 + I_4)] R_L \tag{45} \\
V_o &= [(I_2 - I_1) + (I_4 - I_3)] R_L \tag{46}
\end{align}

Substituting equations (41) and (42) into (46):
\begin{align}
V_o &= (g_{m34} - g_{m12}) V_1 R_L \tag{47}
\end{align}

But,
\begin{align}
g_{m34} &= \frac{I_6}{2V_T} \tag{48} \\
g_{m12} &= \frac{I_5}{2V_T} \tag{49}
\end{align}

Substituting into (47):
\begin{align}
V_o &= V_1 R_L \left(\frac{I_6 - I_5}{2V_T}\right) \tag{50}
\end{align}
\begin{figure}[h]
    \centering
    \includegraphics[width=0.2\textwidth]{Re.png} % Adjust width as necessary
    \caption{}
    \label{fig:Re}
\end{figure}
Relating \( I_5 \) and \( I_6 \) with \( V_2 \), two cases exist:

\paragraph{Case 1: \( R_E = 0 \)}
Current relations are the same as that of a general differential amplifier:
\begin{align}
I_5 - I_6 &= g_{m56} V_2 \tag{51} \\
I_6 - I_5 &= -g_{m56} V_2 \tag{52}
\end{align}

Substituting equation (52) into (50):
\begin{align}
V_o &= -V_1 R_L g_{m56} V_2 \tag{53} \\
V_o &= -\frac{V_1 R_L I_T}{2V_T} V_2 \tag{54} \\
V_o &= K V_1 V_2 \tag{55}
\end{align}
where \( K = -\frac{R_L I_T}{2V_T} \).

\paragraph{Case 2: \( R_E \neq 0 \)}
Current relations between \( I_5 \), \( I_6 \), and \( V_2 \) are obtained by applying KVL on loop \( abcd \):
\begin{align}
V_2 - V_{BE5} - I_5 R_E + I_6 R_E + V_{BE6} &= 0 \tag{56} \\
V_2 &= V_{BE5} - V_{BE6} + (I_5 - I_6) R_E \tag{57}
\end{align}
\begin{figure}[h]
    \centering
    \includegraphics[width=0.2\textwidth]{Case2.png} % Adjust width as necessary
    \caption{}
    \label{fig:Case2}
\end{figure}
Using the diode equation:
\begin{align}
V_2 &= V_T \ln\left(\frac{I_5}{I_6}\right) + (I_5 - I_6) R_E \tag{58}
\end{align}

For large \( R_E \):
\begin{align}
V_2 &\approx (I_5 - I_6) R_E \tag{59}
\end{align}

Therefore:
\begin{align}
I_5 - I_6 &\approx \frac{V_2}{R_E} \tag{60}
\end{align}

Substituting this into equation (50):
\begin{align}
V_o &= -V_1 R_L \frac{V_2}{2V_T R_E} \tag{61} \\
V_o &= K V_1 V_2 \tag{62}
\end{align}
where \( K = -\frac{R_L}{2V_T R_E} \).

\paragraph{Advantages}
\begin{enumerate}
    \item Both \( V_1 \) and \( V_2 \) can take positive and negative values, allowing four-quadrant operation.
    \item Suitable for monolithic fabrication.
    \item Large bandwidth (up to 2 GHz).
    \item Reduced cost due to monolithic fabrication.
\end{enumerate}

\paragraph{Limitations}
\begin{enumerate}
    \item Input voltage \( V_1 \) limited to \( < 4V_T \approx 100 \text{ mV} \).
    \item Dynamic range of \( V_2 \) limited to \( I_T R_E \).
    \item Signals lack common ground reference.
    \item Different input impedances: \( 2\beta R_E \) for \( V_2 \), \( 2h_{ie} \) for \( V_1 \).
\end{enumerate}

\paragraph{Practical Gilbert's Multiplier}
% Insert the first image here
\begin{figure}[h]
    \centering
    \includegraphics[width=1.1\linewidth]{Practical.png} % Adjust width as necessary
    \caption{Practical Gilbert’s Multiplier}
    \label{fig:Practical}
\end{figure}

\subsubsection{Analysis of New Scheme: Differential Voltage to Current Converter}:

% Insert the second image here
\begin{figure}[h]
    \centering
    \includegraphics[width=0.25\textwidth]{NewScheme.png} % Replace with your image path
    \caption{Differential Voltage to Current Converter}
    \label{fig:NewScheme}
\end{figure}

To analyze the configuration, we apply Kirchhoff's Voltage Law (KVL) to loop \( abcd \):

\begin{equation}
V_2 - V_{be5} - V_{RE} + V_{be6} = 0 \tag{33}
\end{equation}

Where \( V_{RE} \) is the voltage drop across \( R_E \). Rearranging gives us:

\begin{equation}
V_2 = V_{be5} - V_{be6} + V_{RE} \tag{34}
\end{equation}

From Ebers-Moll's relation, we have:

\begin{equation}
V_T \ln \left( \frac{I_5}{I_O} \right) = V_{be5} \tag{35}
\end{equation}

\begin{equation}
V_T \ln \left( \frac{I_6}{I_O} \right) = V_{be6} \tag{36}
\end{equation}

Substituting equations (35) and (36) into equation (34):

\begin{equation}
V_2 = V_T \ln \left( \frac{I_5}{I_O} \right) - V_T \ln \left( \frac{I_6}{I_O} \right) + V_{RE} \tag{37}
\end{equation}

Applying Kirchhoff's Current Law (KCL) at node \( E \):

\begin{equation}
I_5 = \frac{I_T}{2} + I_X \tag{38}
\end{equation}

And at node \( F \):

\begin{equation}
I_6 + I_X = \frac{I_T}{2} \tag{7}
\end{equation}

From this, we can express \( I_6 \):

\begin{equation}
I_6 = \frac{I_T}{2} - I_X \tag{8}
\end{equation}

Using the relation between \( I_5 \) and \( I_6 \):

\begin{equation}
I_5 - I_6 = I_X \tag{9}
\end{equation}

This gives:

\begin{equation}
I_X = I_5 - I_6 \tag{10}
\end{equation}

The voltage drop across \( R_E \) can be expressed as:

\begin{equation}
V_{RE} = I_X R_E = (I_5 - I_6) R_E \tag{40}
\end{equation}

Substituting equation (40) into (37):

\begin{equation}
V_2 = V_T \ln \left( \frac{I_5}{I_6} \right) + (I_5 - I_6) R_E \tag{41}
\end{equation}

For matched transistors \( Q_5 \) and \( Q_6 \):

\begin{equation}
V_2 \approx (I_5 - I_6) R_E \tag{42}
\end{equation}

Therefore:

\begin{equation}
I_5 - I_6 \approx \frac{V_2}{R_E} \tag{43}
\end{equation}

The output voltage can then be expressed as:

\begin{equation}
V_o = -R_L V_1 \cdot \frac{V_2}{R_E} \tag{44}
\end{equation}

Considering the approximation \( I_T - 2I_X^2 \), we have for small changes:

\[
\ln 1 = 0 \implies V_2 = R_E (I_5 - I_6) \tag{15}
\]

Next, substituting \( V_2 \) into the equation relating \( V_o \) and \( V_1 \):

\begin{equation}
V_o = -R_L V_1 (I_5 - I_6) \tag{16}
\end{equation}

Thus, we can express \( V_o \) as:

\begin{equation}
V_o = -R_L V_1 \cdot 2V_2 \tag{17}
\end{equation}

This can be rewritten as:

\begin{equation}
V_o = -\frac{2R_L V_1 V_2}{V_T R_E} \tag{18}
\end{equation}

And finally, we can simplify to:

\begin{equation}
V_o = -\frac{R_L V_1 V_2}{V_T \left(\frac{R_E}{2}\right)} \tag{19}
\end{equation}

\subsection{Comparison of Two Configurations}

\subsubsection{Overview}
To compare the two configurations of the linear differential voltage-to-current converters, we analyze the effect of \( R_1 \) in the differential voltage-to-current converter.

\subsubsection{Equivalent Gain Analysis}
Both circuits can achieve the same output characteristics for given input conditions if they are designed to have equivalent gain \( A_V \). The gain relationship can be expressed as:

\begin{equation}
A_V = -\frac{R_L}{R_1} \tag{20}
\end{equation}

\subsubsection{Circuit Configurations}
The two main circuit configurations are:

\begin{itemize}
    \item \textbf{Circuit A:} Basic differential amplifier configuration with gain determined by:
    \begin{itemize}
        \item Input resistor \( R_1 \)
        \item Feedback resistor \( R_f \)
    \end{itemize}
    
    \item \textbf{Circuit B:} Gilbert cell configuration where transconductance values are adjusted to match Circuit A's behavior
\end{itemize}

\subsubsection{Equivalence}
Through proper component selection:
\begin{itemize}
    \item Appropriate \( R_1 \) value in Circuit A
    \item Matching transconductance in Circuit B
\end{itemize}

Both configurations can achieve equivalent gain, making them suitable for similar voltage-to-current conversion applications.

\subsubsection{Analysis taking into account effect of R1 in the different voltage to current converter}
\begin{figure}[h]
    \centering
    \includegraphics[width=0.25\textwidth]{Effect.png} % Adjust width as necessary
    \caption{}
    \label{fig:Practical}
\end{figure}
Applying Kirchhoff's Voltage Law (KVL) to the loop \( abEFcd \), we can derive the following relation:

\begin{equation}
V_2 - V_{be5} - I_5 R_1 - V_{RE} + I_6 R_1 + V_{be6} = 0 \tag{1}
\end{equation}

From this, we can rearrange the equation to express \( V_2 \):

\begin{equation}
V_2 = V_{be5} - V_{be6} + (I_5 - I_6) R_1 + V_{RE} \tag{2}
\end{equation}

Using Eber-Moll's relations, the base-emitter voltages can be expressed as:

\begin{equation}
V_{be5} = V_T \ln \left( \frac{I_5}{I_O} \right) \tag{3}
\end{equation}

The next step would involve deriving an expression for \( V_{be6} \) in a similar manner and substituting it back into equation (2) to establish the relationship between \( V_2 \) and the currents \( I_5 \) and \( I_6 \).

\subsubsection{Voltage Relations with Feedback}

For matched transistors, we have:

\begin{equation}
V_{be6} = V_T \ln \left( \frac{I_6}{I_O} \right) \tag{3}
\end{equation}

Assuming \(\eta = 1\) and \(I_{O5} = I_{O6}\), substituting equation (3) into equation (2) gives:

\begin{equation}
V_2 = V_T \ln \left( \frac{I_5}{I_6} \right) + (I_5 - I_6) R_1 + V_{RE} \tag{4}
\end{equation}

Applying Kirchhoff's Current Law (KCL) at node \(E\):

\begin{equation}
I_5 = \frac{I_T}{2} + I_X \tag{5}
\end{equation}

Applying KCL at node \(F\):

\begin{equation}
I_6 + I_X = \frac{I_T}{2} \tag{6}
\end{equation}

From this, we find:

\begin{equation}
I_6 = \frac{I_T}{2} - I_X \tag{7}
\end{equation}

Now, substituting equation (8) into (4):

\begin{equation}
V_2 = V_T \ln \left( \frac{I_5}{I_6} \right) + (I_5 - I_6) R_1 + \frac{(I_5 - I_6) R_E}{2} \tag{9}
\end{equation}

We can express this as:

\begin{equation}
V_2 = \ln \left( \frac{I_T}{2} \right) + I_X + (I_5 - I_6) \left( R_1 + \frac{R_E}{2} \right) \tag{10}
\end{equation}

Continuing with the simplification:

\begin{equation}
V_2 = \ln(I_T) + 2I_X + (I_5 - I_6) \left( R_1 + \frac{R_E}{2} \right) \tag{11}
\end{equation}

For small \(\Delta I = 2I_X\), we have:

\begin{equation}
V_2 = \ln(I_T) + \Delta I + (I_5 - I_6) \left( R_1 + \frac{R_E}{2} \right) \tag{12}
\end{equation}

Assuming \(\ln(1) = 0\):

\begin{equation}
V_2 \approx (I_5 - I_6) \left( R_1 + \frac{R_E}{2} \right) \tag{13}
\end{equation}

Substituting equation (13) into the equation relating \(V_o\) and \(V_1\):

\begin{equation}
V_o = -R_L V_1 (I_5 - I_6) \tag{14}
\end{equation}

Then we have:

\begin{equation}
V_o = -\frac{R_L V_1 V_2}{V_T \left( R_1 + \frac{R_E}{2} \right)} \tag{15}
\end{equation}

This leads to:

\begin{equation}
V_o = -R_L V_1 V_2 \frac{1}{V_T R_x''} \tag{16}
\end{equation}

where \(R_x'' = R_1 + \frac{R_E}{2}\).

\paragraph{Effects of Feedback}
The effect of \(R_1\) is to increase the local feedback, thereby improving the linearity of the voltage-to-current conversion. This results in a gain reduction at the output of the multiplier compared to:

\begin{equation}
V_0 = -\frac{R_L V_1 V_2}{V_T \left( \frac{R_E}{2} \right)} \tag{17}
\end{equation}

\textit{Summary of Effects:}
- \(R_1\) increases the linear dynamic range of the signal \(V_2\).
- For small deviations, we can express \(V_2\) in terms of the currents:

\[
V_2 = \ln(I_T) + \Delta I + (I_5 - I_6) \left( R_1 + \frac{R_E}{2} \right) \tag{a}
\]

When \(I_5 = 0\):

\[
V_2 = (0 - I_6) \left( R_1 + \frac{R_E}{2} \right) \tag{b}
\]

\subsubsection{Dynamic Range of \( V_2 \)}

For the case when \( I_6 = 0 \):

\begin{equation}
V_2 = -I_T \left( R_1 + \frac{R_E}{2} \right) \tag{C}
\end{equation}

When \( I_6 = 0 \):

\begin{equation}
V_2 = I_5 \left( R_1 + \frac{R_E}{2} \right)
\end{equation}

Then we have:

\begin{equation}
V_2 = I_T \left( R_1 + \frac{R_E}{2} \right) \tag{d}
\end{equation}

Therefore, the range of \( V_2 \) is given by:

\begin{equation}
\text{Range of } V_2 = \pm I_T \left( R_1 + \frac{R_E}{2} \right) \tag{e}
\end{equation}

From equation (e), we see that the dynamic range of the signal \( V_2 \) is increased to:

\[
\pm I_T \left( R_1 + \frac{R_E}{2} \right)
\]

For example, from the practical circuit given above, the dynamic range of the input can be calculated as:

\[
I_T \left( 25\,k + \frac{35}{2}\,k \right)
\]

Given that:

\[
\frac{I_T}{2} = 0.15\,mA \quad \Rightarrow \quad I_T = 0.3\,mA
\]

\textit{Question:} Find the input resistance seen by \( V_y \)


To demonstrate the increment of the linear dynamic range of the input signal using a large \( R_E \):
\begin{figure}[h]
    \centering
    \includegraphics[width=0.3\textwidth]{Increment.png} % Adjust width as necessary
    \caption{Improved Transconductance Circuit Representation}
    \label{fig:Practical}
\end{figure}


Starting from:

\begin{equation}
V_2 = V_T \ln \left( \frac{I_5}{I_6} \right) + (I_5 - I_6) R_E \tag{1}
\end{equation}

Using the equation:

\begin{equation}
I_5 + I_6 = I_T \tag{2}
\end{equation}

We can express \( V_2 \) in terms of \( I_5 \) only:

\begin{equation}
V_2 = V_T \ln \left( \frac{I_5}{I_5 - I_T} \right) + \left[ I_5 - \left( I_T - I_5 \right) \right] R_E \tag{3}
\end{equation}

This simplifies to:

\begin{equation}
V_2 = V_T \ln \left( \frac{I_5}{I_5 - I_T} \right) + [2I_5 - I_T] R_E \tag{4}
\end{equation}

\textit{Plotting \( I_5 \) versus \( V_2 \)}:
\begin{table}[h!]
\centering
\begin{tabular}{|c|c|c|}
\hline
$I_5$ & $V_T \ln \left( \frac{I_5}{I_5 - I_T} \right)$ & $\left[ 2I_5 - I_T \right] R_E$ \\
\hline
$0.1 I_T$ & $26 \ln \left( \frac{0.1I_T}{I_T - 0.1I_T} \right) = -57 \ \text{mV}$ & $-0.8 R_E I_T$ \\
$0.2 I_T$ & $-36$ mV & $-0.6 R_E I_T$ \\
$0.3 I_T$ & $-22$ mV & $-0.4 R_E I_T$ \\
$0.4 I_T$ & $-10.5$ mV & $-0.2 R_E I_T$ \\
$0.5 I_T$ & $0$ mV & $0$ \\
$0.6 I_T$ & $10.5$ mV & $0.2 R_E I_T$ \\
$0.7 I_T$ & $22$ mV & $0.4 R_E I_T$ \\
$0.8 I_T$ & $36$ mV & $0.6 R_E I_T$ \\
$0.9 I_T$ & $57$ mV & $0.8 R_E I_T$ \\
$I_T$     & $76$ mV & $0.9 R_E I_T$ \\
\hline
\end{tabular}
\caption{Values of $I_5$, voltage, and resistance expressions}
\label{table:values}
\end{table}

From the graph above, the linear range of \( V_2 \) is approximately \( \pm 0.9 R_E I_T \). The linear range of \( V_2 \) without \( R_E \) is \( +4V_T \approx 100\,mV \). Thus, by including \( R_E \), the linear range has been extended from \( \pm 4V_T \) to \( 0.9 R_E I_T \).

However, in the basic Gilbert’s multiplier cell, the input signal to \( V_1 \) is still limited to \( \pm 4V_T \), and even in this range, the relationship between \( I_1 \) and \( I_2 \) is not linear. This gives rise to large errors even for small signals of \( +4V_T \approx +100\,mV \).

\subsubsection{Analysis of Small Dynamic Range of Input \( V_1 \) and Currents \( I_1, I_2, I_3, I_4 \)}

Using the Gilbert’s Multiplier cell, we analyze the currents as follows:

From Eber's Moll Relation, we have:

\begin{equation}
I_1 = I_{01} \left( e^{\frac{V_{be1}}{V_T}} - 1 \right) \approx I_{01} e^{\frac{V_{be1}}{V_T}} \tag{1}
\end{equation}

\begin{equation}
I_2 = I_{02} \left( e^{\frac{V_{be2}}{V_T}} - 1 \right) \approx I_{02} e^{\frac{V_{be2}}{V_T}} \tag{2}
\end{equation}

By assuming matched transistors, we can relate \( I_1 \) and \( I_2 \):

\begin{equation}
I_1 = I_{01} e^{\frac{(V_{be1} - V_{be2})}{V_T}} \tag{3}
\end{equation}

Since \( V_{be1} - V_{be2} = V_1 \), we can express:

\begin{equation}
I_1 = I_{02} e^{\frac{V_1}{V_T}} \tag{4}
\end{equation}

This shows that the currents \( I_1 \) and \( I_2 \) are exponentially dependent on the input voltage \( V_1 \). Given the nature of this relationship, even small variations in \( V_1 \) can result in significant changes in the output currents \( I_1 \) and \( I_2 \), leading to a small dynamic range in the input.

\paragraph{Currents \( I_3 \) and \( I_4 \)}
To continue the analysis for the currents \( I_3 \) and \( I_4 \), we can write similar expressions based on the relationships defined by the configuration of the Gilbert Multiplier. However, we will need additional circuit context to derive \( I_3 \) and \( I_4 \) accurately.

The exponential relationship observed emphasizes the nonlinear response of the currents to input voltage changes, illustrating the limitations of the Gilbert Multiplier cell in achieving a broad linear operating range.

\subsubsection{Analysis of Currents \( I_1 \) and \( I_2 \)}

Given that:
\begin{equation}
I_1 + I_2 = I_{EE} \tag{5}
\end{equation}

From (4) and (5), we can relate \( I_1 \) and \( I_2 \) with \( V_1 \). 

Solving (5) in terms of \( I_1 \):
\begin{equation}
I_1 + I_1 e^{-\frac{V_1}{V_T}} = I_{EE} \tag{6}
\end{equation}
From which we can express \( I_2 \) as:
\begin{equation}
I_2 = I_1 e^{-\frac{V_1}{V_T}} \tag{7}
\end{equation}

Rearranging equation (6) gives:
\begin{equation}
I_1 (1 + e^{-\frac{V_1}{V_T}}) = I_{EE} \tag{7}
\end{equation}
Thus,
\begin{equation}
I_1 = \frac{I_{EE}}{1 + e^{-\frac{V_1}{V_T}}} \tag{8}
\end{equation}

Similarly, we can solve for \( I_5 \) in terms of \( I_2 \):
\begin{equation}
I_2 e^{\frac{V_1}{V_T}} + I_2 = I_{EE} \tag{9}
\end{equation}
This can be rearranged to:
\begin{equation}
I_1 (e^{\frac{V_1}{V_T}} + 1) = I_{EE} \tag{10}
\end{equation}
Thus,
\begin{equation}
I_2 = \frac{I_{EE}}{1 + e^{\frac{V_1}{V_T}}} \tag{11}
\end{equation}

The current and voltage relations are then given by:
\begin{equation}
I_1 = \frac{I_{EE}}{1 + e^{-\frac{V_1}{V_T}}} \quad \text{and} \quad I_2 = \frac{I_{EE}}{1 + e^{\frac{V_1}{V_T}}}
\end{equation}

Similarly, the currents \( I_3 \) and \( I_4 \) can be expressed in relation to \( V_1 \) using analogous methods.


\begin{table}[h!]
\centering
\begin{tabular}{|c|c|c|}
\hline
$V_1$ & $I_1$ & $I_2 = I_{EE} - I_1$ \\
\hline
$V_T$   & $0.737 \, I_{EE}$ & $0.263 \, I_{EE}$ \\
$2V_T$  & $0.88 \, I_{EE}$  & $0.12 \, I_{EE}$ \\
$3V_T$  & $0.95 \, I_{EE}$  & $0.05 \, I_{EE}$ \\
$4V_T$  & $0.98 \, I_{EE}$  & $0.02 \, I_{EE}$ \\
\hline
\end{tabular}
\caption{Values of $V_1$, $I_1$, and $I_2$}
\label{table:values}
\end{table}

From the plot of \( I_1 \) and \( I_2 \) versus \( V_1 \), it is evident that the linear range for the variation of the currents is very small, and at \( V_1 = 4V_T \), \( Q_1 \) is already saturated.

\textbf{Note:} In the earlier derivation, it was assumed that \( g_{m12} = \frac{I_5}{V_T} \). In practice, the actual value of \( g_m \) is given by \( \frac{\partial I_1}{\partial V_1} \), which is not constant.

To extend the linear range of the signal \( V_1 \), a preconditioning circuit must be employed, whose value is the inverse transfer function of the current-to-voltage relationship at the input of \( Q_1 \) and \( Q_2 \). 

To design the preconditioning circuit, we will analyze the Gilbert’s multiplier cell using Eber’s Moll relations.

\begin{figure}[h!]
    \centering
    \includegraphics[width=0.6\linewidth]{Multiplier.png}
    \caption{Gilbert’s multiplier cell}
    \label{fig:multiplier}
\end{figure}

\subsubsection{Analysis of Collector Currents}

Using Eber’s Moll relations for the collector currents of transistors \( Q_3 \) and \( Q_4 \):
\begin{equation}
I_{c3} = \frac{I_{c1}}{1 + e^{-\frac{V_1}{V_T}}} \tag{1}
\end{equation}
\begin{equation}
I_{c4} = \frac{I_{c1}}{1 + e^{\frac{V_1}{V_T}}} \tag{2}
\end{equation}

For the collector currents of transistors \( Q_5 \) and \( Q_6 \), we have:
\begin{equation}
I_{c5} = \frac{I_{c2}}{1 + e^{\frac{V_1}{V_T}}} \tag{3}
\end{equation}
\begin{equation}
I_{c6} = \frac{I_{c2}}{1 + e^{-\frac{V_1}{V_T}}} \tag{4}
\end{equation}

For transistors \( Q_1 \) and \( Q_2 \), the current relations are given by:
\begin{equation}
I_{c2} = \frac{I_{EE}}{1 + e^{\frac{V_2}{V_T}}} \tag{5}
\end{equation}

Substituting equations (5) and (6) into equations (1), (2), (3), and (4), we obtain:
\begin{equation}
I_{c3} = \frac{I_{EE}}{(1 + e^{-\frac{V_1}{V_T}})(1 + e^{-\frac{V_2}{V_T}})} \tag{7}
\end{equation}
\begin{equation}
I_{c4} = \frac{I_{EE}}{(1 + e^{\frac{V_1}{V_T}})(1 + e^{-\frac{V_2}{V_T}})} \tag{8}
\end{equation}
\begin{equation}
I_{c5} = \frac{I_{EE}}{(1 + e^{\frac{V_1}{V_T}})(1 + e^{\frac{V_2}{V_T}})} \tag{9}
\end{equation}
\begin{equation}
I_{c6} = \frac{I_{EE}}{(1 + e^{-\frac{V_1}{V_T}})(1 + e^{\frac{V_2}{V_T}})} \tag{10}
\end{equation}

The differential output current is given by:
\begin{equation}
\Delta I = (I_{c3} + I_{c5}) - (I_{c4} + I_{c6}) \tag{11}
\end{equation}

\subsubsection{Differential Output Current Analysis}

The differential output current can be expressed as:
\begin{equation}
\Delta I = (I_3 - I_6) - (I_4 - I_5) \tag{12}
\end{equation}
Substituting the collector currents, we have:
\begin{equation}
\Delta I = I_{EE} - I_{EE} - \frac{(1 + e^{-V_1/V_T})(1 + e^{-V_2/V_T})}{(1 + e^{-V_1/V_T})(1 + e^{V_2/V_T})} \tag{13}
\end{equation}
This simplifies to:
\begin{equation}
\Delta I = I_{EE} - I_{EE} \frac{(1 + e^{V_1/V_T})(1 + e^{-V_2/V_T})}{(1 + e^{V_1/V_T})(1 + e^{V_2/V_T})} \tag{14}
\end{equation}

Letting \( \frac{V_1}{V_T} = x \) and \( \frac{V_2}{V_T} = y \), we rewrite \( \Delta I \):
\begin{equation}
\Delta I = I_{EE} - I_{EE} - \frac{(1 + e^{-x})(1 + e^{-y})}{(1 + e^{-x})(1 + e^{y})} \tag{15}
\end{equation}

Thus:
\begin{equation}
\Delta I = I_{EE} \left( 1 - \frac{1}{(1 + e^{-x})(1 + e^{-y})} \right) \tag{16}
\end{equation}

Considering:
\begin{equation}
\frac{1}{(1 + e^{-x})} = \frac{e^{x}}{1 + e^{x}} \tag{18}
\end{equation}
Substituting into the equation, we obtain:
\begin{equation}
1 - \frac{1}{(1 + e^{-x})} = -\left(1 - \frac{e^{x}}{1 + e^{x}}\right) \tag{19}
\end{equation}

Similarly, for \( y \):
\begin{equation}
1 - \frac{1}{(1 + e^{-y})} = -\left(1 - \frac{e^{y}}{1 + e^{y}}\right) \tag{20}
\end{equation}

Now, recognizing the hyperbolic tangent function:
\begin{equation}
\tanh(x) = \frac{\sinh(x)}{\cosh(x)}
\end{equation}

\subsubsection{Hyperbolic Tangent Derivation}

We start from:
\begin{equation}
\tanh(x) = \frac{e^x - e^{-x}}{e^x + e^{-x}} = \frac{(e^{2x} - 1)}{(e^{2x} + 1)} \tag{21}
\end{equation}

From this, we can express the hyperbolic tangent as:
\begin{equation}
\tanh(x) = -\frac{(1 - e^{2x})}{(1 + e^{2x})} \tag{22}
\end{equation}

The differential output current can be expressed as:
\begin{equation}
\Delta I_o = I_{EE} - \frac{(1 - e^y)}{(1 + e^y)(1 + e^x)} - \frac{(1 - e^x)}{(1 + e^y)(1 + e^x)} \tag{23}
\end{equation}

Using equation (22):
\begin{equation}
\Delta I_o = I_{EE} \cdot \tanh\left(\frac{y}{2}\right) \cdot \tanh\left(\frac{x}{2}\right) \tag{24}
\end{equation}

\subsubsection{Output Voltage Expression}

From the circuit, the output voltage \( V_o \) can be derived as follows:
\begin{align*}
V_o &= [V_{cc} - R_L (I_3 + I_5)] - [V_{cc} - R_L (I_4 + I_6)] \\
&= - R_L (I_3 + I_5) + R_L (I_4 + I_6) \\
&= R_L [(I_4 + I_6) - (I_3 + I_5)] \\
&= R_L \Delta I_o \\
V_o &= R_L I_{EE} \left[ \tanh\left(\frac{V_2}{2V_T}\right) \tanh\left(\frac{V_1}{2V_T}\right) \right] \tag{25}
\end{align*}


\subsubsection{Practical Applications of Equation (25)}

The practical application of equation (25) can be divided into three categories depending on the magnitude of \( V_1 \) and \( V_2 \) with respect to \( V_T \):

1. \textit{When \( V_1, V_2 \ll V_T \)}: 
   This implies \( \frac{1}{2} V_1 < 1 \) and \( \frac{1}{2} V_2 < 1 \).
   Using the series expansion of \( \tanh(x) \approx x - \frac{x^3}{3} + \ldots \) for small \( x \):
   \[
   V_o \approx R_L I_{EE} \frac{1}{2} V_2 \frac{1}{2} V_1 = K V_1 V_2 \frac{1}{V_T^2}
   \]

2. \textit{When \( V_1, V_2 \gg V_T \)}: 
   The output is as given in equation (25). Two schemes can be employed to ensure that the output remains a product of the two inputs \( V_1 \) and \( V_2 \):
   - (a) Use of emitter degeneration resistances in the lower differential amplifier to increase the linear range of that input.
   - (b) Introduction of a non-linearity that pre-distorts the input signal to compensate for the hyperbolic tangent transfer function as shown below.

\subsubsection{Preconditioning Circuit / Pre-distortion Circuit}
\begin{figure}[H]
    \centering
    \includegraphics[width=\linewidth]{Precond.png}
    \caption{Preconditioning/Predistortion Circuit}
    \label{fig:precond}
\end{figure}
\subsubsection{Extending the Ranges of Signals \( V_1 \) and \( V_2 \)}

We are able to extend the ranges of the signals \( V_1 \) and \( V_2 \) by \( +10V \).
\begin{figure}[H]
    \centering
    \begin{minipage}[b]{0.45\textwidth}
        \centering
        \includegraphics[width=0.5\linewidth]{ExtendRange1.png}
        \label{fig:extendrange1}
    \end{minipage}
    \hfill
    \begin{minipage}[b]{0.45\textwidth}
        \centering
        \includegraphics[width=0.5\linewidth]{ExtendRange2.png}
        \label{fig:extendrange2}
    \end{minipage}

    \label{fig:extendranges}
\end{figure}


\subsubsection{Analysis of the Transfer Function}

To demonstrate that the transfer function of the preconditioning circuit is a \(\tanh^{-1}\) hyperbolic function, we begin by analyzing the current relations with respect to \( V_1 \).

\subsubsection{Applying Kirchhoff's Voltage Law (KVL)}

Applying KVL to the input loop of transistors \( Q_1 \) and \( Q_2 \):
\begin{align*}
V_1 - V_{be1} - I_1 R_E + I_2 R_E + V_{be2} &= 0 \tag{1} \\
V_1 &= (I_1 - I_2) R_E + V_{be1} - V_{be2} \tag{2}
\end{align*}

Using the Eber's Moll relation, we can express the base-emitter voltages as:
\begin{align*}
V_{be1} &= V_T \ln(I_1/I_{o1}) \tag{3} \\
V_{be2} &= V_T \ln(I_2/I_{o2}) \tag{4}
\end{align*}

Substituting these expressions into equation (2):
\begin{align*}
V_1 &= (I_1 - I_2) R_E + V_T \ln(I_1/I_{o1}) - V_T \ln(I_2/I_{o2}) \\
&= V_T \ln\left(\frac{I_1}{I_2}\right) + (I_1 - I_2) R_E \tag{5}
\end{align*}

Assuming \( V_T \ln\left(\frac{I_1}{I_2}\right) \approx 0 \) for large \( R_E \):
\[
I_1 - I_2 \approx \frac{V_1}{R_E} \tag{6}
\]

At the emitters of \( Q_1 \) and \( Q_2 \):
\begin{equation}
I_1 + I_2 = I_T \tag{7}
\end{equation}

By substituting \( I_2 = I_T - I_1 \) into equation (6):
\[
I_1 - (I_T - I_1) \approx \frac{V_1}{R_E}
\]

This simplifies to:
\[
2I_1 - I_T \approx \frac{V_1}{R_E} \tag{8}
\]

Thus:
\[
I_1 \approx \frac{I_T + \frac{V_1}{R_E}}{2} \tag{9}
\]

From equation (4):
\[
I_2 \approx \frac{I_T - \frac{V_1}{R_E}}{2} \tag{10}
\]

The transfer function can now be expressed as:
\[
V_1 \approx R_E \left(I_1 - I_2\right) \tag{11}
\]

Substituting for \( I_1 \) and \( I_2 \) gives:
\[
V_1 \approx R_E \left(\frac{I_T + \frac{V_1}{R_E}}{2} - \frac{I_T - \frac{V_1}{R_E}}{2}\right) \tag{12}
\]

This reduces to:
\[
V_1 \approx \frac{R_E}{2}\left(\frac{V_1}{R_E}\right) \tag{13}
\]

Thus, the output voltage \( V_o \) relates to the input signals through a hyperbolic tangent transformation, indicating the transfer function of the preconditioning circuit is indeed a \(\tanh^{-1}\) function, as desired.

\[
\text{Transfer Function: } V_o = K \cdot \tanh^{-1}\left(\frac{V_1}{V_T}\right) \tag{14}
\]

This shows that the input-output relationship can be effectively managed by the preconditioning circuit to extend the linear operational range of the Gilbert cell.

\subsubsection{Solving for \(I_1\) and \(I_2\)}

Starting from the previous equations, we have:
\begin{align*}
2I_1 &= I_T + \frac{V_1}{R_E} \tag{8} \\
I_1 &= \frac{1}{2}\left(I_T + \frac{V_1}{R_E}\right) \tag{9} \\
I_1 &= \frac{I_T}{2} + \frac{V_1}{2 R_E} \tag{10}
\end{align*}

Similarly, we can express \(I_2\):
\begin{align*}
I_1 + I_2 - I_1 + I_2 &= I_T - \frac{V_1}{R_E} \tag{11} \\
2I_2 &= I_T - \frac{V_1}{R_E} \tag{12} \\
I_2 &= \frac{I_T}{2} - \frac{V_1}{2R_E} \tag{13}
\end{align*}

\subsubsection{Considering the Output Voltage \(V_o\)}

The output voltage \(V_o\) is defined as:
\begin{align*}
V_o &= (V_{cc} - V_{D1}) - (V_{cc} - V_{D2}) \tag{14} \\
V_o &= -V_{D1} + V_{D2} \tag{15} \\
V_o &= V_{D2} - V_{D1} \tag{16}
\end{align*}

Assuming the diode currents are approximated as:
\begin{align*}
I_{D1} &= I_{01}\left(e^{\frac{V_f}{V_T}} - 1\right) \approx I_{01} e^{\frac{V_f}{V_T}} \tag{17} \\
I_{D2} &= I_{02} e^{\frac{V_f}{V_T}} \tag{18}
\end{align*}

For matched diodes, we have:
\begin{align*}
I_{D1} &= I_{01} e^{\frac{(V_{D1} - V_{D2})}{V_T}} \tag{19} \\
I_{D2} &= I_{02} \tag{20}
\end{align*}

From these equations, we can relate the voltages across the diodes to the currents:
\begin{align*}
\ln I_1 &= \frac{V_{D1} - V_{D2}}{V_T} \tag{21} \\
V_{D1} - V_{D2} &= V_T \ln I_1 \tag{22} \\
-V_o &= V_T \ln I_1 \tag{23} \\
I_2
\end{align*}

Substituting for \(I_1\) and \(I_2\) from equations (9) and (13) into (23):
\begin{align*}
-V_o &= V_T \ln\left(\frac{I_T/2 + \frac{V_1}{2R_E}}{I_T/2 - \frac{V_1}{2R_E}}\right) \tag{24}
\end{align*}

Letting \(I_T/2 = I_0\) and \(\frac{1}{2R_E} = K\), we have:
\begin{align*}
-V_o &= V_T \ln\left(\frac{I_0 + KV_1}{I_0 - KV_1}\right) \tag{25}
\end{align*}

Using the identity for the hyperbolic tangent inverse function:
\[
\tanh^{-1}(x) = \frac{1}{2} \ln\left(\frac{1 + x}{1 - x}\right) \tag{26}
\]

Thus:
\begin{align*}
-V_o &= V_T \cdot \tanh^{-1}\left(\frac{KV_1}{I_0}\right) \tag{27}
\end{align*}

This can be expressed as:
\[
-V_o = 2V_T \cdot \tanh^{-1}\left(\frac{KV_1}{I_0}\right) \tag{28}
\]

Thus, from (28), the output of the pre-distortion circuit is a \(\tanh^{-1}\) hyperbolic function of the input:
\[
V_o = -2V_T \cdot \tanh^{-1}\left(\frac{KV_1}{I_0}\right) \tag{29}
\]

This analysis demonstrates that the preconditioning circuit effectively extends the range of the input signals while maintaining a relationship characterized by a hyperbolic tangent inverse function.
\begin{center}
\begin{figure}[h!]
\centering
\begin{tikzpicture}[auto, node distance=2cm, thick, >=latex]
    % Input node
    \node (input) at (0,0) {$V_1$};
    
    % Block for pre-distortion circuit with increased width
    \node (block) [draw, rectangle, right of=input, xshift=1cm, 
                   minimum width=2.5cm, minimum height=1.2cm] 
                   {Pre-Conditioning Circuit};
    
    % Output node with more space
    \node (output) [right of=block, xshift=2cm] 
                   {$-2V_T \tanh^{-1} \left(\frac{K V_1}{I_0}\right)$};
    
    % Arrows between nodes
    \draw[->] (input) -- (block);
    \draw[->] (block) -- (output);
\end{tikzpicture}
\end{figure}
\end{center}
\subsection{Application to Gilbert Multiplier Cell}

When two preconditioning circuits are applied to a Gilbert multiplier cell where \( R_E = 0 \), the results are as follows:

\subsubsection{Output without Preconditioning Circuit}
The output voltage is given by:
\begin{equation}
-V_o = R_L I_{EE} \left[\tanh\left(\frac{V_2'}{2V_T}\right) \tanh\left(\frac{V_1'}{2V_T}\right)\right] \tag{1}
\end{equation}

\subsubsection{Output with Preconditioning Circuit}
With the preconditioning circuits applied, the output voltage becomes:

\begin{equation}
-V_o = R_L I_{EE} \left[ \tanh \left( \frac{2V_T}{2V_T} \right) \tanh^{-1} \left( \frac{K V_2}{I_0} \right) \tanh \left( \frac{2V_T}{2V_T} \right) \tanh^{-1} \left( \frac{K V_1}{I_0} \right) \right] \tag{2}
\end{equation}

where \( K = \frac{1}{2R_{E}} \) and \( I_{0} = \frac{I_{T}}{2} \).

This simplifies to:
\begin{equation}
V_o = R_L I_{EE} \left[\frac{K V_2}{I_0} \frac{K V_1}{I_0}\right] \tag{3}
\end{equation}

\subsection{Applications of Gilbert Multiplier Circuits}
The applications of the Gilbert multiplier cell include:
\begin{enumerate}
    \item Analog multiplication
    \item Frequency translation
    \item DSB modulation/demodulation
    \item Product detection
    \item Frequency doubling
    \item Phase detection
    \item FM modulation/demodulation
    \item Narrow band modulation/demodulation
    \item Phase modulation/detection
    \item Analog division
    \item Square root computation
\end{enumerate}
\begin{figure}[H]
    \centering
    % First Image
    \begin{minipage}[b]{0.3\textwidth}
        \centering
        \includegraphics[width=\linewidth]{SquareRooter.png}
        \caption{Square Rooter}
        \label{fig:squarerooter}
    \end{minipage}%
    \hfill
    % Second Image
    \begin{minipage}[b]{0.3\textwidth}
        \centering
        \includegraphics[width=\linewidth]{Division.png}
        \caption{Division}
        \label{fig:division}
    \end{minipage}%
    \hfill
    % Third Image
    \begin{minipage}[b]{0.3\textwidth}
        \centering
        \includegraphics[width=\linewidth]{MeanSquare.png}
        \caption{Mean Square}
        \label{fig:meansquare}
    \end{minipage}

    \caption{Various Arithmetic Functions}
    \label{fig:three_images}
\end{figure}


\chapter{Voltage Regulators}

\section{Introduction}
Most electrical and electronic equipment require specific voltages for their proper operation, hence the need for a regulated DC supply. For example, in industry, the following processes require fixed voltages:
\begin{itemize}
    \item Electroplating
    \item Arc welding: 47–67V DC, 100–250A
    \item Chemical industries: Electrolysis, chemical purification
    \item University electronic equipment: 
    \begin{itemize}
        \item Computers: ±12V for hard drives, +5V for motherboard, ±3.3V for the processor
        \item CRO, laboratory power supplies: ±15V, ±5V
    \end{itemize}
    \item Domestic applications: TVs, DVDs
\end{itemize}

\section{Types of Voltage Regulators}
Regulated DC power supply systems are of many types, and the type of control scheme used for a particular application depends on the following factors:
\begin{itemize}
    \item Feasibility: Is it suitable, portable, or practical for the particular use?
    \item Efficiency
    \item Mechanical construction
    \item Speed of response of the particular regulator
    \item Cost of design of the protection circuit
\end{itemize}

For more accuracy in the output, closed-loop control is usually employed. These are classified into two groups:
\begin{itemize}
    \item AC Regulators
    \item DC Regulators
\end{itemize}

\section{AC Regulators}
In the AC regulation scheme, the control element is incorporated in the AC side of the main rectifier.

\subsection{Disadvantages of the AC Regulation Scheme}
\begin{itemize}
    \item The response time of the circuit elements used is slow. This includes elements like SCRs, line transformers, and the speed of switching relays.
\end{itemize}

\section{D.C Regulators}

D.C regulators are employed to regulate the output voltage in electronic circuits. Here, the control circuit is incorporated on the D.C side of the main rectifier.

Switching devices like \textit{SCRs} (Silicon-Controlled Rectifiers) and \textit{GTOs} (Gate Turn-Off Thyristors) are typically used in high power applications. In contrast, \textit{BJTs} (Bipolar Junction Transistors) and \textit{MOSFETs} (Metal-Oxide-Semiconductor Field-Effect Transistors) are used in medium to low power applications (ranging from 100W to 1W). For micro-power applications, such as 100mW to 200mW, MOSFETs are usually preferred.

In practice, for medium power applications, MOSFETs and BJTs are commonly selected in the design of D.C regulators.

\subsection{Classification of D.C Regulators}

D.C regulators can be broadly classified into two categories:

\begin{enumerate}
    \item Linear Regulators
    \item Switched Regulators
\end{enumerate}

\subsection{Linear Regulators}

Linear regulators include the \textit{series} and \textit{shunt} regulators. These regulators maintain a stable output voltage by dissipating excess power as heat.

\subsection{Switched Regulators}

Switched regulators include \textit{switched-mode regulators}, which use switches (such as transistors) to regulate the output voltage efficiently by converting input energy to the desired output level.

\section{D.C Voltage Regulator}

The function of every D.C voltage regulator is to convert an input voltage into a specific, stable D.C output voltage and maintain that output voltage over a wide range of load currents, input voltage conditions, and temperature variations.

A typical D.C voltage regulator (whether series, shunt, or switched) consists of the following elements:

\begin{enumerate}
    \item \textbf{Voltage Reference Element:} This element provides a known stable voltage level, referred to as $V_{\text{ref}}$.
    \item \textbf{Sampling Element:} The sampling element monitors and samples the output voltage level.
    \item \textbf{Error Amplifier:} The error amplifier compares the sampled output voltage to the reference voltage and generates an error signal based on the difference.
    \item \textbf{Power Control Element:} This component adjusts the input voltage to the desired output level. The power control element is driven by the error signal to ensure proper regulation over varying load conditions.
\end{enumerate}

This structure allows the regulator to maintain a stable output despite changes in input voltage or load conditions.
\begin{figure}[H]
    \centering
    \includegraphics[width=0.4\textwidth]{DCReg.png}
    \caption{Block diagram of a D.C Regulator}
    \label{fig:DCReg}
\end{figure}
\section{Functions of the Various Elements}

\subsection{Reference Element}

The reference element is a fundamental component in all voltage regulators, as the output voltage is directly controlled by the reference voltage, \( V_{\text{ref}} \). Any variation in \( V_{\text{ref}} \) will be interpreted as an output voltage error by the error amplifier, causing the output voltage to change accordingly. To maintain accurate output regulation, \( V_{\text{ref}} \) must be stable, even with variations in supply voltages and junction temperatures.

A typical implementation of the reference element is the use of a \textit{Zener Diode Reference}. The Zener diode operates in its breakdown region, providing a stable voltage reference that is used to control the output voltage of the regulator.

\begin{itemize}
    \item The Zener diode maintains a stable voltage \( V_{\text{Zener}} \), even when the input voltage changes, as long as the input voltage remains above the Zener breakdown voltage.
    \item This stable reference voltage is crucial for accurate regulation, as any fluctuation in the reference voltage will directly affect the output voltage of the regulator.
\end{itemize}
\begin{figure}[H] % H places the figure exactly where it appears in the code
    \centering
    \includegraphics[width=0.2\textwidth]{zener.png} % Adjust width as needed
    \caption{A typical implementation of the reference element using a basic Zener reference.}
    \label{fig:zener-reference} % Label for referencing
\end{figure}
\begin{figure}[H] % 'H' to place the figure exactly here
    \centering
    \includegraphics[width=0.15\textwidth]{zener2.png} % Adjust width as necessary
    \caption{Accurate model Zener reference model.}
    \label{fig:accurate-zener-reference} % Label for referencing
\end{figure}
\subsection{Sampling Element}

The sampling element monitors the output voltage and converts it into a level equal to the reference voltage, \( V_{\text{ref}} \). A variation in the output voltage causes the feedback voltage to change to a value that may either be greater than or less than \( V_{\text{ref}} \). This voltage difference, called the error voltage, directs the regulator to make an appropriate response and thus correct the output voltage change.

A practical implementation of the sampling element is to use a resistive voltage divider:

\[
V_{\text{feedback}} = V_o \frac{R_2}{R_1 + R_2} \tag{1}
\]

where

\[
V_{\text{ref}} = V_o \frac{R_2}{R_1 + R_2} \tag{2}
\]

Rearranging, we get:

\[
V_o = V_{\text{ref}} \left( 1 + \frac{R_1}{R_2} \right)
\]
\begin{figure}[H] % 'H' to place the figure exactly here
    \centering
    \includegraphics[width=0.25\textwidth]{Opoa.png} % Adjust width as necessary
    \caption{Accurate model Zener reference model.}
    \label{fig:accurate-zener-reference} % Label for referencing
\end{figure}
\subsection{Error Amplifier}

The error amplifier monitors the feedback voltage, compares it with the reference voltage, and provides gain for the detected error signal. The output of the error amplifier drives the control element to return the output to the preset level. 

A typical implementation of the error amplifier uses an operational amplifier (OPAMP) with the following performance parameters:
\begin{itemize}
    \item Low offset voltages
    \item High Common Mode Rejection Ratio (CMRR) and Power Supply Rejection Ratio (PSRR)
    \item Low temperature coefficient
    \item Low output impedance
\end{itemize}

\subsection{Control Element}

All previously discussed elements remain virtually unaltered regardless of the type of regulator (linear or switching). However, the control element varies significantly depending on the regulator type. It is this element that classifies regulators as series, shunt, or switching. 

The design of the control element affects:
\begin{itemize}
    \item The minimum input-to-output differential voltage
    \item The circuit efficiency
    \item Power dissipation
\end{itemize}
\section{Typical topologies of the control element}
\begin{figure}[H] % Place exactly here, requires 'float' package
    \centering
    \begin{subfigure}{0.25\textwidth}
        \includegraphics[width=\textwidth]{series.png}
        \caption{Series Regulator}
        \label{fig:series-regulator}
    \end{subfigure}
    \hfill % Adjusts spacing between subfigures
    \begin{subfigure}{0.3\textwidth}
        \includegraphics[width=\textwidth]{shunt.png}
        \caption{Shunt Regulator}
        \label{fig:shunt-regulator}
    \end{subfigure}
    \hfill
    \begin{subfigure}{0.25\textwidth}
        \includegraphics[width=\textwidth]{switching.png}
        \caption{Switching Regulator}
        \label{fig:switching-regulator}
    \end{subfigure}
    \label{fig:regulator-types}
\end{figure}


The control element in this configuration functions as a switch, alternating between ON and OFF states. The output voltage \( V_o \) is determined by the ratio of the ON time (\( t_{\text{ON}} \)) to the total period (\( t_{\text{ON}} + t_{\text{OFF}} \)) as shown in the equation below:

\[
V_o = \frac{t_{\text{ON}}}{t_{\text{ON}} + t_{\text{OFF}}} \, V_1
\]

To adjust the output voltage, we control the ON and OFF times, specifically the pulse width during which the switch remains ON at a fixed frequency. A circuit that facilitates this is known as a pulse width modulator.

\subsection{Practical Series Regulator}

A practical series regulator is illustrated in Figure~\ref{fig:series-regulator}.

\begin{figure}[H]
    \centering
    \includegraphics[width=0.3\textwidth]{SeriesReg.png} % Replace with the actual image file name
    \caption{Practical Series Regulator Circuit}
    \label{fig:series-regulator}
\end{figure}

\subsection{Current Drive Requirement}

In cases where a larger current drive to the load is required, a Darlington pair or two transistors may be used, as illustrated in the figure. This approach compensates for the current limitations of most operational amplifiers (OPAMPs), which typically have a maximum output current limited to 5 mA or less.

\subsection{Analysis}

The output voltage \( V_o' \) can be derived as follows:

\begin{align}
    V_o' &= A_d \left( V_{\text{ref}} - V_o \frac{R_2}{R_1 + R_2} \right) \tag{1} \\
    &= A_d (V_z - V_o \beta) \tag{2}
\end{align}

where \( V_{\text{ref}} = V_z \) and \( \beta = \frac{R_2}{R_1 + R_2} \).

Also, for the output voltage in terms of transistor base-emitter voltages:

\begin{align}
    V_o' &= V_{\text{be1}} + V_{\text{be2}} + V_o \tag{3} \\
    &= 2V_{\text{be}} + V_o \tag{4}
\end{align}

Substituting (4) into the left-hand side of (2):

\begin{align}
    2V_{\text{be}} + V_o &= A_d (V_z - V_o \beta) \tag{5} \\
    V_o + A_d V_o \beta &= A_d V_z - 2V_{\text{be}} \tag{6} \\
    V_o (1 + A_d \beta) &= A_d V_z - 2V_{\text{be}} \tag{7} \\
    V_o &= \frac{A_d V_z - 2V_{\text{be}}}{1 + A_d \beta} \tag{8} \\
    V_o &= \frac{V_z - \frac{2V_{\text{be}}}{A_d}}{\frac{1}{A_d} + \beta} \tag{9}
\end{align}

For large \( A_d \gg 1 \), this simplifies further.


The output voltage \( V_o \) for the regulator can be expressed as follows:

\[
V_o = V_z = \frac{V_z}{\beta} = V_z \left( 1 + \frac{R_1}{R_2} \right) \tag{10}
\]

where \( \beta = \frac{R_2}{R_1 + R_2} \).

\subsection*{Design Requirements}

For effective design, the OPAMP must deliver the necessary base current corresponding to the maximum load current, considering the minimum estimated \( \beta_{\text{d.c}} \) of transistors \( Q_1 \) and \( Q_2 \). 

- The smaller the \( \beta_{\text{d.c}} \) of the transistors, the larger the current drive required from the OPAMP. 
- Power dissipation is approximately 300 mW.

\subsection*{Power Calculations}

The d.c. power delivered to the regulator circuit by the unregulated supply is given by:

\[
P_{\text{in}} = V_{\text{unreg}} \times I_L \tag{1}
\]

The load power is:

\[
P_L = V_L \times I_L \tag{2}
\]

Therefore, the power dissipation of the series element \( P_d \) is:

\begin{align}
    P_d &= P_{\text{in}} - P_L \tag{3} \\
    &= V_{\text{unreg}} I_L - V_L I_L \tag{4}
\end{align}

From equation (4), if the unregulated input voltage has significant variations, the worst-case value should be used to calculate the power dissipation.

\subsection*{Efficiency}

The efficiency \( \eta \) of the regulator is:

\[
\text{Efficiency (\%)} = \frac{\text{Power to load}}{\text{Input power}} \times 100\% = \frac{P_L}{P_{\text{in}}} \times 100\%
\]

Substituting the power values:

\[
\text{Efficiency (\%)} = \frac{V_L I_L}{V_{\text{unreg}} I_L} \times 100\% = \frac{V_L}{V_{\text{unreg}}} \times 100\%
\]

\subsection*{Advantages of the Series Regulator}

\begin{itemize}
    \item It is cost-effective and simple to implement.
    \item Easily integrated into IC fabrication.
\end{itemize}

\subsection*{Disadvantages of the Series Regulator}

\begin{itemize}
    \item In high-power applications, the voltage drop across the control element results in power losses and low efficiency.
    \item The output voltage must always be lower than the input voltage.
    \item Compared to SMPS and shunt regulators, it has a lower efficiency.
\end{itemize}

\subsection*{Questions}

\begin{enumerate}
    \item List the schemes for implementation of d.c. regulators (3 marks).
    
    \item Using a block diagram, explain the principle of operation of the above mentioned regulators (6 marks).
    
    \item The regulator in the figure below is designed to give a load current of 2A from an unregulated supply with \( V_i \) that varies from 15V to 40V. Evaluate the output voltage \( V_o \), given that the OPAMP has a gain of 100,000 and \( V_{beQ1} = V_{beQ2} = 0.65V \) when Q1 and Q2 are operating in the active regions.
    \begin{figure}[H]
    \centering
    \includegraphics[width=\linewidth]{Question.png}
    \caption{Regulator}
    \label{fig:question_image}
\end{figure}

    \item Briefly explain the possible causes of error in the computation.
\end{enumerate}

\textbf{(a)} Explain using a block diagram the principle of operation of closed loop shunt regulator. \hspace{1cm} (6 marks)

\textbf{(b)} The regulator in the figure below is required to supply a load current of 2A from an unregulated supply whose input varies from 20V to 30V. Estimate the value of \(V_o\) and the stability factor \(S_v\) given that the OPAMP has an open loop gain of 100,000. The transistors are silicon devices with \(V_{be_{active}} = 0.65V\) \hspace{1cm} (10 marks)
\begin{figure}[H]
    \centering
    \includegraphics[width=\linewidth]{Question2.png}
    \caption{Regulator}
    \label{fig:question_image}
\end{figure}


\textbf{(c)} State the criterion for selection of transistors given that the OPAMP has a current drive capability of 20mA.

\textbf{(d)} Modify the circuit to incorporate short circuit protection when the load current exceeds 2A.

\chapter{Logarithmic Amplifiers}

The output voltage \( V_o \) of a logarithmic amplifier is related to the input voltage \( V_{in} \) as:

\[
V_o = K \log V_{in}
\]

Where \( K \) is a constant.

Alternatively, the input voltage and output voltage are related by:

\[
e^{\frac{V_o}{K}} = V_{in}
\]

\subsection*{Practical Applications}
\begin{enumerate}
    \item \textbf{Analog computation:} Logarithmic amplifiers are widely used in analog computation due to their ability to handle exponential relationships.
    \item \textbf{Compressors:} In audio signal processing, logarithmic amplifiers are used in compressors to manage dynamic range in applications such as recording studios and public addressing systems.
    \item \textbf{Driving display devices:} Logarithmic amplifiers are often used to drive display devices that show logarithmic scales, such as sound level meters.
    \item \textbf{Log display indicators:} These amplifiers can be used in devices that need to display logarithmic quantities, such as light meters and radiation detectors.
\end{enumerate}

\subsection*{Theory of the Basic Diode Log Amplifier}
\begin{figure}[H]
    \centering
    \includegraphics[width=0.25\textwidth]{BasicLog.png}
    \caption{Basic diode Log Amp}
    \label{fig:question_image}
\end{figure}

For the diode, the current \( I_f \) is given by:

\[
I_f = I_o \left( e^{\frac{V_f}{\eta V_T}} - 1 \right) \tag{1}
\]

Where:
- \( I_o \) is the saturation current.
- \( V_f \) is the forward voltage across the diode.
- \( \eta \) is the ideality factor (typically around 1).
- \( V_T \) is the thermal voltage.

Assuming \( \frac{V_f}{\eta V_T} \gg 1 \), we can approximate:

\[
I_f = I_o e^{\frac{V_f}{\eta V_T}} \tag{2}
\]

For the circuit, the current \( I_f \) is also related to the voltage \( V_s \) and resistance \( R_i \) by:

\[
I_f = \frac{V_s}{R_i} \tag{3}
\]

Substituting equation (3) into equation (2):

\[
\frac{V_s}{R_i} = I_o e^{\frac{V_f}{\eta V_T}} \tag{4}
\]

Relating the forward voltage \( V_f \) to the output voltage \( V_o \), we have:

\[
V_f = -V_o \tag{5}
\]

Substituting equation (5) into equation (4):

\[
\frac{V_s}{R_i} = I_o e^{-\frac{V_o}{\eta V_T}} \tag{6}
\]

Finally, solving for \( V_s \), we get:

\[
V_s = I_o R_i e^{-\frac{V_o}{\eta V_T}} \tag{7}
\]
Taking the logarithm of both sides of equation (7), we get:

\[
\ln(V_s) = -\frac{V_o}{\eta V_T} \tag{8}
\]

Substituting the expression for \( V_s \), we have:

\[
V_o = -\eta V_T \ln \left( \frac{V_s}{I_o R_i} \right) \tag{9}
\]

Thus, the output voltage \( V_o \) is:

\[
V_o = K_1 \ln(K_2 V_s) \tag{10}
\]

Where:
- \( K_1 = -\eta V_T \)
- \( K_2 = \frac{1}{I_o R_i} \)

From equation (10), the basic diode log amplifier functions as a logarithmic amplifier. However, the accuracy of this circuit depends on the relationship between \( V_f \) and \( I_f \). It is found that this accuracy diminishes as \( I_f \) increases. To improve performance, the circuit must be used with small values of \( I_f \).

The practical realization of the basic log amplifier is achieved with a transistor log amplifier, where the base-emitter junction is used as the diode.
\begin{figure}[H]
    \centering
    \includegraphics[width=0.25\textwidth]{Shegone.png}
    \caption{Basic transistor Log Amp}
    \label{fig:question_image}
\end{figure}
Using Eber's Moll relations:

\[
I_e = I_o \left( e^{\frac{V_{be}}{\eta V_T}} - 1 \right) \tag{1}
\]

For \( \frac{V_{be}}{\eta V_T} \gg 1 \), we have:

\[
I_e = I_o e^{\frac{V_{be}}{\eta V_T}} \tag{2}
\]

For the transistor, \( I_c \approx I_e \), and from the circuit:

\[
I_c = I_1 \tag{3}
\]

From the equation \( I_1 = \frac{V_s}{R_i} \), we substitute into equation (2):

\[
I_1 = I_c = I_e = I_o e^{\frac{V_{be}}{\eta V_T}} \tag{5}
\]

Relating \( V_{be} \) to \( V_o \), we have:

\[
V_{be} = -V_o \tag{6}
\]

Substituting equation (6) into equation (5), we get:

\[
V_s = I_o e^{-\frac{V_o}{\eta V_T}} \tag{7}
\]
\[
R_i
\]
\[
V_s = \frac{e^{-V_o / \eta V_T}}{I_o R_i} \tag{8}
\]

Taking the natural logarithm of both sides:

\[
\ln V_s = -\frac{V_o}{\eta V_T} \tag{9}
\]
\[
I_o R_i
\]

Rearranging for \( V_o \):

\[
V_o = -\eta V_T \ln \left( \frac{V_s}{I_o R_i} \right) \tag{10}
\]

This can be rewritten as:

\[
V_o = K_1 \ln K V_s \tag{11}
\]

Where:
- \( \eta = 1 \) for Germanium
- \( \eta = 2 \) for Silicon
- \( V_T = \frac{kT}{q} \approx 0.0257 \, \text{V} \approx 26 \, \text{mV} \)

Thus,

\[
V_o = -0.026 \ln \left( \frac{V_s}{I_o R_i} \right) \tag{12}
\]

Using \( \ln(x) = 2.303 \log_{10}(x) \), we can rewrite this as:

\[
V_o = -0.026 \times 2.303 \ln V_s \tag{13}
\]
\[
I_o R_i
\]

\[
V_o = -0.05986 \ln \left( \frac{V_s}{I_o R_i} \right) \tag{13}
\]



\begin{itemize}
    \item Equation (8) relates the input voltage \( V_s \) to the output voltage \( V_o \) using the reverse saturation current \( I_o \) and resistance \( R_i \).
    \item In equation (9), the natural logarithm is applied to both sides.
    \item Equation (10) expresses the output voltage \( V_o \) in terms of the logarithm of \( V_s \), \( I_o \), and \( R_i \).
    \item Equation (11) is a rearranged form of the previous equation.
    \item Equation (12) gives a more specific expression for \( V_o \) using a typical value for \( V_T \).
    \item Equation (13) introduces a conversion from the natural logarithm to the base-10 logarithm for practical computation.
\end{itemize}

\[
V_o = -0.05986 \ln \left( \frac{V_s}{I_o R_i} \right) \tag{13}
\]


\begin{itemize}
    \item From equation (12) or (13), we see that the output \( V_o \) depends on \( I_o \), the reverse saturation current, which is temperature-dependent. Typically, \( I_o \) doubles for every 10°C increase in temperature, which can cause large errors if the ambient temperature increases.
    \item A solution to this problem is to design a matched transistor or matched diode log amplifier, which compensates for temperature variations and reduces the influence of \( I_o \).
\end{itemize}
\begin{figure}[H]
    \centering
    \includegraphics[width=\linewidth]{MatchedAmp.png}
    \caption{Matched Transistor Log Amp}
    \label{fig:question_image}
\end{figure}
Monolithic fabrication of $Q_1$, $Q_2$ for same temperature coupling.
\begin{figure}[H]
    \centering
    \includegraphics[width=0.25\textwidth]{DiffAmp.png}
    \caption{Considering first the difference amplifie}
    \label{fig:question_image}
\end{figure}
Since A3 is operated in linear region, we apply superposition theorem i.e. net response due to sum of individual responses
\begin{figure}[H]
    \centering
    \includegraphics[width=0.25\textwidth]{DueAmp.png}
    \caption{Output due to $V_a$ only}
    \label{fig:question_image}
\end{figure}


\[
V_{o1} = -\frac{R_f}{R_i} V_i \tag{1}
\]
\[
V_o = A_R V_a \tag{2}
\]
\[
V_o = - A V_a \tag{3}
\]
\begin{figure}[H]
    \centering
    \includegraphics[width=0.25\textwidth]{MarkAmp.png}
    \caption{Output due to $V_b$ only}
    \label{fig:question_image}
\end{figure}
\[
V_{o2} = V_P + \frac{V_P A_R}{R} \tag{4}
\]
\[
V_{o2} = V_P \left( 1 + \frac{A_R}{R} \right) \tag{4a}
\]
\[
V_{o2} = V_P \left( 1 + A \right) \tag{5}
\]
\[
V_P = \frac{V_b A_R}{R + A_R} = \frac{V_b A}{R(1 + A)} \tag{6}
\]

Substituting equation (6) in equation (5):

\[
V_{o2} = \frac{V_b A (1 + A)}{1 + A} = A V_b \tag{7}
\]

Therefore, the total output response is:

\[
V_{oT} = V_{o1} + V_{o2} = - A V_a + A V_b = A (V_a - V_b) \tag{8}
\]

The output \( V_a \) in equation (8) is the output of the basic transistor log amplifier comprising of \( A_1 \), while \( V_b \) is the output of \( A_2 \).

From earlier derived results for the basic transistor log amplifier:

\[
V_a = - \eta V_T \ln \left( \frac{V_{s1}}{I_{o1} R_s} \right) \tag{9}
\]

\[
V_b = - \eta V_T \ln \left( \frac{V_{s2}}{I_{o2} R_s} \right) \tag{10}
\]

Substituting equations (9) and (10) into equation (8):

\[
V_{oT} = A \left( - \eta V_T \ln \frac{V_{s2}}{I_{o2} R_s} + \eta V_T \ln \frac{V_{s1}}{I_{o1} R_s} \right) \tag{11}
\]

\[
V_{oT} = A \eta V_T \ln \left( \frac{V_{s1} I_{o2} R_s}{V_{s2} I_{o1} R_s} \right) \tag{12}
\]

\[
V_{oT} = A \eta V_T \ln \left( \frac{V_{s1}}{V_{s2}} \right) \tag{13}
\]

From equation (13), if \( Q_1 \) and \( Q_2 \) are matched, the output is independent of the reverse saturation current \( I_o \), and hence gives a better performance than the basic transistor log amplifier. However, the output is still dependent on \( V_T \). So long as \( V_T \) is constant, the output will be correct.
\begin{figure}[H]
    \centering
    \includegraphics[width=\linewidth]{TransitLife.png}
    \caption{Practical logarithmic amplifier using matched transistors}
    \label{fig:question_image}
\end{figure}


\[
V_o = V + \frac{V R_4}{R_3} \tag{1}
\]
\[
V_o = V \left( 1 + \frac{R_4}{R_3} \right) \tag{2}
\]

\[
V_o = V \frac{R_3 + R_4}{R_3} \tag{3}
\]

To determine \( V \), apply Kirchhoff's Voltage Law (KVL) along loop \( a \) \( b \) \( c \):

\[
-V_{be1} + V_{be2} = V \tag{1a}
\]

Also, from Eber's Moll relation for Q1 and Q2:

\[
I_c = I_o \left( e^{\frac{V_{be}}{\eta V_T}} - 1 \right) \tag{2}
\]

For \( \frac{V_{be}}{\eta V_T} \gg 1 \), we approximate:

\[
I_c \approx I_o e^{\frac{V_{be}}{\eta V_T}} \tag{3}
\]

\[
I_{c1} = I_{o1} e^{\frac{V_{be1}}{\eta V_T}} \tag{4}
\]

Taking the natural logarithm:

\[
\ln I_{c1} = \frac{V_{be1}}{\eta V_T} \tag{5}
\]

Thus:

\[
V_{be1} = \eta V_T \ln \left( \frac{I_{c1}}{I_{o1}} \right) \tag{6}
\]

Similarly:

\[
V_{be2} = \eta V_T \ln \left( \frac{I_{c2}}{I_{o2}} \right) \tag{7}
\]

Substituting equations (6) and (7) into equation (1a):

\[
-\eta V_T \ln \left( \frac{I_{c1}}{I_{o1}} \right) + \eta V_T \ln \left( \frac{I_{c2}}{I_{o2}} \right) \tag{8}
\]

\[
\eta V_T \ln \left( \frac{I_{c2}}{I_{o2}} \right) \cdot \frac{I_{o1}}{I_{o2}} = V \tag{9}
\]

For matched transistors \( Q1 \) and \( Q2 \), \( I_{o1} = I_{o2} \):

\[
V = \eta V_T \ln \left( \frac{I_{c2}}{I_{c1}} \right) \tag{10}
\]

\[
V = -\eta V_T \ln \left( \frac{I_{c1}}{I_{c2}} \right) \tag{11}
\]

Now, for the collector currents:

\[
I_{c1} = \frac{V_s}{R_1} \tag{12}
\]

\[
I_{c2} = \frac{V_R}{R_2} \tag{13}
\]

Thus:

\[
V = -\eta V_T \ln \left( \frac{V_s R_2}{V_R R_1} \right) \tag{14}
\]

From equation (1):

\[
V_o = V \frac{R_3 + R_4}{R_3} \tag{15}
\]

Substituting for \( V \):

\[
V_o = -\frac{R_3 + R_4}{R_3} \eta V_T \ln \left( \frac{V_s R_2}{V_R R_1} \right) \tag{16}
\]

Let \( K_1 = \eta V_T \) and \( K_2 = \frac{R_2}{R_1 V_R} \):

\[
V_o = -K_1 \ln \left( K_2 V_s \right) \tag{17}
\]

Substituting the values for the resistances:

\[
\eta = 1, \quad R_3 + R_4 = 0.5 + 29.5 = 60, \quad R_3 = 0.5
\]

\[
V_o = -0.0259 \times 60 \ln \left( V_s \times 20k \right) \tag{18}
\]

\[
V_o = -1.554 \ln \left( 0.2 V_s \right) \tag{19}
\]

\[
V_o = -3.5788 \log_{10} \left( 0.2 V_s \right) \tag{20}
\]

For the above practical circuit, the input dynamic range is typically from 2 mV to 20 V, and the output is between -12 V and -2.5 V.

For the circuit to be used practically, it must be calibrated. The following steps are recommended:

\begin{enumerate}
    \item Set \( V_s = 0 \), and adjust the potentiometer \( P_1 \) until \( V' = 0 \), as \( V' = A_d (V_d) \).
    \item Set \( V_s = V_R \frac{R_1}{R_2} \), implying:

    \[
    V_o = -V_T \frac{R_3 + R_4}{R_3} \ln \left( \frac{V_R R_1 R_2}{R_3 R_2 R_1 V_R} \right)
    \]

    If \( V_o \neq 0 \), adjust \( P_2 \) until \( V_o = 0 \).
\end{enumerate}

From equation (16), it is seen that \( V_o \) is temperature-dependent (dependent on \( V_T \)). If \( \frac{R_4}{R_3} \) can be made to decrease with temperature by using a positive temperature coefficient for \( R_3 \) (so that \( R_3 \) increases with temperature), then by suitable selection of values for \( R_3 \) and \( R_4 \), the entire expression for \( V_o \) can be made temperature-independent.

\[
V_o = -K_T \left( 1 + \frac{R_4}{R_3 (1 + \alpha_T T)} \right) \ln \left( \frac{V_s}{R_1 V_R} \right) \tag{21}
\]

\textbf{Question:} Design for \( V_o \) such that \( V_o \) is independent of \( T \). 

\[
\frac{\partial V_o}{\partial T} = 0 \tag{22}
\]

\begin{figure}[H]
    \centering
    \includegraphics[width=\linewidth]{Psalm.png}
    \caption{An improved Matched Transistor Log Amp with constant current source}
    \label{fig:question_image}
\end{figure}

D1 and D2 are protection diodes against large reverse bias of the B-E junction of \( Q_1 \) and \( Q_2 \). 

If \( V^+ \) of \( A_2 \) is equal to \( V \), this implies that \( V^- \) of \( A_2 \) is also \( V \). Therefore, the output voltage \( V_o \) is:

\[
V_o = V + I R_3 \tag{1}
\]

\[
V_o = V + \frac{V R_3}{R_T + R_4} \tag{2}
\]

\[
V_o = V \left( 1 + \frac{R_3}{R_T + R_4} \right) \tag{3}
\]

\[
V_o = V \frac{R_T + R_4 + R_3}{R_T + R_4} \tag{4}
\]

Considering the reference current \( I_{\text{ref}} \):

\[
I_{\text{ref}} = \frac{V_z}{R_5} \tag{5}
\]

If \( h_{FE}(Q_2) \gg 1 \), i.e., \( I_{b2} \approx 0 \) and zero bias current for \( A_2 \), then:

\[
I_{c2} \approx I_{\text{ref}} \tag{6}
\]

Apply Kirchhoff’s Voltage Law (KVL) to loop \( abc \):

\[
-V_{be1} + V_{be2} = V \tag{7}
\]

From Eber’s Moll relation for \( Q_1 \) and \( Q_2 \):

\[
I_{c1} = I_{o1} \left( e^{\frac{V_{be1}}{\eta V_T}} - 1 \right) \tag{8}
\]

For large \( \frac{V_{be1}}{\eta V_T} \), we approximate:

\[
I_{c1} \approx I_{o1} e^{\frac{V_{be1}}{\eta V_T}} \tag{9}
\]

Thus:

\[
\ln I_{c1} = \frac{V_{be1}}{\eta V_T} \tag{10}
\]

\[
\eta V_T \ln I_{c1} = V_{be1} \tag{11}
\]

Similarly:

\[
\eta V_T \ln I_{c2} = V_{be2} \tag{12}
\]

Substituting equations (11) and (12) into equation (7):

\[
\eta V_T \ln I_{c1} + \eta V_T \ln I_{c2} = V \tag{13}
\]

\[
\eta V_T \ln I_{c2} \times \frac{I_{o1}}{I_{o2}} = V \tag{14}
\]

For matched transistors \( Q_1 \) and \( Q_2 \), \( I_{o1} = I_{o2} \), so:

\[
V = \eta V_T \ln \left( \frac{I_{c2}}{I_{c1}} \right) \tag{15}
\]

\[
V = -\eta V_T \ln \left( \frac{I_{c1}}{I_{c2}} \right) \tag{16}
\]

From equation (4) and substituting into equation (16):

\[
V_o = V \frac{R_T + R_4 + R_3}{R_T + R_4} \tag{17}
\]

\[
V_o = -\eta V_T \ln \left( \frac{I_{c1} \left( R_T + R_4 + R_3 \right)}{I_{c2} \left( R_T + R_4 \right)} \right) \tag{18}
\]

Substitute for \( I_{c1} = \frac{V_s}{R_1} \) and \( I_{c2} = I_{\text{ref}} = \frac{V_z}{R_5} \):

\[
V_o = -\frac{R_T + R_4 + R_3}{R_T + R_4} \eta V_T \ln \left( \frac{V_s}{R_1} \times \frac{R_5}{V_z} \right) \tag{19}
\]

This concludes the calculation for \( V_o \) in terms of the reference voltage \( V_z \) and the various circuit components.

\chapter{Lectures}
\section*{Lecture 1: 16/09/2024}

\subsection{Course Introduction}

Mr. Ombura welcomed Moses, our class representative, together with Sephine, and all students enrolled in FEE 501: Applied Electronics. He hoped that everyone could see everything loud and clear. He requested that they ensure their microphones were turned on. He emphasized that it was crucial for interactive participation, especially since some students were attempting to log in. He typically used a full-screen presentation to maximize screen space, allowing him to hear any connection issues.

He asked to inform the students that if they joined the class late, he might not be able to admit them, as it disrupted the flow of the lecture.

\subsection{Course Overview}

Mr. Ombura stated that the course covered the following topics:

\begin{enumerate}
    \item Nonlinear Analog Systems
    \item Chopper Techniques
    \item Locking and Amplifiers
    \item Logarithmic and Anti-Log Amplifiers
    \item Multipliers and Dividers
    \item Function Generators
    \item Oscillators and Ramp Oscillators
    \item Relaxation Oscillators
    \item DC Power Supplies
    \item Capacitor Filters
    \item Rectifiers
    \item Voltage Regulators: Zener, Linear, and Switching Regulators
    \item Protection Circuits and Short Circuit Limits
\end{enumerate}

He noted that due to the extensive nature of these topics, assignments would be provided to ensure a solid understanding. While the course covered the basics, Power Electronics was a specialized field, and those pursuing it would gain a significant advantage.

\subsection{Basic Principles of Power Electronics}

\subsubsection{Diode Configurations}

Mr. Ombura explained that when connecting diodes:

\begin{itemize}
    \item \textbf{Series Connection}: Increases the blocking voltage. For instance, two diodes rated at 100V each could block up to 200V.
    \item \textbf{Parallel Connection}: Increases the current capacity. For example, two transistors each handling 10A in parallel could handle up to 20A.
\end{itemize}

\subsubsection{Inverters and Converters}

He further elaborated:

\begin{itemize}
    \item \textbf{Inverters}: Convert DC to AC. They are essential in solar energy systems where solar panels produce DC, but most household appliances require AC.
    \item \textbf{DC-DC Converters}: Convert DC voltage from one level to another. Examples include laptop power supplies that convert 240V AC to 19V DC using a full bridge rectifier, resulting in approximately 340V DC, which is then stepped down.
\end{itemize}

\subsubsection{Types of DC-DC Converters}

Mr. Ombura categorized DC-DC converters as follows:

\begin{enumerate}
    \item \textbf{Step-Down (Buck) Converters}: Reduce voltage from a higher to a lower level.
    \item \textbf{Step-Up (Boost) Converters}: Increase voltage from a lower to a higher level.
\end{enumerate}

\subsection{Applications of DC-DC Converters}

He identified several applications:

\begin{itemize}
    \item \textbf{Computers}: Desktop power supplies provide multiple DC voltages (e.g., +12V, -12V, +3.3V) from a higher input voltage.
    \item \textbf{Electric Vehicles}: Convert household voltage (e.g., 240V AC) to higher DC voltages (e.g., 400V DC) required for battery charging.
    \item \textbf{Televisions}: Modern TVs use DC-DC converters to power internal circuits, similar to computers.
\end{itemize}

\subsection{Variable Frequency Drives (VFD)}

Mr. Ombura discussed the evolution from cycloconverters to Variable Frequency Drives due to their higher efficiency. VFDs are widely used in:

\begin{itemize}
    \item Electric bikes and motorcycles
    \item Electric cars
    \item Controlling motor speeds by adjusting the AC output frequency
\end{itemize}

\subsection{Chopper Stabilized Amplifiers}

He explained that chopper stabilized amplifiers are designed to amplify very slowly varying, low-amplitude signals, which can be considered as DC signals. These are critical in applications such as:

\begin{itemize}
    \item \textbf{Transducers}: Measuring stress and deformation in structures using strain gauges.
    \item \textbf{Temperature Monitoring}: In closed-loop control systems using thermocouples.
    \item \textbf{Geological Measurements}: Monitoring ground movements, falls, and earthquakes.
\end{itemize}

\subsection{Operational Amplifiers (Op-Amps)}
Mr. Ombura highlighted that operational amplifiers are fundamental components in many electronic circuits, including chopper-stabilized amplifiers. Key points include:

\begin{itemize}[itemsep=0pt, topsep=2pt, leftmargin=4mm]
    \item \textbf{Differential Input Stage}: Compares two input signals.
    \item \textbf{Voltage Gain Stage}: Provides the necessary amplification.
    \item \textbf{Buffer Stage}: Ensures stability and protection with Class B outputs.
\end{itemize}

\textbf{Challenges with Op-Amps:}
\begin{itemize}[itemsep=0pt, topsep=2pt, leftmargin=4mm]
    \item \textbf{Temperature Dependency}: Collector current and gain vary with temperature, leading to offset voltages.
    \item \textbf{Noise}: Includes Johnson (thermal) noise and flicker noise, which can affect signal integrity.
    \item \textbf{Offset Voltages}: Result from mismatched transistors, causing inaccuracies in low-level signal amplification.
\end{itemize}

\textbf{Instrumentation Amplifiers:} Mr. Ombura explained that instrumentation amplifiers are used to address limitations in basic op-amp configurations when amplifying low-level, slowly varying signals. They offer high input impedance, superior offset voltage, and current characteristics, making them ideal for precise measurements.

\textbf{Practical Considerations:}
\begin{itemize}[itemsep=0pt, topsep=2pt, leftmargin=4mm]
    \item Design amplifiers to handle very low-frequency signals (approaching DC) without significant offset or noise.
    \item Use matched transistor pairs and temperature-controlled substrates to minimize drift.
    \item Consider chopper-stabilized amplifiers for superior temperature and drift management.
\end{itemize}

\textbf{Conclusion:} Mr. Ombura emphasized enhancing practical implementation of the Gilbert multiplier by incorporating output referencing and improving dynamic range through resistor \( R_1 \). Balancing linearity and gain is crucial for real-world applications. The next class will analyze predistortion circuits in detail.

\textbf{Illustration: Cascode Configuration}
\begin{figure}[H]
    \begin{minipage}[t]{0.45\textwidth}
        \centering
        \includegraphics[width=0.6\linewidth]{Lectures/CascodeLect1.png}
        \caption*{\small \textbf{Cascode Configuration:} This setup enhances gain and bandwidth.}
    \end{minipage}%
    \hfill
    \begin{minipage}[t]{0.5\textwidth}
        \subsubsection*{Comparison of Configurations}
        \begin{itemize}[itemsep=0pt, topsep=2pt, leftmargin=4mm]
            \item \textbf{Common Emitter:} High gain but limited high-frequency performance due to Miller capacitance.
            \item \textbf{Common Base:} Excellent high-frequency performance and isolation but lower voltage gain.
            \item \textbf{Cascode:} Combines high voltage gain (CE) with high-frequency performance (CB) for superior overall performance.
        \end{itemize}
        \subsubsection*{Summary:} 
        The \textbf{cascode configuration} is ideal for applications requiring high gain, excellent isolation, and superior high-frequency performance.
    \end{minipage}
\end{figure}

\textbf{Questions and Troubleshooting:} Students raised concerns about signal transmission and amplifier functionality. Issues such as network connectivity were addressed, and the necessity of recording lectures was discussed to accommodate technical difficulties.


\begin{tcolorbox}[colframe=orange!50!black, colback=white, coltitle=white, title={Why are most amplifiers unable to effectively amplify very slowly varying, low-amplitude signals, which can largely be considered D.C.?}]
   

\begin{enumerate}
    \item \textbf{Frequency Response:} Amplifiers are typically designed with a certain frequency response range. Signals that are close to DC (0 Hz) are at the very low end of the frequency spectrum and often fall outside the design bandwidth of most amplifiers. These amplifiers are optimized for higher frequencies and may have capacitive coupling that blocks DC components.
    
    \item \textbf{Offset and Drift:} Amplifiers can experience offset voltage and drift over time, which can significantly impact their ability to accurately amplify low amplitude DC signals. These offsets and drifts can introduce errors and noise that are comparable in magnitude to the signal itself, making the amplification unreliable.
    
    \item \textbf{Noise:} Low amplitude signals are often susceptible to noise, including thermal noise, flicker noise, and other sources of interference. Amplifying such signals without introducing significant noise and distortion requires careful design and often specialized low-noise amplifiers.
    
    \item \textbf{Power Supply Rejection Ratio (PSRR):} Amplifiers need to maintain a stable output despite variations in the power supply. DC and very low-frequency signals can be affected by fluctuations in the power supply, leading to inaccurate amplification.
    
    \item \textbf{Input Bias Current:} In some amplifier designs, the input bias current can create a voltage drop across the input impedance, leading to inaccuracies when dealing with low amplitude DC signals. This effect can be more pronounced at low frequencies.
    
    \item \textbf{Thermal Effects:} Amplifiers generate heat during operation, which can lead to thermal drift and affect the stability of the amplification for DC signals. Temperature changes can cause variations in the amplifier's characteristics, leading to signal distortion.
\end{enumerate}

To amplify very slowly varying or DC signals effectively, specialized instrumentation amplifiers or DC-coupled amplifiers are used. These amplifiers are designed to handle low-frequency or DC signals with high precision, low noise, and minimal offset drift.
\end{tcolorbox}
\subsection{Future Topics}

In the next class, we will focus on the internal construction of chopper stabilized amplifiers, including modulators, AC amplifiers, and demodulators. We will also explore practical circuit designs and address common challenges faced in amplifier configurations.
\begin{center}\rule{0.5\linewidth}{0.5pt}\end{center}

\section*{Lecture 2: 19/09/2024}
Low-level signals that vary very slowly with respect to time can be considered DC signals or very slowly varying DC signals. These signals operate within the DC frequency range, necessitating the use of amplifiers capable of functioning up to DC frequencies. As discussed in the third-year amplifier course, only an emitter-coupled amplifier, also known as a differential amplifier, can amplify signals all the way down to DC. In the previous class, it was demonstrated that the best example of a good DC amplifier is the operational amplifier.

\subsection{Operational and Instrumentation Amplifiers}
Operational amplifiers are so named because they are used to perform mathematical operations and are integral components in computers. A modified version, known as the instrumentation amplifier, is optimized for very low noise. This optimization is achieved by minimizing the noise generated by the transistors, especially those in the differential pair.

\subsubsection{Noise in Amplifiers}
In these types of amplifiers, various components contribute to noise generation:
\begin{itemize}
    \item \textbf{Resistors:} The resistors used to bias and couple the transistors generate noise. The type of resistor affects the noise characteristics:
    \begin{itemize}
        \item \textit{Metal Oxide Resistors:} High-quality resistors made from metal oxide generate minimal noise.
        \item \textit{Carbon Resistors:} Commonly found in shops and laboratories, these resistors generate significant noise.
    \end{itemize}
\end{itemize}
Even with the best instrumentation amplifiers, noise in a much higher frequency band than the input signal can still pose challenges.

\subsection{Drift in Instrumentation Amplifiers}
Instrumentation amplifiers suffer from \textit{V\textsubscript{IO}} (input offset voltage) and \textit{I\textsubscript{IO}} (input offset current) drift. These drifts are on the same order of magnitude as the signal to be amplified and are temperature-dependent. For example, in Nairobi, laboratory temperatures can range from 15°C in the morning to 22–25°C in the afternoon, affecting the output of transistors and the overall amplifier. This drift complicates the differentiation between the useful signal and noise, as the drift voltages and currents are comparable to the amplified signal.

\subsection{Chopper-Stabilized Amplifiers}
To address the drift issues in DC amplifiers, chopper-stabilized amplifiers offer superior temperature and drift performance. These amplifiers consist of three main modules:
\begin{enumerate}
    \item \textbf{Modulator:} Converts the slowly varying DC signal into a higher frequency amplitude-modulated signal by chopping the input signal using a switch controlled by a square or rectangular wave.
    \item \textbf{AC Amplifier (Demodulator):} Similar to a high-quality operational amplifier with capacitors that block DC, ensuring only the AC component is amplified.
    \item \textbf{Demodulator:} Converts the amplified AC signal back into a DC signal.
\end{enumerate}

\subsubsection{Modulation Techniques}
Common modulation techniques include:
\begin{itemize}
    \item \textit{Pulse Amplitude Modulation (PAM)}
    \item \textit{Double-Sideband (DSB) Modulation}
    \item \textit{Frequency Shift Keying (FSK)}
    \item \textit{Quadrature Phase Shift Keying (QPSK)}
\end{itemize}
For chopper amplifiers, pulse amplitude modulation is primarily used.

\subsubsection{Chopper Switches}
The chopper switch is critical in converting the DC signal to a modulated signal. Ideally, a chopper switch should have:
\begin{itemize}
    \item \textbf{Zero On Resistance:} To prevent loss of input voltage.
    \item \textbf{Infinite Off Resistance:} To ensure no signal passes when the switch is open.
    \item \textbf{High-Speed Switching:} To handle rapid modulation requirements.
    \item \textbf{Good Isolation:} Prevents control voltage from appearing at the output.
    \item \textbf{No Offset Voltage:} To avoid introducing errors into the output signal.
    \item \textbf{Zero Control Power:} Ideally, no power is required to activate the switch.
    \item \textbf{Infinite Current and Voltage Handling:} Can pass infinite current when on and block infinite voltage when off.
\end{itemize}
In practice, achieving these ideal characteristics is challenging, leading to the use of different types of switches.

\subsubsection{Types of Chopper Switches}
\subsubsection{Electromechanical Relays}
Used primarily in the early 1960s, electromechanical relays come in two types:
\begin{itemize}
    \item \textbf{Normal Relays:} Larger size, require more power, slower switching speeds (typically below 60 Hz), and prone to contact bounce, causing signal distortion.
    \item \textbf{Reed Relays:} Smaller, faster, require less power, but were rarely used beyond the mid-1960s due to their diminishing advantages over time.
\end{itemize}
Advantages of electromechanical relays include:
\begin{itemize}
    \item On/off characteristics close to ideal switches.
    \item No drift in output terminals.
    \item Good isolation between control voltage and output.
\end{itemize}
Disadvantages:
\begin{itemize}
    \item Low switching speeds.
    \item Contact bounce leading to signal distortion.
    \item Bulky construction and high power consumption.
    \item Not suitable for remote applications due to high energy requirements.
\end{itemize}

\subsubsection{Semiconductor Switches}
In modern applications, semiconductor switches have largely replaced electromechanical relays. Types include:
\begin{itemize}
    \item \textbf{BJT Transistors:} High switching speeds but suffer from large offset voltages (approximately 0.1 V for germanium and 0.2 V for silicon transistors), which can negate the amplified signal.
    \item \textbf{Field Effect Transistors (FETs):} Provide better control and lower noise compared to BJTs.
    \item \textbf{Metal Oxide FETs:} Offer enhanced performance characteristics.
\end{itemize}
\subsection{Challenges with BJT Switches}
Although BJTs have high switching speeds, their large offset voltages can severely impact signal integrity. For example, a 0.2 V drop across a BJT switch can nullify a 50 mV signal. Additionally, BJTs are unidirectional; they cannot effectively switch signals when polarity reverses, limiting their applicability in bidirectional signal processing.
\subsection{Illustrations}
\begin{figure}[H]  % Ensure the figure appears exactly here
    \begin{minipage}[b]{0.5\textwidth}  % First minipage for the image
        % tcolorbox with image
        \begin{tcolorbox}[width=\textwidth, colback=white, boxrule=0.5mm, colframe=black, rounded corners]

            \includegraphics[width=\linewidth]{Lectures/Lecture2AC.png}  % Image path
            {\footnotesize \textbf This image illustrates the block diagram of the signal processing process}
        \end{tcolorbox}
    \end{minipage}%
    \hfill  % Create space between the minipages
    \begin{minipage}[b]{0.5\textwidth}  % Second minipage for the text
        \small
        \subsubsection*{Comparison}
Operational Amplifiers were named so because they were used to perform mathematical operations.
The difference between the general op-amp and instrumentation op-amp is that the instrumentation op-amp is actually optimized to generate very low noise.

    \end{minipage}
\end{figure}
\begin{figure}[H]  % Ensure the figure appears exactly here
    \begin{minipage}[b]{0.4\textwidth}  % First minipage for the image
        % tcolorbox with image
        \begin{tcolorbox}[width=\textwidth, colback=white, boxrule=0.5mm, colframe=black, rounded corners]

            \includegraphics[width=\linewidth]{Lectures/AfterMod.png}  % Image path
            {\footnotesize \textbf This image illustrates what happens after modulation}
        \end{tcolorbox}
    \end{minipage}%
    \hfill  % Create space between the minipages
    \begin{minipage}[b]{0.4\textwidth}  % Second minipage for the text
        \small
        \subsubsection*{}
Here some sort of convolution takes places with:
$$
V_{0}(t) = V_{i}(t) * V_{c}(t)
$$

    \end{minipage}
\end{figure}
\subsection{Conclusion}
Chopper-stabilized amplifiers provide a robust solution to drift and noise issues in DC amplifiers. However, selecting the appropriate switching mechanism is crucial to maintain signal integrity and performance. While electromechanical relays offered a near-ideal switching characteristic in the past, semiconductor switches have become the preferred choice despite their inherent challenges, such as offset voltages and unidirectional switching limitations.
\begin{center}\rule{0.5\linewidth}{0.5pt}\end{center}

\section*{Lecture 3: 23/09/2024}

\subsection{Introduction}
Our topic today is the chopper stabilized amplifier. It is called a "chopper" because the input signal is chopped before being fed into an AC amplifier. Chopping performs what is known as frequency translation. Frequency translation is necessary because we want the signal to be modulated with the voltages of the AC amplifier. This process is referred to as amplitude modulation (AM), specifically pulse amplitude modulation.

\subsection{Frequency Translation}
Frequency translation involves shifting the frequency of the input signal. For example, if chopping is performed at 10 kHz and the information signal is a slowly varying DC signal (less than 1 Hz), the signal will be translated from a baseband with a slow-varying DC signal to around 10 kHz. The student who emailed me was not on the list, but let's continue.

Frequency translation is necessary because we want the information signal to be well above the drift voltage frequencies, which are very low. This ensures that when we recover the signal at the end of the chopper stabilized amplifier, we can discriminate between the drift voltages and our information signal.

\subsection{Chopping Mechanism}
To perform chopping, we require switches. We use semiconductor switches, which consist of BJTs (Bipolar Junction Transistors), FETs (Field Effect Transistors), and MOSFETs (Metal-Oxide-Semiconductor Field-Effect Transistors). BJTs are rarely used because the offset voltage of the BJT is quite large.

In the last class, I presented a diagram where an NPN transistor acts as a switch for chopping. For a silicon BJT, the offset voltage is about 0.2 volts for small signal transistors, while germanium transistors have about 0.1 volts. With a 50 kHz chopping frequency, the voltage drop across the transistor switch (collector-emitter junction) prevents the signal from appearing at the output.

Another disadvantage of BJTs is that for an NPN transistor to chop, the collector must remain positive relative to the emitter. If the input signal polarity changes, making the collector negative, the transistor cannot operate in that region because the collector-emitter junction becomes reverse-biased. Therefore, BJTs are never used for chopping because of the large voltage drop across the collector and emitter when used as a switch, and they can only chop in one direction.

\subsection{Field Effect Transistors (FETs)}
The next commonly used switch is the Field Effect Transistor (FET). FETs are preferred because they offer several advantages:
\begin{enumerate}
    \item \textbf{No Offset Voltage:} FETs pass through the origin, meaning there is no offset voltage between the drain and the source.
    \item \textbf{High Switching Speed:} FETs have a very high switching speed, making them suitable for applications like satellite LNBs (Low-Noise Block converters) that operate at very high frequencies.
    \item \textbf{Bidirectional Switching:} FETs can chop both positive and negative voltages due to their ability to conduct in both directions.
    \item \textbf{Low On-Resistance:} Although not ideal, the on-resistance of FETs is approximately 25 ohms for small signal devices. There are specialized FETs manufactured specifically for use in choppers.
\end{enumerate}

\subsection{Advantages of FETs}
FETs offer several benefits over BJTs, including no offset voltage between drain and source, high switching speeds, bidirectional operation, and low on-resistance. Even though the on-resistance is not ideal (practically, we aim for zero when closed and infinity when open), FETs provide some of the best characteristics achievable with practical devices. Additionally, they are small in size and have low power requirements for switching, which refers to the control voltage needed to operate the switch.

\subsection{Disadvantages of FETs}
Despite their advantages, FETs have non-ideal on and off characteristics. The on-resistance is about 25 ohms, and the off-resistance is around $10^{10}$ ohms. Other disadvantages include:
\begin{itemize}
    \item \textbf{Switching Spikes:} Due to incomplete isolation between the output terminal and the control voltage, switching spikes may appear at the output. This is caused by internal capacitances, such as $C_{GD}$ (Gate-Drain Capacitance) and $C_{GS}$ (Gate-Source Capacitance), which can generate spikes when driven by rectangular pulses.
\end{itemize}
These switching spikes add errors to the chopped signal. However, there are methods to mitigate these issues, which we will explore later.

\subsection{MOSFETs as Switches}
The MOSFET (Metal-Oxide-Semiconductor Field-Effect Transistor) is a modification of the FET, featuring a metal-oxide-silicon layer between the gate and the channel, which significantly increases input impedance (typically around $10^{12}$ ohms). This insulation layer ensures that the power required to drive the MOSFET on or off is almost zero, as the gate current is negligible.

The drain-source characteristics of MOSFETs are identical to those of FETs, but the insulated gate allows for lower power consumption, making MOSFETs the preferred choice in most modern chopper stabilized amplifiers. MOSFETs are mass-produced and cost-effective compared to other FETs, further enhancing their suitability.

\subsection{Electromechanical Switches}
Electromechanical switches, such as relays, are not commonly used in modern applications due to their bulkiness, high power consumption, and expense. They are typically replaced by semiconductor switches like MOSFETs. However, for educational purposes, it's useful to understand the differences:
\begin{itemize}
    \item \textbf{Series Switches:} Utilize relays to switch in series.
    \item \textbf{Shunt Switches:} Incorporate relays for shunting current.
\end{itemize}
Given their disadvantages, electromechanical switches are largely obsolete in favor of semiconductor alternatives.

\subsection{AC Amplifier in Chopper Stabilized Systems}
The next component in our chopper stabilized amplifier is the AC amplifier. It is designed to amplify only AC signals and is constructed by placing a capacitor in series with a low-noise instrumentation amplifier. Typically, this amplifier is originally a DC amplifier (operational amplifier) modified with DC blocking capacitors to prevent DC drift from passing through to the signal.
\begin{figure}[h!]
    \centering
    \includegraphics[width=0.4\textwidth]{Lectures/Lecture2AC.png} % Replace with your image file

\end{figure}


The AC amplifier must have adequate bandwidth to pass all sidebands. The function of the AC amplifier is to amplify signals whose frequencies are well above the drift voltages of both the chopper switch at its input and the amplifier itself. The capacitors at the input and output act as high-pass filters, allowing only high-frequency signals to propagate to the next stage.

\subsection{Operational Amplifier Characteristics}
\begin{figure}[h!]
    \centering
    \includegraphics[width=0.25\textwidth]{OpAMP.png} % Replace with your image file

\end{figure}

Operational amplifiers (op-amps) used in these systems, such as the widely-known LM741, have specific electrical characteristics that are crucial for their performance:
\begin{itemize}
    \item \textbf{Input Offset Voltage:} The voltage difference required at the input terminals to make the output zero, typically ranging from 1 mV to 5 mV.
    \item \textbf{Input Resistance:} At 25°C, the input resistance ranges from 3 MΩ to about 2 MΩ, facilitating efficient voltage transfer from the source to the amplifier.
    \item \textbf{Slew Rate:} Approximately 0.5 V/μs, indicating the maximum rate at which the output voltage can change. Signals faster than this rate will not be accurately followed by the op-amp.
    \item \textbf{Input Offset Current:} Ranges from 3 to 30 nA, with a maximum of 170 nA. This current contributes to the voltage at the input and can affect the accuracy of the amplification.
    \item \textbf{Temperature Drift:} The input offset current drifts by about 0.5 nA/°C, affecting the stability of the amplifier with temperature changes.
\end{itemize}

\subsection{Design Considerations}
As engineers, designing a chopper stabilized amplifier involves simulating the circuit, often using software like Proteus. Students are encouraged to simulate these circuits to understand the interaction between different components:

\begin{itemize}
    \item \textbf{Modulator:} The first part consists of the chopper switch.
    \item \textbf{AC Amplifier:} Uses blocking capacitors and a low-noise instrumentation amplifier to amplify the chopped signal.
\end{itemize}

\subsection{Example Circuit Analysis}
Consider a typical chopper stabilized amplifier circuit using an LM741 op-amp:

\begin{figure}[h!]
    \centering
    \includegraphics[width=\linewidth]{Lectures/LM741.png}

    \label{fig:example_centered}
\end{figure}
\begin{itemize}
    \item \textbf{Power Supply:} Typically $\pm15$ V, verified against the manufacturer's recommended operating conditions.
    \item \textbf{Biasing:} Utilizes a $V_{BE}$ multiplier to provide the necessary bias voltage and prevent thermal runaway in the output transistors.
    \item \textbf{Protection:} Includes short-circuit protection mechanisms using transistors and resistors to safeguard against excessive current.
\end{itemize}
\subsection{Design Assignment:}
Design and simulate the LM741 op amp on Proteus. It will be worth 5 marks of your C.A.T.
\subsection{Illustration}

\begin{figure}[H]
    \centering
    \includegraphics[width=\linewidth]{Lectures/ExampleLM.png}
    \caption{Illustrative Example of the 741 Op Amp}
    \label{fig:example_centered}
\end{figure}

\begin{figure}[H]
    \centering
    % First Image
    \begin{minipage}[t]{0.45\textwidth}
        \centering
        \includegraphics[width=\linewidth, height=5cm, keepaspectratio]{Lectures/SimplifiedLM.png}
        \small \textbf{Simplified, Conceptual Schematic Diagram of the 741 Op Amp}
    \end{minipage}%
    \hfill
    % Second Image
    \begin{minipage}[t]{0.45\textwidth}
        \centering
        \includegraphics[width=\linewidth, height=5cm, keepaspectratio]{Lectures/SimplifiedLM1.png}
        \small \textbf{DC Analysis of the 741 Op Amp}
    \end{minipage}
\end{figure}

\begin{figure}[H]
    \centering
    % First Image
    \begin{minipage}[t]{0.45\textwidth}
        \centering
        \includegraphics[width=\linewidth, height=5cm, keepaspectratio]{Lectures/SimplifiedLM2.png}
        \small \textbf{Simplified Schematic of the 741 Op Amp with Idealized Biasing}
    \end{minipage}%
    \hfill
    % Second Image
    \begin{minipage}[t]{0.45\textwidth}
        \centering
        \includegraphics[width=\linewidth, height=5cm, keepaspectratio]{Lectures/SimplifiedLM3.png}
        \small \textbf{Input Stage Biasing of the 741 Op Amp}
    \end{minipage}
\end{figure}

\begin{figure}[H]
    \centering
    % First Image
    \begin{minipage}[t]{0.45\textwidth}
        \centering
        \includegraphics[width=\linewidth, height=5cm, keepaspectratio]{Lectures/SimplifiedLM4.png}
        \small \textbf{Darlington Gain Stage Biasing}
    \end{minipage}%
    \hfill
    % Second Image
    \begin{minipage}[t]{0.45\textwidth}
        \centering
        \includegraphics[width=\linewidth, height=5cm, keepaspectratio]{Lectures/SimplifiedLM5.png}
        \small \textbf{Output Stage Biasing of the 741 Op Amp}
    \end{minipage}
\end{figure}

\subsection{$V_{BE}$ Multiplier Function}
The $V_{BE}$ multiplier circuit ensures proper biasing of the output stage to eliminate crossover distortion and compensate for temperature variations. It adjusts the voltage between points A and B to maintain a stable operating point despite temperature changes, thus preventing thermal runaway in the output transistors.

\subsection{Protection Mechanisms}
Short-circuit protection is implemented by monitoring the voltage across specific resistors and transistors. If the voltage exceeds a threshold (e.g., 0.5 V across a 25 $\Omega$ resistor), the protection circuit limits the current to prevent damage to the amplifier components.

\subsection{Conclusion}
In summary, chopper-stabilized amplifiers rely on frequency translation achieved through chopping, high-speed semiconductor switches like FETs and MOSFETs, and carefully designed AC amplifiers to minimize drift and noise. Proper biasing and protection mechanisms are essential to ensure stable and accurate amplification. Understanding these components and their interactions is crucial for designing effective chopper-stabilized amplifier systems.

\begin{center}\rule{0.5\linewidth}{0.5pt}\end{center}


\section*{Lecture 4: 26/09/2024}

\subsection{Introduction}

In this lecture, we delve into chopper-stabilized amplifiers, focusing on their composition and the characteristics of operational amplifiers (op-amps) used within these systems. We will explore the modules that constitute a CH stabilized amplifier, analyze a typical op-amp circuit, and discuss essential parameters that influence amplifier performance.

\subsection{Chopper Stabilized Amplifier Modules}

A Chopper stabilized amplifier comprises three primary modules:

\begin{enumerate}
    \item \textbf{Modulator}
    \item \textbf{AC Amplifier}
    \item \textbf{Demodulator}
\end{enumerate}

A high-quality AC amplifier within this system is an instrumentation amplifier. Unlike standard op-amps, instrumentation amplifiers utilize high-quality components with very low noise, ensuring minimal noise amplification across stages.

\subsection{Op-Amp Circuit Design}

\subsubsection{Circuit Stages}

Examining the second diagram of a typical Texas Instruments op-amp, the circuit can be divided into three distinct blocks:

\begin{enumerate}
    \item \textbf{First Stage: Inverting and Non-Inverting Circuits}
    \begin{itemize}
        \item \textbf{Components}: Q5 and Q6
        \item \textbf{Function}: Acts as the initial amplification stage with protection provided by Q5 and Q6.
    \end{itemize}

    \item \textbf{Middle Stage: Biasing through Current Mirrors}
    \begin{itemize}
        \item \textbf{Components}: Q8, Q10, and Q11
        \item \textbf{Function}: Maintains biasing through current mirrors ensuring stable operation.
    \end{itemize}

    \item \textbf{Second Stage: Active Load and Output Stage}
    \begin{itemize}
        \item \textbf{Components}: Q15, Q16, Q13, Q14, and Q20
        \item \textbf{Function}: Comprises the active load (primarily Q13) and the Class B output stage utilizing emitter followers (Q14 and Q20).
    \end{itemize}
\end{enumerate}

\subsubsection{Noise Considerations}

The gain of the output stage is just below unity due to the presence of emitter followers. It is crucial to use resistors that generate minimal noise to prevent noise from the first stage from being amplified by the second stage. The thermal noise generated by a resistor is given by:

\[
V_{\text{noise (RMS)}} = \sqrt{4kTBR}
\]

where:
\begin{itemize}
    \item \( T \) = Temperature in Kelvin
    \item \( B \) = Bandwidth
    \item \( R \) = Resistance in ohms
\end{itemize}

Metal oxide resistors are typically preferred for their low noise characteristics. All components are fabricated into an IC chip to maintain consistency and performance.

\subsection{Characteristics of the 741 Op-Amp}

\subsubsection{Grades and Applications}

The 741 op-amp comes in various grades tailored for different markets:
\begin{itemize}
    \item \textbf{Defense Market}
    \item \textbf{Commercial/Industrial Markets}
    \item \textbf{Military Applications}
\end{itemize}

These grades vary in quality and price, designed to operate in environments ranging from extreme cold regions like Greenland and Antarctica to deserts.

\subsubsection{Input Offset and Bias Currents}

\begin{itemize}
    \item \textbf{Input Offset Voltage}: Typically around 5 mV.
    \item \textbf{Input Offset Current}: The current required to balance the op-amp when pin 3 (inverting input) and pin 2 are grounded. This current is essential for achieving zero output voltage.
    \item \textbf{Bias Currents}: Approximately 100 nanoamperes, ensuring proper operation by maintaining around 0.1 milliamps through the inputs.
\end{itemize}

\subsubsection{Rejection Ratios}

\begin{itemize}
    \item \textbf{Common Mode Rejection Ratio (CMRR)}: Ideally infinite, but practically around 100 dB.
    \item \textbf{Power Supply Rejection Ratio (PSRR)}: Typically around 20 microvolts per volt, indicating the ability to minimize power supply variations from affecting the output.
\end{itemize}

\subsection{Slew Rate and Capacitor \( C_1 \)}

\subsubsection{Definition and Importance}

The slew rate defines the maximum rate at which the output voltage can change and is determined by the capacitor \( C_1 \).

\subsubsection{Calculation of Slew Rate}

If \( V_{\text{not}}(t) = A \cos(\Omega t) \), then:

\[
\frac{dV_{\text{not}}(t)}{dt} = -A \Omega \sin(\Omega t)
\]

Taking the magnitude:

\[
C A \Omega = I_{\text{in}}
\]

At maximum \( \Omega t \):

\[
\frac{V_{\text{not}} \cdot \Delta t}{C} = I_{\text{in}}
\]

Thus, the slow rate is:

\[
\Delta t = \frac{I_{\text{in}}}{C}
\]

To increase the slew rate, \( C \) should be minimized. However, \( C \) also provides feedback between the output and input, requiring careful adjustment to maintain amplifier stability.

\subsubsection{Impact on Amplifier Performance}

A slow rate of one volt per microsecond implies:

\[
\Delta V \cdot \Delta t = 1 \, \mu\text{s}
\]

This parameter, along with the unity gain frequency (frequency at which gain is one) and the full power bandwidth (e.g., 50 kHz), determines the amplifier's performance in amplifying signals without distortion.

\subsection{Block Diagram of a Chopper Stabilized Amplifier}

The complete CH DC amplifier system includes:
\begin{itemize}
    \item \textbf{Modulator}
    \item \textbf{AC Amplifier}
    \item \textbf{Demodulator}
    \item \textbf{Oscillator} (drives square waves)
\end{itemize}

These components work synchronously, ensuring that any changes in one part of the system are reflected across all modules. The CH DC amplifier is termed so because it amplifies slowly varying DC voltages, analogous to a carrier system in telecommunications.

\subsection{Carrier-Based Implementation}

\subsubsection{Modulator as a Double-Sideband (DSB) Multiplier}

The modulator multiplies the information signal \( M(t) \) with a carrier signal \( \cos(\Omega_c t) \). In the frequency domain, this results in a spectrum containing the original signal and sidebands at \( \Omega_c \).

\subsubsection{Trigonometric Identities and Signal Spectrum}

Using the identity:

$$
\cos(A) \cos(B) = \frac{1}{2} \left[ \cos(A+B) + \cos(A-B) \right]
$$


we can simplify the modulation process, separating the signal into components at baseband and twice the carrier frequency.

\subsubsection{Signal Recovery with Low-Pass Filter}

To extract the original signal from the modulated output, a low-pass filter with a cutoff frequency slightly higher than the highest frequency component of \( M(t) \) is employed. The overall gain of the system is:

\[
\frac{A^2 K}{2}
\]

resulting in the recovered signal:

\[
\frac{A^2 K}{2} M(t)
\]

Proper design of the modulator and demodulator is essential to minimize drift voltages and prevent errors in signal amplification.

\subsection{Alternative Implementation: Positive Gain-Negative Gain Amplifier}

\subsubsection{Modulator and Demodulator Design}

This implementation utilizes:
\begin{itemize}
    \item \textbf{Positive Gain Modulator}
    \item \textbf{Negative Gain Demodulator}
\end{itemize}

Represented typically with op-amps, these configurations require precise control of gain to maintain signal integrity.

\subsubsection{Switch Operation and F Switch Characteristics}

An F switch, incorporating a diode, ensures that the voltage does not exceed certain thresholds:
\begin{itemize}
    \item \textbf{Positive Voltage}: Clamped at approximately 0.1 volts.
    \item \textbf{Negative Voltage}: Requires sufficient bias (e.g., -0.3 to -4 volts) to turn off the MOSFET completely.
\end{itemize}

\subsubsection{Circuit Configuration with Switch States}

\begin{enumerate}
    \item \textbf{Switch Closed}: Connects \( R_6 \) and \( R_7 \) to Earth, configuring the circuit as a simple inverting amplifier with feedback resistance \( R_F \) and input resistance \( R \).
    
    \[
    V_{\text{out}} = -\frac{R_F}{R} V_I
    \]
    
    If \( R_F = R \):
    
    \[
    V_{\text{out}} = -V_{MT}
    \]
    
    \item \textbf{Switch Open}: Alters the circuit configuration, effectively splitting the supply and treating it as an amplifier with two inputs (inverting and non-inverting), resulting in an overall output:
    
    \[
    V_{\text{out}} = V_{\text{out1}} + V_{\text{out2}}
    \]
    
    where:
    \begin{itemize}
        \item \( V_{\text{out1}} \) corresponds to the inverting input (\( U1 \))
        \item \( V_{\text{out2}} \) corresponds to the non-inverting input (\( U2 \))
    \end{itemize}
\end{enumerate}

\subsection{Conclusion}

Understanding the sectional components of chopper-stabilized amplifiers and the nuanced characteristics of op-amps like the 741 is crucial for designing high-performance amplification systems. Proper configuration of circuit stages, noise minimization, and precise control of parameters such as slew rate and gain ensure the stability and accuracy of the amplifier. Future lectures will build upon these foundations, exploring advanced implementations and optimizations.

If there are any questions, feel free to ask. We will continue this discussion in the next session.
\begin{center}\rule{0.5\linewidth}{0.5pt}\end{center}

\section*{Lecture 5: 30/10/2024}

\subsection{Introduction}
In this lecture, we will discuss the behavior of switches within circuit configurations, specifically focusing on their operation in open and closed states. We will analyze the equivalent circuits, explore the characteristics of FETs (Field-Effect Transistors), and conduct an error analysis when using FETs as switches in chopper amplifiers.
\begin{figure}[h]
    \centering
    \includegraphics[width=0.4\textwidth]{Stab1.png}
    \caption{Chopper stabilized Amplifier}
\end{figure}
\subsection{Case 1: Switch Closed}
When the switch \( S \) is closed, the control voltage to the FET is approximately zero. Under this condition, we can draw an equivalent circuit where the input \( V_{\text{MT}} \) is connected to the ground through the input resistance \( R_{\text{in}} \) and feedback resistance \( R_8 \). Assuming that \( V_{\text{MT}} \) is an ideal voltage source with an internal resistance much less than \( R_7 \), the resistor \( R_S \) does not significantly affect the circuit's behavior.

This configuration resembles a standard inverting amplifier. Therefore, the output voltage \( V_{\text{out}_1} \) can be expressed as:
\[
V_{\text{out}_1} = -\frac{R_{\text{feedback}}}{R_{\text{in}}} V_{\text{MT}}
\]
Given that \( R_{\text{feedback}} = R \) , the output simplifies to:
\[
V_o^{'}(t) = -V_{\text{MT}}
\]
This is the first result obtained: when the switch is closed, the output is the input multiplied by \(-1\).

\subsection{Case 2: Switch Open}
Next, we consider the scenario where the switch \( S \) is open, meaning the FET is off. In this case, the equivalent circuit changes, and the input voltage \( V_{\text{MT}} \) is connected through \( R_7 \) and \( R_8 \) to produce the output voltage \( V_o^{''}(t) \).

Assuming the op-amp operates in its linear region and employing the principle of superposition, the total output \( V_{\text{op}} \) is the sum of the responses due to the sources connected to the inverting and non-inverting inputs. Specifically:
\[
V_{op}(t) = V_o^{'}(t)  + V_o^{''}(t)  = -V_{\text{MT}} + 2V_{\text{MT}} = V_{\text{MT}}
\]
Thus, when the switch is open, the output is the input multiplied by \( +1 \). Combining both cases, we conclude that the amplifier behaves as a positive-negative amplifier:
\[
\begin{cases}
S \text{ closed} & \Rightarrow V_{\text{out}} = -V_{\text{MT}} \\
S \text{ open}   & \Rightarrow V_{\text{out}} = +V_{\text{MT}}
\end{cases}
\]

\subsection{Characteristics of FETs}
We examine the general characteristics of an N-channel FET. The drain current versus drain-source voltage (\( I_D \) vs. \( V_{DS} \)) curves vary with different gate-source voltages (\( V_{GS} \)). Key regions of operation include:
\begin{itemize}
    \item \textbf{Linear (Ohmic) Region:} \( V_{DS} \) is low, and \( I_D \) increases linearly with \( V_{DS} \).
    \item \textbf{Pre-Pinch-Off Region:} Occurs between \( V_{GS} = 0 \) and \( V_{GS} = V_P \), where the current begins to saturate.
    \item \textbf{Pinch-Off Region:} At higher \( V_{DS} \), the channel pinches off, and the current levels off before breakdown.
\end{itemize}
Operating the FET in the pinch-off region can lead to breakdown, which is undesirable for stable amplifier operation.

\subsection{Error Analysis in Switch Configurations}
When using FETs as switches in chopper amplifiers, errors can arise from both the on and off states due to the non-ideal characteristics of the FETs.

\subsubsection{On-State Error}
In the on-state, the FET has a finite \( R_{DS(on)} \), typically around 25 ohms. Ideally, the closed switch should present zero resistance, making the output voltage \( V_L = 0 \). However, due to the finite \( R_{DS(on)} \), an error is introduced:
\[
\text{Error}_{\text{on}} = V_L - V_{\text{ideal}} = \frac{R_{DS(on)} \cdot V_G}{R_1 + R_G + R_{DS(on)}}
\]
To minimize this error, large values of \( R_G \) and \( R_1 \) are preferred. However, this requirement contradicts the need to minimize the off-state error, necessitating a compromise in practical designs.

\subsubsection{Off-State Error}
In the off-state, the FET has a high \( R_{DS(off)} \), around \( 10^{10} \) ohms. Ideally, the open switch should allow the output voltage to equal the input voltage \( V_L = V_G \). However, due to the series resistance \( R_G \) and \( R_1 \), not all the input voltage appears at the output:
\[
\text{Error}_{\text{off}} = V_L - V_{\text{ideal}} = -\frac{V_G}{1 + \frac{R_L}{R_G + R_1}}
\]
Minimizing this error involves maximizing \( R_L \) and minimizing \( R_1 \), which again conflicts with the on-state error requirements.

\subsection{Non-Quantifiable Errors}
Apart from the quantifiable errors in the on and off states, there are non-quantifiable errors caused by switching spikes. These spikes occur due to the dynamic nature of capacitances \( C_{GS} \) and \( C_{GD} \) in the FETs. Switching spikes can introduce offset voltages, which add to the output signal, resulting in errors that are difficult to model mathematically. To mitigate these spikes:
\begin{enumerate}
    \item \textbf{Reduce the Rate of Change of Control Voltage:} Instead of using rectangular waveforms, ramp waveforms can slow the transition rates, decreasing the amplitude of switching spikes.
    \item \textbf{Decrease Load and Generator Resistance:} Lowering these resistances reduces the voltage fed through the capacitive reactance, thereby minimizing spike amplitudapplication of gilbert
    
\end{enumerate}

\subsection{Modulation and Demodulation}
The positive-negative amplifier configurations can be used as modulators and demodulators. Pulse Amplitude Modulation (PAM) is achieved by multiplying the input signal with the switching signal. The original input can be recovered by passing the modulated signal through another similar amplifier configured as a demodulator. This process is illustrated in the graphical solution section, where the synchronization of the switching signals ensures accurate modulation and demodulation.

\subsection{Drain-Source Characteristics}
An experimental setup was discussed to obtain the drain-source voltage (\( V_{DS} \)) versus drain current (\( I_D \)) characteristics of FETs. Key observations include:
\begin{itemize}
    \item In the \textbf{Pinch-Off Region}, the drain current remains constant despite increases in \( V_{DS} \).
    \item In the \textbf{Linear Region}, \( I_D \) increases linearly with \( V_{DS} \).
\end{itemize}
For chopper applications, operating the FET near the origin ensures low \( V_{DS} \) and minimal current, preventing breakdown and maintaining linear operation.

\subsection{Conclusion}
The use of FETs as switches in chopper amplifiers introduces both quantifiable and non-quantifiable errors. Balancing the requirements for minimal on-state and off-state errors is crucial for optimal amplifier performance. Additionally, addressing switching spikes through waveform shaping and resistance adjustments is essential for reducing offset voltages and ensuring signal integrity.

\subsection{Assignments}
\begin{enumerate}
    \item \textbf{Graphical Solution:} 
        \begin{itemize}
            \item Implement the modulation and demodulation process using the positive-negative amplifier configuration
            \item Verify the recovery of the original input signal
        \end{itemize}
    
    \item \textbf{Error Minimization:}
        \begin{itemize} 
            \item Analyze practical values for \( R_G \), \( R_1 \), and \( R_L \)
            \item Determine optimal balance between on-state and off-state errors
        \end{itemize}
        
    \item \textbf{Switching Spikes Analysis:}
        \begin{itemize}
            \item Use an oscilloscope to observe switching spikes in a prototype circuit
            \item Propose methods to further minimize these spikes
        \end{itemize}
\end{enumerate}
\subsection{References}
\begin{itemize}
    \item Manufacturer data sheets, including those from Hoffi50 and Texas 
    Instruments, were utilized to obtain precise \(R_{\text{DS(on)}}\) and 
    \(R_{\text{DS(off)}}\) values, which are critical for the error analysis 
    in the design of power electronics circuits.
    
    \item Fundamental concepts from introductory circuit analysis textbooks 
    and first-year differential calculus were applied to determine optimal 
    resistance
\end{itemize}

\subsection{Next Class}
In the upcoming class, we will delve deeper into the operational dynamics of FET switches, explore additional error sources, and review student assignments. Please ensure you have completed the assigned tasks and are prepared to discuss your findings.
\begin{center}\rule{0.5\linewidth}{0.5pt}\end{center}

\section*{Lecture 6: 03/10/2024}

In the previous class, we studied the FET as a switch and investigated how using the FET to produce Pulse Amplitude Modulation (PAM) generates errors. We began by examining the FET in the SH configuration, deriving the relevant equations. We identified that, apart from the non-ideal behavior of the switch—which introduces \textit{off} and \textit{on} errors—additional errors arise due to the internal gate capacitances.

\subsection{Equivalent Circuit of the FET in Configuration}

Consider a slowly varying voltage \( V_G \) and a switching voltage \( V_C \). Due to the internal capacitances \( C_{GD} \), \( C_{GS} \), and \( C_{DS} \), switching spikes are generated at the output of the system. This phenomenon occurs because a pulse drive causes the voltage to rise suddenly, introducing high-frequency components from the rectangular waveform. The same effect applies when the voltage falls rapidly.
\begin{figure}[h]
    \centering
    \includegraphics[width=0.4\textwidth]{Square4.png}
    \caption{}
\end{figure}
\subsection{Timing Diagram and Control Voltage}

Refer to the timing diagram where the control voltage \( V_C(t) \) transitions between approximately 0 volts and -5 volts, while \( V_L \) represents the load voltage. During the intervals when the control voltage rises from -5 volts to 0 volts, part of the signal flows to the output, and part flows to the ground, depending on the values of \( R_1 \) and the generator resistance \( R_G \). This results in pulses at the output, known as feed-through spikes. The magnitude of these spikes depends on the internal capacitances \( C_{GD} \), \( C_{GS} \), \( C_{DS} \), as well as the resistances \( R_1 \), \( R_G \), and \( R_L \).

\subsection{Reducing Feed-Through Spikes}

To mitigate feed-through spikes caused by internal capacitances, we can implement the following strategies:

\begin{enumerate}
    \item \textbf{Reduce the Rate of Change of the Control Voltage:} Instead of using a sharp, rising rectangular waveform, employ a more gradually varying waveform. This approach minimizes high-frequency components, thereby reducing the influence of \( C_{GD} \) and decreasing the amplitude of feed-through spikes.
    
    \item \textbf{Decrease Load and Generator Resistance:} Lowering \( R_L \) and \( R_G \) can help reduce the magnitude of the spikes. 
\end{enumerate}

In the last class, we demonstrated how to resolve switching waveforms into their harmonics. By selecting a specific harmonic and applying linear equations, we can determine the amplitude of that frequency component. For instance, considering the drain-to-source path with \( R_{DS} \) varying between \( 10^{10} \) ohms and \( R_L \), we can simplify the equivalent circuit to:

\[
V_{\text{out}} = V_C(t) \times \frac{R_L \parallel X_{DS} \parallel (R_1 + R_G)}{X_{GD} + R}
\]

where \( R = R_1 + R_G \parallel R_L \parallel X_{DS} \).

\subsection{Impact of Transducer Resistance}

If the transducer has a high internal resistance \( R_G \), the output of the feed-through signal will be minimized. For example, if \( R_L = 0 \), the output will theoretically be zero, though \( R_L \) should not be zero in practical applications as it serves as the input resistance to the AC amplifier.

\subsection{Drive Voltage Optimization}

To minimize switching spikes, consider the following:

\begin{itemize}
    \item \textbf{Reduce the Drive Frequency:} Lowering the drive frequency decreases the reactance of the capacitances, reducing the amplitude of the spikes.
    
    \item \textbf{Use Minimal Control Voltage:} Utilize the smallest control voltage necessary to turn the device on and off, which diminishes the magnitude of \( V_C(t) \) and thereby reduces spike amplitudes.
    
    \item \textbf{Waveform Shaping:} Employ ramp or sinusoidal waveforms instead of pure rectangular pulses to limit high-frequency components.
\end{itemize}

\subsection{Experimental Validation}

These theoretical results are typically validated experimentally using a sensitive oscilloscope. Manufacturers' datasheets provide the minimum voltage required to switch the device on and off. In the lab, you can adjust the signal generator's input and monitor the output spikes, reducing the voltage until the spikes become acceptable.

\subsection{Series Switch Configuration}

Transitioning from the SH switch, we now analyze the series switch configuration. In this setup, the FET is connected in series with the load and generator. The practical circuit for driving the FET using a small-signal transistor is straightforward. The gate drive switches the gate from ground to a negative potential, greater than the pinch-off voltage specified in the FET's datasheet (typically between -3V and -5V for an inter-channel FET).

\subsection{Off-State Error Analysis}

When the switch is in the off state, the error is defined as:

\[
\text{Error}_{\text{off}} = V_L - 0 = V_L
\]

Here, \( V_L \) is the voltage drop across \( R_L \), calculated as:

\[
V_L = V_G \times \frac{R_L}{R_G + R_{DS,\text{off}} + R_L}
\]

Given that \( R_{DS,\text{off}} \) is typically \( 10^{10} \) ohms, which is much greater than \( R_L \) (typically ranging from 10K to 100K ohms), we can approximate:

\[
\text{Error}_{\text{off}} \approx V_G \times \frac{R_L}{R_G + R_{DS,\text{off}}} \approx \frac{R_L}{R_G + R_{DS,\text{off}}} \times V_G
\]

For high-impedance transducers, such as condenser microphones with capacitive transducers, \( R_G \) is very large, making the error negligible at low frequencies.

\subsection{On-State Error Analysis}

When the switch is in the on state, the error is defined as:

\[
\text{Error}_{\text{on}} = V_L - V_G
\]

Calculating \( V_L \):

\[
V_L = V_G \times \frac{R_L}{R_G + R_{DS,\text{on}} + R_L}
\]

Given that \( R_{DS,\text{on}} \) is low (typically 25-30 ohms), and \( R_L \) is large, we approximate:

\[
\text{Error}_{\text{on}} \approx V_G \times \frac{R_L}{R_G + R_{DS,\text{on}} + R_L} \approx V_G \times \frac{R_L}{R_G + R_L}
\]

To minimize on-state error, \( R_L \) should be large relative to \( R_G \). However, this requirement conflicts with the need to minimize off-state error, where \( R_L \) should be small. Thus, a compromise must be reached to optimize the total error, potentially by minimizing the partial differential of the total error with respect to \( R_L \).

\subsection{Feed-Through Spikes in Series Choppers}

In series chopper configurations, feed-through spikes are influenced by the control signal and the switching action of the FET. The control voltage typically ranges between -5V and 0V. When the device switches on and off, the current paths through \( C_{GS} \), \( C_{GD} \), and \( R_G \) determine the magnitude of the spikes.

Key points include:

\begin{itemize}
    \item During switching, the predominant current flows through \( C_{GS} \), generating positive and negative spikes at the output.
    
    \item The time constant of the spikes is mainly governed by \( C_{GS} \) and \( R_L \).
    
    \item Reducing \( R_L \) or \( C_{GS} \) can help minimize spike amplitude.
\end{itemize}

\subsection{Minimizing Switching Spikes}

To effectively minimize switching spikes:

\begin{enumerate}
    \item \textbf{Decrease the Drive Frequency:} Lowering the frequency reduces the capacitive reactance, thereby decreasing the spike voltage.
    
    \item \textbf{Lower \( R_L \):} Reducing the load resistance can help minimize spike amplitudes, though it must be balanced against other design requirements.
    
    \item \textbf{Optimize \( R_{DS,\text{on}} \):} Ensuring \( R_{DS,\text{on}} \) is as low as possible minimizes on-state errors.
    
    \item \textbf{Use Minimal Control Voltage:} Applying just enough voltage to switch the device on and off reduces the magnitude of the feed-through voltage.
    
    \item \textbf{Waveform Shaping:} Implementing ramp or sinusoidal waveforms instead of sharp pulses can lower high-frequency components and reduce spike amplitudes.
\end{enumerate}

\subsection{Conclusion}

To summarize, minimizing switching spikes involves a combination of reducing the drive frequency, optimizing load and generator resistances, utilizing minimal control voltages, and shaping the control waveform. Balancing these factors is essential to achieve optimal performance with minimal errors in both the on and off states.

\subsection{Questions and Further Discussion}

If there are any questions regarding this material, feel free to ask. Otherwise, we will conclude today's lecture here.
\begin{center}\rule{0.5\linewidth}{0.5pt}\end{center}

\section*{Lecture 7: 7/10/2024}

\subsection{Introduction}
In the last class, we analyzed the performance of various chopper switches. We began with the \textbf{Shunt Chopper I}, where the switch is connected in parallel with the supply, the generator (or signal source), and the load. Subsequently, we examined the \textbf{Series Chopper}, where the switch is placed in series with the signal to be chopped. Our analysis included both the \textit{on-state} and the \textit{off-state} errors that the switch generates.

\subsection{Off-State Error Analysis}
Ideally, when the switch is off, no signal should pass from the generator to the load. However, due to the non-ideal nature of the switch, specifically when the switch is off, the resistance (\( R_{DS_{\text{off}}} \)) is not infinite as required. Typically, \( R_{DS_{\text{off}}} \) is around \(10^{10}\) ohms. This non-ideality causes a small voltage to appear at the output, referred to as the \textbf{error voltage} (\( E_{\text{off}} \)).

The error voltage can be expressed as:
\begin{equation}
E_{\text{off}} = V_G \frac{R_L}{R_G + R_{DS_{\text{off}}}}
\end{equation}

Since \( R_{DS_{\text{off}}} \) is significantly larger than \( R_L \), the equation simplifies to:
\begin{equation}
E_{\text{off}} \approx \frac{V_G R_L}{R_G + R_{DS_{\text{off}}}} \approx \frac{V_G R_L}{R_G}
\end{equation}

From this, we conclude that \( E_{\text{off}} \) becomes predominant only when \( R_L \) is large.

\subsection{On-State Error Analysis}
When the switch is on, ideally, the entire generator voltage (\( V_G \)) should appear across the load (\( R_L \)). However, due to the non-ideal switch, some voltage is lost within the circuit. The output voltage (\( V_{\text{out}} \)) is thus slightly less than \( V_G \), resulting in an \textbf{on-state error} (\( E_{\text{on}} \)) defined as:
\begin{equation}
E_{\text{on}} = V_L - V_G
\end{equation}
where
\begin{equation}
V_L = V_G \frac{R_L}{R_G + R_{DS_{\text{on}}} + R_L}
\end{equation}

Rearranging the equation for \( E_{\text{on}} \):
\begin{equation}
E_{\text{on}} = V_G \left( \frac{R_L}{R_G + R_{DS_{\text{on}}} + R_L} - 1 \right )
\end{equation}

Simplifying further:
\begin{equation}
E_{\text{on}} = V_G \frac{1 - \frac{R_L}{R_G + R_{DS_{\text{on}}} + R_L}}{R_G + R_{DS_{\text{on}}} + R_L} = -V_G \frac{R_G + R_{DS_{\text{on}}}}{R_G + R_{DS_{\text{on}}} + R_L}
\end{equation}

\subsection{Minimizing Errors}
To minimize both \( E_{\text{on}} \) and \( E_{\text{off}} \), we derive the following requirements:
\begin{itemize}
    \item \( R_{DS_{\text{on}}} \) should be as small as possible.
    \item \( R_L \) should be large to minimize \( E_{\text{off}} \), but small to minimize \( E_{\text{on}} \).
\end{itemize}

This presents a contradiction, as increasing \( R_L \) reduces \( E_{\text{off}} \) but increases \( E_{\text{on}} \). To resolve this, calculus can be employed to find optimal values for \( R_L \) and \( R_{DS_{\text{on}}} \) that minimize the overall error.

\subsection{Error Due to Switching Spikes}
Switching spikes introduce additional errors in the series chopper. These errors originate from internal capacitive effects due to the reverse-biased junctions of the n-type and p-type materials within the switch. These junction capacitances vary with the applied voltage, leading to transient spikes during switching.

\subsubsection{Control Voltage and Capacitances}
The control voltage (\( V_C(t) \)) is typically a rectangular waveform varying from \(-V_P\) to approximately zero volts. To analyze the effect of switching spikes, we consider only the fundamental frequency component of the square wave control voltage, neglecting higher harmonics due to their diminishing amplitudes.

Each switching switch has different capacitances:
\begin{itemize}
    \item \( C_{DS} \): Between drain and source.
    \item \( C_{GS} \): Between gate and source.
    \item \( C_{DG} \): Between drain and gate.
\end{itemize}

Assuming \( V_G = 0 \) and focusing on the fundamental component, the equivalent circuit can be analyzed using linear circuit analysis methods.
\subsection{Illustrations}


\begin{figure}[H]
    \centering
    \includegraphics[width=0.7\columnwidth]{Series1.png}
\end{figure}

\begin{figure}[H]
    \centering
    \includegraphics[width=0.7\columnwidth]{Series2.png}
\end{figure}

\begin{figure}[H]
    \centering
    \includegraphics[width=0.7\columnwidth]{Series3.png}
\end{figure}

\begin{figure}[H]
    \centering
    \includegraphics[width=0.5\columnwidth]{Series4.png}
\end{figure}

\begin{figure}[H]
    \centering
    \includegraphics[width=0.7\columnwidth]{Series5.png}
\end{figure}

\begin{figure}[H]
    \centering
    \includegraphics[width=0.5\columnwidth]{Series6.png}
\end{figure}

\begin{figure}[H]
    \centering
    \includegraphics[width=0.7\columnwidth]{Series7.png}
\end{figure}

\begin{figure}[H]
    \centering
    \includegraphics[width=0.7\columnwidth]{Series8.png}
\end{figure}

\begin{figure}[H]
    \centering
    \includegraphics[width=0.7\columnwidth]{Series9.png}
\end{figure}

\begin{figure}[H]
    \centering
    \includegraphics[width=0.7\columnwidth]{Series10.png}
\end{figure}


\subsubsection{Superposition Theorem Application}
Using the superposition theorem, the output voltage \( V_{\text{out}}(t) \) can be expressed as the sum of responses due to each independent source:
\begin{equation}
V_{\text{out}}(t) = V_{\text{out}_1}(t) + V_{\text{out}_2}(t)
\end{equation}

Where:
\begin{align}
V_{\text{out}_1}(t) &= V_{C1}(t) \cdot \frac{R_L \parallel X_{GS}}{X_{DS} + (R_G \parallel X_{DG}) + (R_L \parallel X_{GS})} \\
V_{\text{out}_2}(t) &= V_{C2}(t) \cdot \frac{R_L \parallel X_{GS}}{X_{DS} + (R_G \parallel X_{DG}) + (R_L \parallel X_{GS})}
\end{align}


\subsection{Practical Considerations and Assignments}
To minimize the effects of internal capacitances:
\begin{itemize}
    \item Decrease the frequency of the drive signal.
    \item Decrease the value of \( R_L \).
    \item Decrease the gate drive voltage, ensuring it does not exceed the pinch-off voltage.
    \item Use a ramp or alternative voltage waveform instead of sharp rising square waves to reduce harmonic content.
\end{itemize}

\subsubsection{Assignment}
Please work on the following examples by substituting the derived equations:
\begin{enumerate}
    \item Calculate the on-state and off-state errors when \( R_G = 0 \), \( R_L = 2\text{k}\Omega \), \( R_{DS_{\text{off}}} = 10^{10}\Omega \), and \( R_{DS_{\text{on}}} = 25\Omega \).
    \item Plot a table showing how the on-state and off-state errors vary with \( R_L \) values of 2kΩ, 10kΩ, 50kΩ, and 100kΩ.
\end{enumerate}

Compare the performance of shunt and series chopper configurations under varying load resistances and generator resistances. Analyze which configuration provides minimal error under different operating conditions.

\subsection{Conclusion}
In this lecture, we derived the on-state and off-state errors for series chopper configurations and discussed methods to minimize these errors. Understanding the interplay between \( R_L \), \( R_G \), and \( R_{DS} \) is crucial for designing efficient chopper circuits with minimal signal distortion.

\subsection{Additional Resources}
Refer to the Google Classroom for circuit diagrams, additional data, and assignments. Ensure to review the provided documents in Word format to avoid equation distortions present in PDFs.
\begin{center}\rule{0.5\linewidth}{0.5pt}\end{center}

\section*{Lecture 8: 14/10/2024}

\subsection{Introduction}
In the previous table, we included rows for shunt Chopper and Series Chopper, detailing their on and off errors under the condition where the generator resistance (\( R_G \)) is zero for various load resistances (\( R_L \)). Typically, \( R_L \) serves as the input to the AC amplifier. We computed the on and off errors for both Series and shunt Choppers across different load resistances and confirmed their performance metrics. This analysis helps determine the most suitable switching configuration based on the error rates.

\subsection{Performance Comparison of Series and Shunt Choppers}
Our findings indicate that, in general, the Series Chopper outperforms the shunt Chopper in terms of minimizing errors. 

\subsection{Introduction to Series-Shunt Chopper Configuration}
We now introduce another switching configuration known as the Series-Shunt Chopper. In this setup, transistors are connected in a specific manner:
\begin{itemize}
    \item The generator is connected with a resistance \( R_G \).
    \item The first transistor \( F_1 \) is in series with the load.
    \item The second transistor \( F_2 \) is in shunt configuration with the load.
\end{itemize}
This configuration is referred to as the Series-Shunt Chopper. We will analyze the switching errors—both on and off—and incorporate them into the original table that previously included only Series and shunt configurations. This consolidation allows for a comprehensive comparison using a consistent set of parameters, primarily the signal level \( V_G \), to determine the configuration that yields the least error.

\subsection{Error Analysis of Series-Shunt Chopper}
The Series-shunt Chopper employs two transistors: one in series and one in shunt configuration. We will analyze the errors when:
\begin{enumerate}
    \item The series transistor is on, and the shunt transistor is off (Case 1).
    \item The series transistor is off, and the shunt transistor is on (Case 2).
\end{enumerate}

\subsubsection{Case 1: Series On, shunt Off}
When the series transistor is on and the shunt transistor is off, the expected behavior is that the switch transmits all input signal (\( V_G \)) to the output. However, due to non-ideal switching devices, some error is introduced.

\paragraph{Error Calculation}
Let:
\begin{align*}
V_{\text{indicated}} &= V_L \quad \text{(Voltage across load)} \\
V_{\text{true}} &= V_G \quad \text{(Generator voltage)} \\
R_{\text{equiv}} &= R_{DS\_off} \parallel R_L
\end{align*}
The error is given by:
\[
\text{Error} = V_{\text{indicated}} - V_{\text{true}} = V_L - V_G
\]
Since \( R_{\text{equiv}} = \frac{R_{DS\_off} \cdot R_L}{R_{DS\_off} + R_L} \), the voltage across the load is:
\[
V_L = V_G \times \frac{R_{\text{equiv}}}{R_G + R_{DS\_on} + R_{\text{equiv}}}
\]
Simplifying:
\[
V_L = V_G \times \frac{1 - R_{\text{equiv}}}{R_G + R_{DS\_on} + R_{\text{equiv}}}
\]
Upon further simplification, assuming \( R_{\text{equiv}} \approx R_L \) (since \( R_{DS\_off} \) is large and \( R_L \) is typically between 10kΩ and 100kΩ), the error expression becomes:
\[
\text{Error} = V_G \left(1 - \frac{R_L}{R_G + R_{DS\_on} + R_L}\right)
\]
To minimize this error:
\begin{itemize}
    \item \( R_G \) should be minimized.
    \item \( R_{DS\_on} \) should be minimized.
    \item \( R_L \) should be maximized.
\end{itemize}

\subsubsection{Case 2: Series Off, shunt On}
In this scenario, the series transistor is off, and the shunt transistor is on. Ideally, the output voltage should be zero. However, due to non-ideal switching, a small voltage \( V_L \) appears at the output, resulting in an error.

\paragraph{Error Calculation}
\[
\text{Error} = V_L - 0 = V_L
\]
Since:
\[
V_L = V_G \times \frac{R_{\text{equiv}}}{R_G + R_{DS\_off} + R_{\text{equiv}}}
\]
And \( R_{\text{equiv}} \approx R_{DS\_on} \) (given \( R_L \) is large),
\[
\text{Error} = V_G \times \frac{R_{DS\_on}}{R_G + R_{DS\_off} + R_{DS\_on}}
\]
To minimize this error:
\begin{itemize}
    \item \( R_{DS\_off} \) should be maximized.
    \item \( R_{\text{equiv}} \) (i.e., \( R_{DS\_on} \)) should be minimized.
\end{itemize}

\subsection{Optimization Conditions}
For both cases to have minimal errors simultaneously:
\begin{itemize}
    \item \( R_{DS\_on} \) should be as small as possible.
    \item \( R_L \) should be as large as possible.
    \item \( R_{DS\_off} \) should be as large as possible.
\end{itemize}
These conditions ensure that the numerator in the error expressions remains small while the denominator remains large, thereby minimizing the overall error.

\subsection{Error Computation for Different Configurations}
We are tasked with computing the errors for three configurations under varying conditions:
\begin{enumerate}
    \item \( R_G = 1\,k\Omega \), \( R_L = 1\,k\Omega \), \( R_{DS\_off} = 10\,\Omega \), \( R_{DS\_on} = 25\,\Omega \).
    \item \( R_G = 100\,k\Omega \), \( R_L = 1\,k\Omega \), \( R_{DS\_off} = 10\,\Omega \), \( R_{DS\_on} = 25\,\Omega \).
    \item \( R_G = 100\,k\Omega \), \( R_L = 100\,k\Omega \), \( R_{DS\_off} = 10\,\Omega \), \( R_{DS\_on} = 25\,\Omega \).
\end{enumerate}
%A table should be created to summarize the on and off errors for each configuration:
%\begin{table}[h!]
  %  \centering
   % \begin{tabular}{|c|c|c|c|}
    %    \hline
     %   \textbf{Configuration} & \( R_G \) & \( R_L \) & \textbf{Errors} \\
      %  \hline
       % Shunt Chopper & 1\,k\Omega & 1\,k\Omega & On: ... \ Off: ... \\
        %\hline
        %Series Chopper & 1\,k\Omega & 1\,k\Omega & On: ... \ Off: ... \\
        %\hline
      %  %Series-Shunt Chopper & 1\,k\Omega & 1\,k\Omega & On: ... \ Off: ... \\
       % \hline
%        %Shunt Chopper & 100\,k\Omega & 1\,k\Omega & On: ... \ Off: ... \\
 %       \hline
   %     Series Chopper & 100\,k\Omega & 1\,k\Omega & On: ... \ Off: ... \\
  %      \hline
     %   Series-Shunt Chopper & 100\,k\Omega & 1\,k\Omega & On: ... \ Off: ... \\
    %    \hline
      %  Series-Shunt Chopper & 100\,k\Omega & 100\,k\Omega & On: ... \ Off: ... \\
       % \hline
   % \end{tabular}
    %\caption{Error Analysis for Different Chopper Configurations}
%\end{table}

\noindent By substituting the given values into the derived equations, we can determine which configuration offers the best performance in each scenario.

\subsection{Error Due to Switching Spikes}
Apart from the on and off errors, switching spikes introduce additional errors not covered in the previous analysis. These spikes result from the internal capacitive elements of the switching devices.

\subsubsection{Series-shunt Configuration and Switching Spikes}
In the Series-shunt configuration, the switching spikes generated by the series and shunt devices tend to cancel each other out due to their opposite polarities. This results in superior spike performance compared to other configurations. The minimal spike errors contribute to the overall reduced error in this configuration.

\subsection{Practical Circuit Considerations}
In laboratory settings, transistors such as the \texttt{2N4856A} are utilized for Chopper amplifiers due to their low \( R_{DS\_on} \) of 25Ω and minimal internal capacitances. These transistors are designed for applications including analog switches, sample-and-hold circuits, and chopper amplifiers. The common-source capacitance typically averages around 10 pF, which ensures that switching spikes are negligible.

\subsubsection{Dynamic Capacitance}
The capacitance in these devices is dynamic and varies with the gate-source voltage (\( V_{GS} \)). For instance, when \( V_{DS} = 0 \),
\[
C_{\text{gate-source}} \approx 9\,\text{pF at } V_{GS} = 0\,V \quad \text{and} \quad 5\,\text{pF at } V_{GS} = -4\,V
\]
This dynamic behavior justifies the assumption of stationary capacitances in our theoretical analysis.

\subsection{Transistor Switching Circuit Design}
\begin{figure}[h]
    \centering
    \includegraphics[width=1.09\linewidth]{Square12.png}
    \caption{}
\end{figure}
The Series-shunt Chopper employs two transistors, \( Q_1 \) and \( Q_2 \), configured to provide two voltages that are \(180^\circ\) % Inline math
 out of phase(Antiphase circuit). This ensures that when one transistor is on, the other is off, and vice versa.

\subsubsection{Base Current Calculation}
To drive \( Q_1 \) into saturation, the base current (\( I_B \)) must satisfy:
\[
I_B = \frac{V_C - 0.7}{1.2\,k\Omega}
\]
Where \( V_C \) is the control voltage. The collector current (\( I_C \)) is then:
\[
I_C = \beta_1 I_B \geq I_{C_{\text{max}}}
\]
Ensuring:
\[
\beta_1 I_B \geq I_{C_{\text{max}}}
\]
This condition guarantees that \( Q_1 \) operates in saturation, enabling reliable switching between on and off states.

\subsubsection{Switching Operation}
When \( Q_1 \) is on, the voltage at the collector of \( Q_1 \) drops to approximately \( 0.2 \, \text{V} \), turning \( Q_2 \) off (since biasing \( Q_2 \) requires at least \( 0.7 \, \text{V} \)). Conversely, when \( Q_2 \) is on, the voltage at its collector drops to \( 0 \, \text{V} \), turning \( Q_1 \) off. This alternating behavior ensures precise switching control.


\paragraph{Voltage Levels}
- When \( Q_2 \) is off:
    \[
    V_{G1} \approx 0\,V \quad \Rightarrow \quad \text{2N4856A($FET_1$) was on}
    \]
    \[
    V_{G2} = \frac{22k}{100k + 22k} \times -10\,V \approx -1.8\,V \quad \Rightarrow \quad \text{$FET_2$ is off}
    \]
- When \( Q_2 \) is on:
    \[
    V_{G1} = \frac{25\,\Omega}{100k + 25\,\Omega} \times -10\,V \approx 0\,V \quad \Rightarrow \quad Q_1 \text{ is off}
    \]
    \[
    V_{G2} = \frac{22k}{100k + 22k} \times -10\,V \approx -1.8\,V \quad \Rightarrow \quad Q_2 \text{ is on}
    \]

\subsection{Transistor Selection and Design Considerations}
Selecting the appropriate transistors is crucial for minimizing switching errors. Key considerations include:

\begin{itemize}
    \item \textbf{Low \( R_{DS_{\text{on}}} \):} Reduces on-state error.
    \item \textbf{High \( R_{DS_{\text{off}}} \):} Minimizes off-state error.
    \item \textbf{Low Capacitance:} Reduces switching spikes.
\end{itemize}


For instance, the \texttt{2N4856A} transistor offers \( R_{DS\_on} = 25\,\Omega \) and low gate-source capacitance, making it suitable for high-performance chopper applications.

\subsection{Exam Questions}
\begin{enumerate}
    \item \textbf{Define a Chopper-Stabilized Amplifier:} 
    Describe its method of operation and state the applications to which it is well-suited, providing examples. Explain the principle of operation.

    \item \textbf{Compare Chopper Configurations:} 
    Distinguish between Series, Shunt, and Series-Shunt Choppers. Discuss the criteria used to select a particular configuration for specific operations.

    \item \textbf{Drift Voltage Calculation:} 
    Given an input of \( 20 \, \text{mV DC} \) and a maximum drift voltage of \( 0.001 \cos(2\pi f t) \) at point \( Y \), determine the drift voltage magnitude if the input to the AC amplifier is shorted.
     \item \textbf{Total Error in Chopper-Stabilized System:} 
    Calculate the overall error considering modulators, AC amplifiers, and demodulators. Discuss how to mitigate these errors to improve system performance.

    \begin{tcolorbox}[
        colback=blue!5!white, 
        colframe=blue!75!black, 
        title=Hints, 
        boxrule=0.3mm, % Thin border
        sharp corners, % Compact appearance
        boxsep=1mm, % Minimal padding
        left=1mm, right=1mm, top=1mm, bottom=1mm, % Tight margins
        before skip=2mm, % Space before the box
        after skip=2mm,  % Space after the box
        title style={font=\bfseries, align=left, size=\small}, % Small, bold title
        fontupper=\small % Small content font for compactness
    ]
    1. The gain is approximately \( 10 \).

    2. The chopper circuit consists of three main parts:
    \begin{itemize}[leftmargin=3mm, itemsep=0mm]
        \item \textbf{Modulator:} Configured in a series-shunt configuration.
        \item \textbf{AC Amplifier:} Includes components with \( 1\,\text{nF} \) and \( 1\,\text{pF} \) capacitances.
        \item \textbf{Demodulator:} Configured in a series-shunt configuration.
    \end{itemize}

    3. To calculate the total error:
    \begin{enumerate}[leftmargin=3mm, itemsep=0mm]
        \item Determine the ON error and OFF error for each part.
        \item For each part, take the greater of the ON error and OFF error.
        \item Sum the errors of the three parts to obtain the total error.
    \end{enumerate}
\end{tcolorbox}


   
\end{enumerate}

\subsection{Final Remarks}
Students are encouraged to perform the outlined exercises to reinforce their understanding of chopper configurations and error analysis. Practical implementation in the lab using appropriate transistors and resistor values will provide hands-on experience in minimizing errors and optimizing circuit performance.
\begin{center}\rule{0.5\linewidth}{0.5pt}\end{center}

\section*{Lecture 9: 21/10/2024}

\subsection{Introduction}
Today, we are starting a new topic: log amplifiers. Please ensure you have completed the simple exercises from the previous topic. The principle behind a log amplifier is straightforward. We convert a low-variant DC voltage to a high frequency using pulse modulation. This high-frequency signal is then amplified using an amplifier, followed by demodulation to recover the baseband signal. This process helps eliminate noise due to voltage (\(V_{IO}\)) and current (\(I_{O}\)), which is the principle behind the CH stabilizing amplifier.

\subsection{Definition and Basic Operation}
A log amplifier is an amplifier whose output is described by the following equation:
\begin{equation}
V_{\text{out}} = K \cdot \log(V_{IN})
\label{eq:log_amp}
\end{equation}
where \(K\) is a constant. Alternatively, this can be expressed as:
\begin{equation}
e^{V_{\text{out}}/K} = V_{IN}
\label{eq:log_amp_rearranged}
\end{equation}
This is simply a rearrangement of Equation \ref{eq:log_amp}.

\subsection{Practical Applications}
Log amplifiers have several practical applications:
\begin{enumerate}
    \item \textbf{Analog Computation:} Before the advent of digital computers, analog amplifiers like operational amplifiers were used to perform complex operations such as differentiation, summation, and division. Although digital supercomputers are now prevalent, analog amplifiers remain faster for certain applications because they provide almost instantaneous output when a signal is input, unlike digital computers that require program execution through clock cycles.
    
    \item \textbf{Compressors in Signal Processing:} In recording studios and public address systems, log amplifiers are used in compressors to reduce the dynamic range of signals. This allows the signal to fit within the transmission medium's dynamic range, such as recording tape or communication channels. When the signal needs to be recovered, an exponential amplifier (expander) is used to restore the original dynamic range.
    
    \item \textbf{Driving Display Devices:} Log amplifiers drive display devices with logarithmic indications, such as sound level meters, where calibration is typically in logarithmic scale.
    
    \item \textbf{Log Display Indicators:} They are also used for log display indicators in various applications.
\end{enumerate}

\subsection{Basic Log Amplifier Circuit}
The most basic log amplifier circuit is depicted in Figure \ref{fig:basic_log_amp}. In this circuit, the input voltage \(V_S\) is connected through a resistor \(R\) to the inverting input of an operational amplifier (op-amp). A diode is connected across the op-amp, with the voltage across the diode denoted as \(V_F\). The output voltage is \(V_{\text{out}}\).

\begin{figure}[H]
    \centering
    \includegraphics[width=0.25\textwidth]{BasicLog.png}
    \caption{Basic diode Log Amp}
    \label{fig:question_image}
\end{figure}

\subsubsection{Diode Equation}
The diode equation is given by:
\begin{equation}
I_F = I_S \cdot e^{V_F/(nV_T)}
\label{eq:diode}
\end{equation}
where:
\begin{itemize}
    \item \(I_F\) is the forward current through the diode.
    \item \(I_S\) is the saturation current.
    \item \(V_F\) is the forward voltage across the diode.
    \item \(n\) is the ideality factor (\(n=1\) for germanium and \(n=2\) for silicon).
    \item \(V_T\) is the thermal voltage.
\end{itemize}
When the diode starts to conduct, \(V_F/(nV_T) \gg 1\), allowing us to approximate Equation \ref{eq:diode} as:
\begin{equation}
I_F \approx I_S \cdot e^{V_F/(nV_T)}
\label{eq:diode_approx}
\end{equation}

\subsubsection{Current Through the Circuit}
Assuming an ideal op-amp, the current \(I_F\) flows through the diode and does not enter the op-amp. Therefore, we can write:
\begin{equation}
I_F = \frac{V_S}{R}
\label{eq:current}
\end{equation}

\subsubsection{Relating \(V_F\) to \(V_{\text{out}}\)}
In the circuit, the op-amp maintains a virtual earth at the inverting input. This implies that:
\begin{equation}
V_F = -V_{\text{out}}
\label{eq:vf_vout}
\end{equation}
Substituting Equation \ref{eq:vf_vout} into Equation \ref{eq:current}, we get:
\begin{equation}
\frac{V_S}{R} = I_S \cdot e^{-V_{\text{out}}/(nV_T)}
\label{eq:final}
\end{equation}
Taking the natural logarithm of both sides:
\begin{equation}
\ln\left(\frac{V_S}{I_S R}\right) = -\frac{V_{\text{out}}}{nV_T}
\label{eq:log_relation}
\end{equation}
Rearranging for \(V_{\text{out}}\):
\begin{equation}
V_{\text{out}} = -nV_T \ln\left(\frac{V_S}{I_S R}\right)
\label{eq:vout_final}
\end{equation}
This can be expressed in the form of Equation \ref{eq:log_amp} by defining the constants:
\begin{equation}
K_1 = -nV_T \quad \text{and} \quad K_2 = \frac{1}{I_S R}
\end{equation}
Thus,
\begin{equation}
V_{\text{out}} = K_1 \ln(K_2 V_S)
\label{eq:general_log_amp}
\end{equation}

\subsection{Improving Accuracy}
The accuracy of the basic log amplifier depends on the relationship between \(V_F\) and \(I_F\). To achieve more accurate computations, \(I_F\) must be minimized. This leads to the development of the basic transistor log amplifier, which uses the base-emitter junction of a transistor to ensure a smaller current passes through the diode.

\subsection{Basic Transistor Log Amplifier}
The basic transistor log amplifier circuit is shown in Figure \ref{fig:transistor_log_amp}. In this configuration, the diode is replaced with the base-emitter junction of a transistor, reducing the current through the diode and improving accuracy.

\begin{figure}[h]
    \centering
    \includegraphics[width=0.25\textwidth]{Shegone.png}
    \caption{Basic Transistor Log Amplifier Circuit}
    \label{fig:transistor_log_amp}
\end{figure}

Using the transistor's properties, the output voltage \(V_{\text{out}}\) can be expressed as:
\begin{equation}
V_{\text{out}} = -nV_T \ln\left(\frac{V_{S1}}{V_{S2}}\right)
\label{eq:transistor_log_amp}
\end{equation}
where \(V_{S1}\) and \(V_{S2}\) are the input voltages to the two transistors in the matched pair configuration.

\subsection{Practical Log Amplifier using Matched Transistors}
To eliminate the dependency on the reverse saturation current \(I_S\) and improve temperature stability, a matched transistor log amplifier is used. This configuration uses two matched transistors to cancel out the effects of temperature variations on \(I_S\).

\begin{figure}[h]
    \centering
    \includegraphics[width=\linewidth]{MatchedAmp.png}
    \caption{Matched Transistor Log Amplifier Circuit}
    \label{fig:matched_transistor_log_amp}
\end{figure}

In this circuit, transistors \(Q_1\) and \(Q_2\) are fabricated on the same silicon substrate to ensure they have identical characteristics. The output voltage is given by:
\begin{equation}
V_{\text{out}} = a \ln\left(\frac{V_{S1}}{V_{S2}}\right)
\label{eq:matched_log_amp}
\end{equation}
where \(a\) is a constant determined by the circuit parameters.

\subsection{Practical Considerations}
In practical implementations, offset voltages can introduce errors. To address this, null offsetting circuits are incorporated using potentiometers \(P_1\) and \(P_2\). These adjustments ensure that the output voltage is zero when there is no input signal.

\subsection{Conclusion}
Log amplifiers are essential components in analog computation, signal processing, and various display applications. By utilizing transistor-based configurations and matched transistor pairs, the accuracy and stability of log amplifiers can be significantly enhanced, making them suitable for practical laboratory implementations.
\begin{center}\rule{0.5\linewidth}{0.5pt}\end{center}

\section*{Lecture 10: 24/10/2024}

Welcome back to Lecture 10. It looks like our class size has slightly increased from 39 to 43 students. I hope everyone can see the presentation on the screen. If you're having any issues, please let me know.

\subsection{Review of Log Amplifiers}

In our last class, we discussed the log amplifier, a specialized type of amplifier whose output is the logarithm of its input voltage. We explored various applications of log amplifiers in analog systems.

\subsection{Applications of Log Amplifiers}

Log amplifiers are commonly used in:

\begin{itemize}
    \item \textbf{Recording Studios:} For signal compression, ensuring that signals fit within the dynamic range of a channel. This compression allows signals to be later expanded.
    \item \textbf{Dynamic Range Management:} Matching the input signals to the dynamic range limitations of recording media.
\end{itemize}

\subsection{Historical Context of Audio Recording}

\subsubsection{Transition from Cassette to CD}

Most of you may not have been born during the era when cassette tapes were prevalent. Before CDs became mainstream, music was primarily recorded on vinyl plates or cassette tapes. CDs offered several advantages:

\begin{itemize}
    \item \textbf{Cost-Effective and Portable:} CDs were cheaper and easier to transport due to their compact size.
    \item \textbf{Higher Signal Quality:} CDs provided a higher signal-to-noise ratio (approximately 70 dB) compared to tapes.
\end{itemize}

\subsection{Tape Recording in Television}

During the early days of television, both audio and video were recorded on tapes. However, tapes have a limited dynamic range, necessitating the use of noise reduction techniques.

\subsection{Noise Reduction Techniques}

\subsubsection{Dolby Noise Reduction}

Dolby, named after the engineer who developed it, introduced several noise reduction systems:

\begin{itemize}
    \item \textbf{Dolby A, B, C:} Different levels of noise reduction, with Dolby C being the highest performer.
    \item \textbf{Compression and Expansion:} High-quality cassette decks use Dolby B compression for home use, while studios and cinemas use Dolby C. These systems compress the audio signal during recording and expand it during playback to reduce noise.
\end{itemize}

\subsubsection{Types of Recording Tapes}

There are three main types of tapes with varying dynamic ranges:

\begin{itemize}
    \item \textbf{Ferric Oxide Tape:} Also known as brown tape, with a dynamic range of about 55 dB.
    \item \textbf{Chromium Tape:} Offers a slightly higher dynamic range of approximately 60-62 dB.
    \item \textbf{Metal Tape:} More expensive, providing a dynamic range of around 65 dB.
\end{itemize}

\subsection{Practical Log Amplifier Circuit}
\begin{figure}[H]
    \centering
    \includegraphics[width=\linewidth]{TransitLife.png}
    \caption{Practical logarithmic amplifier using matched transistors}
    \label{fig:question_image}
\end{figure}
In the last class, we covered a practical log amplifier circuit. Unlike theoretical circuits that assume ideal components, the practical log amplifier accounts for non-idealities such as offset voltages in operational amplifiers (op-amps).

\subsubsection{Circuit Overview}

The practical log amplifier includes null offsetting circuits to cancel out offset voltages. Key components include:

\begin{itemize}
    \item \textbf{Op-Amp A1:} Can be offset using potentiometer P1. If P1 is set halfway, the voltage across the 10 MΩ resistor is zero.
    \item \textbf{Balancing Current:} The circuit provides a small current to balance the differential inputs of the op-amp, accommodating variations in current gain and transistor characteristics.
\end{itemize}

\subsection{Offset Nulling in Op-Amps}

Even when the inputs of an op-amp are grounded, an output offset voltage may appear. The null offsetting circuit allows adjustment of this offset:

\begin{equation}
    V_{\text{out}} = 0 \text{ V}
\end{equation}

If the output is not zero, adjust potentiometer P1 until it is.

\subsection{Circuit Analysis}

\subsubsection{Analyzing the Output of Amplifier A2}

Consider amplifier A2 with the following components:

\begin{itemize}
    \item \textbf{Resistor R4 (29.5 kΩ):} Feedback resistor.
    \item \textbf{Resistor R3 (0.5 kΩ):} Connected to ground.
\end{itemize}

The gain of the op-amp is approximately:

\[
\text{Gain} = 1 + \frac{R4}{R3} = 1 + \frac{29.5 \text{ kΩ}}{0.5 \text{ kΩ}} = 60
\]

Given a supply voltage of ±15 V, the maximum output voltage is ±15 V, limiting the input voltage to:

\[
V_{\text{in}} = \frac{15 \text{ V}}{60} = 0.25 \text{ V}
\]

For safer operation, using a maximum input of 5 V results in:

\[
V_{\text{in}} = \frac{5 \text{ V}}{60} \approx 0.083 \text{ V}
\]

\subsubsection{Determining Voltages and Currents}

Using Kirchhoff's Voltage Law (KVL) and transistor equations:

\begin{align}
    I_C &= I_S e^{\frac{V_{BE}}{n V_T}} \\
    \ln\left(\frac{I_C}{I_S}\right) &= \frac{V_{BE}}{n V_T}
\end{align}

Where:

\begin{itemize}
    \item \( I_C \) is the collector current.
    \item \( I_S \) is the saturation current.
    \item \( V_{BE} \) is the base-emitter voltage.
    \item \( n \) is the ideality factor.
    \item \( V_T \) is the thermal voltage.
\end{itemize}

\subsection{Temperature Dependence and Calibration}

\subsubsection{Temperature Independence}

To make \( V_{\text{out}} \) independent of temperature, adjust resistor \( R3 \) to decrease as temperature increases. This can be achieved by using a resistor with a positive temperature coefficient:

\[
R3(T) = R3 \times (1 + \alpha T)
\]

Where \( \alpha \) is the temperature coefficient.

\subsection{Calibration Steps}

To calibrate the practical log amplifier in the lab:

\begin{enumerate}
    \item \textbf{Set \( V_S = 0 \text{ V} \):} Short the input to ground. The output \( V_{\text{dash}} \) should be zero. Adjust potentiometer P1 until \( V_{\text{dash}} = 0 \text{ V} \).
    \item \textbf{Set \( V_S = V_R \frac{R1}{R2} \):} Adjust potentiometer P2 until the output voltage \( V_{\text{not}} = 0 \text{ V} \).
\end{enumerate}

Once calibrated, the circuit is ready to compute logarithms.

\subsection{Conclusion}

By selecting appropriate resistors \( R3 \) and \( R4 \) with temperature coefficients, the output \( V_{\text{not}} \) can be made temperature independent. The final equation for \( V \) in terms of circuit parameters is:

\[
V = -n V_T \ln\left(\frac{I1}{I_C2}\right)
\]

Where \( I1 \) is related to \( V_S \) and \( I_C2 \) is approximated as \( \frac{V_R}{R2} \).

\subsection{Final Remarks}

Ensure all variables are correctly set and equations are accurately applied. In future classes, we will explore more refined techniques to eliminate the need for approximations in current calculations. Please prepare for the next exercise, which focuses on designing a temperature-independent log amplifier.

\begin{center}\rule{0.5\linewidth}{0.5pt}\end{center}

\section*{Lecture 11: 28/10/2024}

\subsection{Introduction}
In this lecture, we are concluding our discussion on logarithmic amplifiers. In the previous class, I demonstrated a practical logarithmic amplifier, often referred to as the Logamp, which you can construct in the lab to obtain tangible results. During the derivation of this practical Logamp, we successfully calibrated the circuit. However, we encountered the necessity to make certain approximations in determining the transistor current.

\subsection{Circuit Modification}
Initially, we approximated that \( I_{C2} = \frac{V_R}{R} \), where \( R \) is the resistor connected to \( V_{CC} \). This approximation was based on the resistor value and the voltage drop \( V_R \). However, in the modified circuit presented here, we have replaced this resistor with a combination of \( R_5 \) and operational amplifier A3. This change eliminates the need for the previous approximation, leading to a more accurate and reliable circuit. This enhanced design is termed the \textbf{Improved Transistor Logamp}.

\subsection{Circuit Components and Functionality}
The Improved Transistor Logamp is practical due to the inclusion of several key components:
\begin{itemize}
    \item \textbf{Potentiometers}: \( P1 \) with a value of 100kΩ and \( P2 \) with a value of 10kΩ are used to set the potential and offset the output of operational amplifiers A2 and A3, respectively. These ensure that the outputs are zero when the differential input voltage is zero.
    \item \textbf{Protection Diodes}: Diodes \( D1 \) and \( D2 \) are incorporated to protect the operational amplifiers from excessive voltage that could damage the transistors.
\end{itemize}

\subsection{Analyzing the Circuit}
To analyze the logarithmic relationship between voltage and current in this circuit, we begin at Point A. Diodes \( D1 \) and \( D2 \) serve as protective elements, preventing the op-amps from operating beyond their voltage limits. Applying Kirchhoff's Voltage Law (KVL) around Loop ABC, we derive the following equations:

\[
- V_{BE1} + V_{BE2} = V
\]

Assuming the arbitrary voltage \( V \) at the inverting input of operational amplifier A2, the voltage drop across resistor \( R3 \) is given by:

\[
V + \frac{V}{R_T + R_4} R_3 = V_{ref}
\]

Simplifying, we obtain:

\[
V_{ref} = V \left(1 + \frac{R_3}{R_T + R_4}\right)
\]

\subsection{Constant Current Source}
The circuit includes a constant current source formed by \( R5 \), diode \( D3 \), and operational amplifier A3. The current \( I_5 \) through \( R5 \) is defined as:

\[
I_5 = \frac{V_Z}{R5}
\]

Assuming a high current gain (\( \beta \)) for transistor Q3, the collector current \( I_{C3} \) remains approximately equal to \( I5 \), making it a stable reference current independent of supply voltage variations.

\subsection{Calibration Procedure}
Calibrating the Improved Transistor Logamp involves the following steps:

\begin{enumerate}
    \item \textbf{Initial Setup}: Short-circuit \( V_S \) to the ground. Ensure that the output voltage \( V1' \) from operational amplifier A1 equals zero by adjusting potentiometer \( P1 \).
    \item \textbf{Setting Reference Current}: Set \( V_S = \frac{R1 \cdot V_Z}{R5} \). Verify that the logarithmic relationship holds by ensuring \( V_{ref} = 0 \).
\end{enumerate}

\subsection{Advantages of the Improved Design}
The Improved Transistor Logamp offers several benefits:
\begin{itemize}
    \item \textbf{Elimination of Approximations}: By modifying the resistor configuration, the need for previous approximations is removed, enhancing accuracy.
    \item \textbf{Stable Reference Current}: The constant current source ensures that \( I_{C2} \) remains unaffected by supply voltage fluctuations.
    \item \textbf{Protected Operation}: Protection diodes safeguard the operational amplifiers from voltage spikes, preventing potential damage.
\end{itemize}

\subsection{Applications}
Logarithmic amplifiers are crucial in various applications such as:
\begin{itemize}
    \item \textbf{Analog Computation}: Used in solving differential equations and performing logarithmic transformations.
    \item \textbf{Compressors}: In audio processing, logamps help in compressing and expanding dynamic ranges.
    \item \textbf{Instrumentation}: Provide logarithmic indications for measurements and controls.
\end{itemize}

\subsection{Conclusion}
The Improved Transistor Logamp presents a refined approach to logarithmic amplification, offering enhanced accuracy and stability. By integrating a constant current source and protective diodes, this design is well-suited for practical laboratory applications and various industrial uses.
\begin{center}\rule{0.5\linewidth}{0.5pt}\end{center}

\section*{Lecture 12: 31/10/2024}

\subsection*{Determining \( V_{\text{o}} \)}

Using Equation 1, we can rearrange and solve for \( V \) to determine \( V_{\text{not}} \). This implies that if we know \( V \), then we can express \( V_{\text{0}} \) as:

\[
V_{\text{o}} = V \cdot \frac{R_T + R_3}{R_T}
\]

Remember, \( R_T \) is a temperature-dependent parameter. This relationship will be used to design the circuit so that the output does not vary as a function of \( V_T \), which is temperature dependent. However, for simplicity, I'll be using Equation 1 and Equation 2. Now, let's determine \( V \) as a function of the base-source voltage (\( V_{BS} \)) of the transistor.

\subsection*{Analyzing the Transistor Circuit}

I'll start at Point A, the base of Q2. If it's not very clear, Point A is here, Point B is here, and Point C is the base of Q1. Starting at Point A, we traverse against the potential \( V_{BE2} \). Applying Kirchhoff's Voltage Law (KVL) to loop ABC implies:

\[
-V + V_{BE1} = V
\]

But what is \( V_{BE2} \) in terms of the known voltage-current relations for the transistor? We've already derived it, so I'll just present the result:

\[
-n V \ln\left(\frac{I_{C2}}{I_2}\right) + \ln\left(\frac{I_{C1}}{I_{\text{not1}}}\right) = V
\]

Next, let's simplify Equation 4:

\[
V = n V \ln\left(\frac{I_{C1}}{I_{\text{not1}}}\right) \cdot I_{\text{not2}} I_2
\]

We can state that for my transistors, the saturation currents are equal. Therefore, Equation 5 can be written as:

\[
n E \ln\left(\frac{I_{C1}}{I_2}\right) = V
\]

Now, we need to find \( I_{C1} \) and \( I_{C2} \) in terms of the circuit parameters.

\subsection*{Determining Collector Currents}

Returning to the circuit, let's start with finding \( I_{C1} \). \( I_{C1} \) is the collector current of the first transistor. In this circuit, this is your \( I_{C1} \), and the current over here is \( I_2 \).

Since the positive input of A1 is grounded, it means the negative input of A1 is a virtual earth. Therefore, the current that flows through resistor \( R1 \) is:

\[
I_1 = \frac{V_{S1}}{R1}
\]

Similarly, for the second transistor, if you look at the input of amplifier A2, the positive side is grounded, and therefore the negative input of A2 is also a virtual earth. So, the collector of Q2 is also at virtual earth, and therefore the current \( I_{C2} \) is the current that flows through resistor \( R2 \), which is:

\[
I_2 = \frac{V_{S2}}{R2}
\]

We can substitute these values:

\[
I_{C1} = \frac{V_{S1}}{R1}, \quad I_{C2} = \frac{V_{S2}}{R2}
\]

Let us denote these results as Equation 7.

\subsection*{Substituting into Equations}

From Equation 5 and Equation 6, we substitute the results of Equation 7 into Equation 6:

\[
V = n V \ln\left(\frac{I_{C1}}{I_2}\right) = n V \ln\left(\frac{\frac{V_{S1}}{R1}}{\frac{V_{S2}}{R2}}\right) = n V \ln\left(\frac{V_{S1} \cdot R2}{V_{S2} \cdot R1}\right)
\]

Thus, we have determined \( V \). Now, knowing \( V \), we can determine the output voltage \( V_{\text{not}} \) from Equation 2:

\[
V_{\text{not}} = V \cdot \frac{R_T + R3}{R_T}
\]

Expanding this, we obtain:

\[
V_{\text{not}} = \left(1 + \frac{R3}{R_T}\right) V
\]

Substituting the expression for \( V \):

\[
V_{\text{not}} = \left(1 + \frac{R3}{R_T}\right) \cdot n V_T \ln\left(\frac{R2}{R1} \cdot \frac{V_{S1}}{V_{S2}}\right)
\]

This demonstrates that the circuit performs logarithmic computation—a log amplifier. The scaling factor \( K \) can be adjusted via the second voltage \( V_{S2} \):

\[
K = \frac{R2}{R1} \cdot V_{S2}
\]

For example, in a circuit that performs compression, you can vary the compression rate \( K \), which can be very useful in various circuit applications.

\subsection*{Conclusion}

I hope everyone has understood how we derived these relationships. If there are no questions, we can proceed to the next topic.

\subsection*{Analog Multipliers and Dividers}

The next topic is analog multipliers and dividers. However, we'll focus on analog multipliers because dividers can be derived from multipliers. Apart from microprocessors and microcontrollers, analog multipliers are among the other very useful electronic circuits. Without them, there would be no long-distance communication because modern modulation techniques rely heavily on multipliers. For instance, FM, AM, FSK, QPSK, SSB, DSB—all these circuits require an analog multiplier. Similarly, computer modems also require analog multipliers.

\subsection*{Definition of an Analog Multiplier}

An ideal analog multiplier is a circuit whose output is a function of two input voltages \( V_X \) and \( V_Y \), such that:

\[
V_{\text{out}} = K \cdot V_X \cdot V_Y
\]

where \( K \) is a constant scaling factor. The symbol of a multiplier is typically represented as a rectangle with a triangle on top and an output.

\subsection*{Characteristics of an Ideal Multiplier}


1. \textbf{Infinite Input Impedance:} The input signals $V_X$ and $V_Y$ should have infinite input impedance, meaning the multiplier does not load the inputs.\\

2. \textbf{Zero Output Impedance:} The output impedance should be zero so that all the generated voltage appears at the load.\\

3. \textbf{Infinite Bandwidth:} It should be able to multiply signals from very low frequencies (DC) to infinitely high frequencies.\\

4. \textbf{Linearity:} The output should be linear for all ranges of the input. For example, if $V_X$ is fixed at 1V and $V_Y$ changes from 1V to 10V, the output should change linearly without distortion.


\subsection*{Practical Limitations of Real Multipliers}

1. \textbf{Limited Voltage Swing:} Typically restricted to $\pm 10V$ for normal operations, despite the power supply being $\pm 15V$.\\

2. \textbf{Limited Input Signals:} $V_X$ and $V_Y$ are usually limited to $\pm 10V$.\\

3. \textbf{Limited Output Current:} Real multipliers cannot provide infinite current; they have limited output current capabilities due to transistor limitations.\\

4. \textbf{Non-Zero Output Impedance:} Real multipliers have an output impedance that is not zero, requiring a large load resistance to ensure maximum voltage transfer.\\

5. \textbf{Offset Errors:} Due to the use of differential amplifiers, real multipliers suffer from offset errors.\\

6. \textbf{Finite Bandwidth:} Depends on the transistors used; high-frequency transistors can multiply signals up to GHz ranges.

\subsection*{Implementing Multipliers}

There are several practical schemes to implement analog multipliers:

\begin{enumerate}
    \item \textbf{Log-Anti-Log Method:} Utilizes logarithmic and anti-logarithmic circuits to perform multiplication based on the identity
    \[
    XY = \text{antilog}(\log X + \log Y).
    \]
    
    \item \textbf{Quarter Square Technique:} Uses the identity
    \[
    XY = \frac{1}{4} \left[ (X + Y)^2 - (X - Y)^2 \right]
    \]
    to implement multiplication through adders, subtractors, and squaring functions.

    \item \textbf{Transconductance Multiplier:} The most widely used method in the industry, involving variable transconductance elements.
\end{enumerate}

\subsubsection*{Log-Anti-Log Method}

In the Log-Anti-Log method, the relationship \( XY = \text{antilog}(\log X + \log Y) \) is used. To implement this multiplier:

\begin{enumerate}
    \item Start with two log amplifiers that compute \( \log X \) and \( \log Y \).
    \item Sum the outputs of the log amplifiers.
    \item Pass the sum through an anti-log amplifier to obtain \( XY \).
\end{enumerate}

However, designing effective log amplifiers requires many operational amplifiers (Op-Amps), which introduces significant offset errors and increases the cost due to the large number of components. Additionally, log and anti-log circuits are typically unidirectional and have limited dynamic range and bandwidth.

\subsubsection*{Quarter Square Technique}

The Quarter Square Technique employs the identity:

\[
XY = \frac{1}{4} \left[(X + Y)^2 - (X - Y)^2\right]
\]

Implementation involves:

\begin{enumerate}
    \item Adding and subtracting the input voltages \( X \) and \( Y \).
    \item Squaring the results using diode-based squaring circuits.
    \item Subtracting the squared terms.
    \item Multiplying the result by a quarter to obtain \( XY \).
\end{enumerate}

While this method is simpler than the Log-Anti-Log method, it still requires precise squaring circuits and suffers from limitations related to the speed and offset errors of Op-Amps.

\subsubsection*{Transconductance Multiplier}

The Transconductance Multiplier is the most commonly used method in the industry. It leverages variable transconductance elements to achieve multiplication with higher accuracy and better performance compared to other methods. This technique is preferred due to its scalability and efficiency in various applications, including modulation and signal processing.

\subsection*{Conclusion}

These are the primary methods used to implement analog multipliers. While each method has its advantages and limitations, the choice of technique depends on the specific requirements of the application, such as cost, accuracy, speed, and frequency response.

\subsection*{Questions and Next Steps}

I hope everyone has understood how we derived the logarithmic computations and the various methods to implement analog multipliers. Are there any questions? If not, we can proceed to the next topic.
\begin{center}\rule{0.5\linewidth}{0.5pt}\end{center}

\section*{Lecture 13: 04/11/2024}

\subsection{Introduction to Analog Multipliers}

Today, we will continue our discussion on analog multipliers, a topic we began yesterday. In telecommunications, analog multipliers are often represented by a symbol consisting of a square with a triangle on top, typically marked as \( x \) and \( y \). Here, the signal \( V_x \) is connected to one input, \( V_y \) to the other, and the output is \( V_{out} \). All these signals are referenced with respect to the Earth.

In the introduction, we explored the concept of an ID multiplier, its limitations in practical applications, and how we can realize analog multipliers effectively.

\subsection{Log-Antilog Method}

The first scheme we discussed is the \textbf{Log-Antilog} method. This approach utilizes log amplifiers and anti-log amplifiers. The fundamental equation governing this method is:

\[
XY = \text{antilog}(\log X + \log Y)
\]

To implement this circuit, two log amplifiers are required, followed by an anti-log amplifier to obtain the product \( XY \). However, upon studying high-quality log and anti-log amplifiers, we observe that they are highly complicated, containing numerous operational amplifiers (op-amps). These op-amps suffer from offsets, necessitating multiple setting circuits to compensate. Additionally, log and anti-log circuits that handle unipolar signals are not unidirectional, making the design even more complex. Due to these challenges, this implementation is seldom used in practice and is primarily considered academic.

\subsection{Quarter Square Method}

The next implementation technique is the \textbf{Quarter Square} method, which is the focus of today's lecture. If time permits, we will also introduce the \textbf{Transconductance Method}, sometimes referred to as the \textbf{Variable Transconductance Multiplier} circuit.

\subsubsection{Mathematical Basis}

In the Quarter Square method, we use the identity:

\[
\frac{1}{4}[(x + y)^2 - (x - y)^2] = XY
\]

Expanding both terms:

\[
(x + y)^2 = x^2 + 2xy + y^2
\]
\[
(x - y)^2 = x^2 - 2xy + y^2
\]

Subtracting the two:

\[
(x + y)^2 - (x - y)^2 = 4xy \implies \frac{1}{4}(x + y)^2 - \frac{1}{4}(x - y)^2 = XY
\]

\subsubsection{Circuit Implementation}

To implement this identity, the circuit requires:

\begin{itemize}
    \item Adders for \( x + y \) and \( x - y \)
    \item Squaring circuits for both \( (x + y) \) and \( (x - y) \)
    \item A subtractor to compute \( \frac{1}{4}[(x + y)^2 - (x - y)^2] \)
\end{itemize}

The block diagram of the circuit is as follows:

\begin{figure}[h]
    \centering
    \includegraphics[width=\linewidth]{CircuitImplementation.png}
    \caption{Quarter Square Multiplier Circuit}
    \label{fig:quarter_square}
\end{figure}

In this circuit:

\begin{itemize}
    \item \( V_x \) is connected to the inverting input of amplifier \( A_1 \), resulting in an output of \( -x \).
    \item Similarly, \( V_y \) is connected to the inverting input of amplifier \( A_2 \), resulting in an output of \( -y \).
    \item These signals pass through diodes \( D_2 \), \( D_3 \), and \( D_4 \) to perform the necessary additions and subtractions.
\end{itemize}

\subsection{Simplified Analysis}

To simplify the analysis, consider a simplified circuit with four diodes \( D_1, D_2, D_3, D_4 \). The currents through these diodes are denoted as \( I_A \) and \( I_B \).

Let:

\[
Z = X + Y
\]

Then:

\[
I_B = I_{N} \left(e^{KZ} - 1\right)
\]
\[
I_A = I_{N} \left(-KZ + \frac{(KZ)^2}{2}\right)
\]

Summing the currents:

\[
I_1 = I_A + I_B = I_{N} \left(2KZ^2\right)
\]

\subsection{Output Voltage Calculation}

The output voltage \( V_{out} \) can be expressed as:

\[
V_{out} = -I_3 R_F = -I_{N} K^2 (X^2 + Y^2 - (X - Y)^2)
\]

To achieve multiplication, we set:

\[
R_F I_{N} K^2 = \frac{1}{4}
\]

\subsection{Limitations of the Quarter Square Method}

While the Quarter Square method simplifies the multiplier design, the accuracy is limited due to:

\begin{itemize}
    \item Approximate squaring using diodes
    \item Offset errors from operational amplifiers
    \item Complexity in setting up the necessary resistor and amplifier configurations
\end{itemize}

Therefore, this implementation is rarely used in practical applications.

\subsection{Variable Transconductance Method}

The next scheme is the \textbf{Variable Transconductance Method}, also known as the \textbf{Transconductance Multiplier}. This method varies the transconductance of a differential amplifier using a second signal to achieve multiplication.

\subsubsection{Basic Two Quadrant Multiplier}

The simplest implementation of this method is the \textbf{Basic Two Quadrant Multiplier}, which provides multiplication results in only two quadrants. Here, one variable can take positive values, while the other can take both positive and negative values.

The output voltage \( V_{out} \) is given by:

\[
V_{out} = -G_m V_1 R_L
\]

where \( G_m \) is the transconductance, which varies with input \( V_2 \):

\[
G_m = K V_2
\]

Substituting:

\[
V_{out} = -K V_1 V_2 R_L
\]

\subsubsection{Circuit Implementation Challenges}

This method faces several challenges:

\begin{itemize}
    \item Difficulty in measuring inputs and outputs simultaneously due to different ground references
    \item Limitations in handling DC voltages unless \( V_2 \) overcomes the \( V_{BE} \) threshold
    \item Floating output requiring additional amplification to reference ground
\end{itemize}

These limitations make the Basic Two Quadrant Multiplier less practical for certain applications.

\subsection{Gilbert Multiplier Cell}

To overcome the limitations of the Basic Two Quadrant Multiplier, the \textbf{Gilbert Multiplier Cell} was developed. This configuration connects two differential amplifiers in a parallel and specialized manner, enhancing performance and allowing for high-frequency operations.

\subsubsection{Circuit Overview}

\begin{figure}[h]
    \centering
    \includegraphics[width=0.3\textwidth]{CELL.png}
    \caption{Gilbert Multiplier Cell}
    \label{fig:gilbert_multiplier}
\end{figure}

In this circuit:

\begin{itemize}
    \item \( Q_1 \) and \( Q_2 \) form the first differential amplifier.
    \item \( Q_3 \) and \( Q_4 \) form the second differential amplifier.
    \item \( Q_5 \) and \( Q_6 \) are used to vary the transconductance based on the input \( V_2 \).
    \item A constant current source \( I_{total} \) ensures stable operation.
\end{itemize}

\subsubsection{Assumptions for Analysis}

\begin{itemize}
    \item Transistors \( Q_1, Q_2, Q_3, Q_4, Q_5, Q_6 \) are matched for optimal performance.
    \item Base currents are negligible due to high-beta transistors.
    \item The circuit is properly biased, ignoring static currents for dynamic analysis.
\end{itemize}

\subsubsection{Operation Principle}

The Gilbert Multiplier Cell operates by shifting currents between transistors based on the input voltages \( V_1 \) and \( V_2 \). The dynamic currents, resulting from signal variations, are analyzed to determine the output multiplication.

\subsection{Conclusion}

Analog multiplier circuits, while theoretically sound, present significant practical challenges in terms of complexity, accuracy, and biasing requirements. The Quarter Square method, although simpler, suffers from inaccuracies due to diode approximations and op-amp offsets. The Variable Transconductance method offers better performance but introduces its own set of challenges, particularly regarding ground references and DC operation.

The Gilbert Multiplier Cell emerges as a robust solution, addressing many of these limitations by providing high-frequency operation and better accuracy, making it suitable for applications in communication receivers and signal processing circuits.

\subsection{Applications}

Analog multipliers are essential in various applications, including:

\begin{itemize}
    \item \textbf{Communication Receivers:} Used in modulators and demodulators for processes like Double Side Band (DSB) modulation, Frequency Modulation (FM), and Phase Modulation (PM).
    \item \textbf{Frequency Translation:} Multiplying two cosine waves yields sum and difference frequencies, useful for signal processing.
    \item \textbf{Modulation Techniques:} Employed in both modulation and demodulation processes within modems.
\end{itemize}

Despite their complexities, analog multipliers remain a cornerstone in high-frequency and signal processing applications, particularly where precision and speed are paramount.

\subsection{Further Improvements}

To enhance the basic differential multiplier:

\begin{itemize}
    \item Introduce additional amplification stages to reference the output to the ground.
    \item Employ robust biasing techniques to minimize offset errors.
    \item Utilize matched transistors to maintain symmetry and improve accuracy.
\end{itemize}

These improvements pave the way for more reliable and accurate analog multipliers, expanding their applicability in advanced electronic systems.\
\begin{center}\rule{0.5\linewidth}{0.5pt}\end{center}

\section*{Lecture 14: 07/11/2024}

\subsection{Introduction}
Welcome everyone. It looks like a few of you are still joining the session. We'll wait for a few more students to join before we begin. Today, we'll cover the Gilbert's Multiplier Cell, its design, analysis, and practical implementation.

\subsection{Gilbert's Multiplier Cell Overview}
In our last class, we introduced the Gilbert's Multiplier Cell, which is capable of performing multiplication operations. The cell comprises:
\begin{itemize}
    \item Two differential amplifiers in parallel forming the output.
    \item Another differential amplifier comprising of transistors $Q_5$ and $Q_6$ to vary the transconductance.
\end{itemize}

\subsection{Current Analysis}
Let’s define the collector currents as follows:
\begin{align*}
    &I_1 = \text{Collector current of } Q_1, \\
    &I_2 = \text{Collector current of } Q_2, \\
    &I_3 = \text{Collector current of } Q_3, \\
    &I_4 = \text{Collector current of } Q_4, \\
    &I_5 = \text{Collector current of } Q_5, \\
    &I_6 = \text{Collector current of } Q_6.
\end{align*}
These are the signal currents, not the bias currents.

\subsubsection{Basic Equations}
Based on the assumption that the emitter current is approximately equal to the collector current due to the high gain of the transistors, we derive:
\[
I_1 + I_2 = I_5,
\]
\[
I_3 + I_4 = I_6.
\]
For a general differential amplifier, the relationship is given by:
\[
i_1 - i_2 = g_{m12} \cdot V_1 R_L,
\]
where $g_{m12}$ is the transconductance of transistors $Q_1$ and $Q_2$, and $V_1$ is the input voltage.

\subsection{Output Voltage Analysis}
The output voltage $V_{\text{out}}$ can be expressed as the difference in voltage drops across the load resistors:
\[
V_{\text{out}} = V_{CC} - I_2 R_L - I_4 R_L.
\]
Simplifying, we obtain:
\[
V_{\text{out}} = V_{CC} - R_L (I_2 + I_4).
\]
Substituting the current relationships, we derive:
\[
V_{\text{out}} = V_1 R_L \left( \frac{I_6}{V_T} - \frac{I_5}{V_T} \right),
\]
where $V_T$ is the thermal voltage.

\subsection{Relationship Between $V_1$, $V_2$, $I_5$, and $I_6$}
To establish how $V_{\text{out}}$ varies as a function of $V_1$ and $V_2$, consider the collector currents of $Q_5$ and $Q_6$. The equations are derived based on the differential pair configurations and the presence of emitter degeneration resistances.

\subsection{Case Analysis}
\subsubsection{Case 1: $R_E = 0$}
When the emitter resistance $R_E$ is zero, the differential amplifier comprising $Q_5$ and $Q_6$ behaves identically to the general differential amplifier. The relationship simplifies to:
\[
I_6 - I_5 = g_{m56} \cdot V_2.
\]
Substituting into the output voltage equation:
\[
V_{\text{out}} = \frac{V_1 R_L}{V_T} (I_6 - I_5).
\]
Since $g_{m56} = \frac{I_6}{V_T}$ and $g_{m12} = \frac{I_5}{V_T}$, we can further express:
\[
V_{\text{out}} = V_1 V_2 \cdot \frac{R_L}{V_T}.
\]
This represents the standard form of the multiplier with a scalar gain.

\subsubsection{Case 2: $R_E \neq 0$}
When emitter resistance $R_E$ is present, the relationship between $V_2$, $I_5$, and $I_6$ changes. The analysis involves applying Kirchhoff's Voltage Law (KVL) to the loop involving $Q_5$ and $Q_6$:
\[
V_2 = V_{BE5} - V_{BE6}.
\]
Taking the natural logarithm and linearizing for small signal variations, we arrive at:
\[
V_2 = (I_5 - I_6) R_E.
\]
Substituting back into the output voltage equation:
\[
V_{\text{out}} = -\frac{V_1 R_L}{V_T} \cdot \frac{V_2}{R_E}.
\]
Thus, the gain is now inversely proportional to $R_E$, improving linearity at the expense of reduced gain:
\[
V_{\text{out}} = -\frac{V_1 V_2 R_L}{V_T R_E}.
\]

\subsection{Advantages of the Gilbert's Multiplier Cell}
\begin{itemize}
    \item \textbf{Four-Quadrant Operation}: Capable of handling both positive and negative input voltages, allowing multiplication in all four quadrants.
    \item \textbf{Monolithic Fabrication}: Easily fabricated on a single silicon wafer, enabling mass production and integration into ICs.
    \item \textbf{Large Bandwidth}: Supports high-frequency operations, suitable for applications in communication circuits.
    \item \textbf{Low-Cost Production}: Economies of scale in production reduce the cost per unit significantly.
\end{itemize}

\subsection{Limitations of the Gilbert's Multiplier Cell}
\begin{itemize}
    \item \textbf{Limited Input Range}: The differential input voltage is restricted to approximately $\pm 4V_T$ to prevent clipping.
    \item \textbf{Floating Output}: The output does not have a common reference to ground, necessitating additional circuitry for proper interfacing.
    \item \textbf{Differential Resistor Mismatch}: Discrepancies in input resistances seen by $V_1$ and $V_2$ can affect the performance and linearity.
\end{itemize}

\subsection{Practical Implementation of the Multiplier}
A practical multiplier circuit incorporates additional components to enhance performance:
\begin{itemize}
    \item \textbf{Preconditioning Circuit}: Comprising transistors $Q_7$ and $Q_8$, this stage distorts the input signal using a hyperbolic tangent inverse function, significantly extending the dynamic range to approximately $\pm 10$ volts.
    \item \textbf{Constant Current Sources}: Transistors $Q_{15}$ and $Q_{16}$ form the current sources required for biasing the differential pairs.
\end{itemize}

\subsubsection{Dynamic Range Extension}
By applying the preconditioning circuit, the input signals $V_Z$ and $V_Y$ are referenced to ground, allowing for a larger and more linear dynamic range in the multiplication operation. The extended range improves the multiplier's usability in various applications without distortion.

\subsection{Conclusion}
Today, we've delved into the design and analysis of the Gilbert's Multiplier Cell, exploring both theoretical and practical aspects. We discussed the critical role of emitter resistances, the impact on gain and linearity, and the advantages that make the Gilbert's multiplier a staple in analog circuit design. In the next class, we'll further analyze the multiplier cell's performance with different emitter resistances and explore additional configuration nuances to enhance its functionality.

If there are any questions or if you need clarification on today's topics, feel free to reach out or bring them to the next class session. Please ensure to bring your lab circuits on time to avoid delays in our schedule.

\subsection*{Questions and Answers}
\begin{itemize}
    \item \textbf{Q:} Why is the output of the multiplier floating?
    
    \textbf{A:} The output is floating because it does not have a common reference to ground. This requires an additional amplifier to establish a proper reference point for the output signal.
    
    \item \textbf{Q:} How does the emitter resistance affect the multiplier's performance?
    
    \textbf{A:} Increasing the emitter resistance improves the linearity of the multiplier by expanding the dynamic range but reduces the overall gain of the circuit.
\end{itemize}
\begin{center}\rule{0.5\linewidth}{0.5pt}\end{center}

\section*{Lecture 15: 11/11/2024}

In the last class, we concluded the analysis of the Gilbert multiplier cell. However, for it to be practically useful, it requires an associated output circuitry. The circuit displayed here is not a practical Gilbert multiplier as the output part is missing. Due to space constraints, I couldn't draw it here. Nevertheless, I'll provide an overview of the complete ON Semiconductor Gilbert multiplier MCU.

\subsection{Overview of MC1495 Linear Gilbert Multiplier}
\begin{figure}[h]
    \centering
    \includegraphics[width=\linewidth]{MC1495.png}
    \caption{MC1495}
    \label{fig:gilbert_multiplier}
\end{figure}

The MC1495 is a linear Gilbert multiplier that offers excellent linearity and detailed technical information. The schematic showcases its internal structure. On the left side, there is the predistortion circuit comprising transistors Q1 AND Q2; and a gilbert multiplier with Q5, Q6, Q7, Q8, Q3, and Q4. These transistors are equivalent to Q5 and Q6 in our previous design.

\subsection{Integration of the Predistortion Circuit and Gilbert Multiplier Cell}

The Gilbert multiplier cell and the predistortion circuit form the input part of the multiplier. To obtain an output referenced to the ground, differential amplifiers are used. The MC1494 and MC1495 ICs are almost identical internally, with both utilizing differential amplifiers for output ground referencing.

\subsection{Complete Gilbert Multiplier Circuit}

The complete circuit includes the MC1494 within a dashed area, encompassing the Gilbert multiplier cell. The output is routed through differential amplifiers to ensure it is referenced to the ground. This configuration results in a practical multiplier with three main components:
\begin{enumerate}
    \item \textbf{Predistortion (Preconditioning) Circuit}: Extends the linear range of the input signal.
    \item \textbf{Gilbert Multiplier Cell}: Multiplies the input signals.
    \item \textbf{Output Stage}: References the output to the ground using differential amplifiers.
\end{enumerate}

\subsection{Dynamic Range and Gain Analysis}

When analyzing the impact of incorporating resistor $R_E$, we observe that:
\begin{itemize}
    \item \textbf{Gain}: The inclusion of $R_E$ decreases the system gain.
    \item \textbf{Dynamic Range}: $R_E$ extends the dynamic range of the signal $V_2$.
\end{itemize}

For example, if $R_E = 0$, the maximum voltage of $V_2$ is approximately $4V_T$ (where $V_T \approx 26\,\text{mV}$ at room temperature, translating to approximately $100\,\text{mV}$). This is limited for applications requiring a larger dynamic range, such as multiplying $10\,\text{V} \times 10\,\text{V}$ signals.

\subsection{Practical Implementations of the Gilbert Multiplier}
\begin{figure}[h]
    \centering
    \includegraphics[width=\linewidth]{Practical.png} % Adjust width as necessary
    \caption{}
    \label{fig:Practical}
\end{figure}
\subsubsection{Circuit Modification with Resistor $R_1$}

In the new arrangement, resistor $R_1$ is introduced between the emitters of Q9 and Q10 (previously Q5 and Q6) along with a bridging resistor of $35\,\text{k}\Omega$. This modifies the constant current source setup, enhancing the practicality of the multiplier.

\subsubsection{DC Biasing and Current Analysis}

To determine the DC bias of the circuit:
\begin{enumerate}
    \item The current $I_{BB}$ from the constant current source is split equally between Q15 and Q16, resulting in $I_{BB}/2$ for each.
    \item The voltage at the base of Q17 ($V_P$) is calculated as:
    \[
    V_P = 4K \times I_{BB} + V_{BE17}
    \]
    \item The emitter current of Q16 ($I_E16$) is:
    \[
    I_{E16} = \frac{V_P - V_{BE16}}{8K}
    \]
    \item Since $Q5$ and $Q6$ are matched, the currents at their meters are each $I_{BB}/2$.
\end{enumerate}

\subsubsection{Signal Analysis}

When analyzing the collector currents $I_5$ and $I_6$ as a function of $V_2$, two cases are considered:
\begin{enumerate}
    \item \textbf{Case 1}: $R_1 = 0$
    \item \textbf{Case 2}: $R_1 \neq 0$
\end{enumerate}

Applying Kirchhoff's Voltage Law (KVL) and simplifying under the small-signal approximation ($\Delta I = 2I_X$ is small), the relationship between $I_5$, $I_6$, and $V_2$ is derived as:
\[
V_2 \approx \frac{I}{2} \times (I_5 - I_6)
\]

Substituting into the original Gilbert multiplier equation:
\[
V_{out} = -\frac{R_L V_1 V_2}{V_T R_E/2} = -\frac{2R_L V_1 V_2}{V_T R_E}
\]
This maintains the multiplication functionality while reducing the gain due to the inclusion of $R_E$, thereby improving linearity.

\subsection{Dynamic Range Enhancement}

By introducing $R_1$, the dynamic range is increased as $V_2$ can now accommodate larger voltage values:
\[
V_2 = \pm \frac{I_T \times (R_1 + R_E/2)}{2}
\]
For instance, with $I_T = 0.3\,\text{mA}$, $R_1 = 25\,\text{k}\Omega$, and $R_E = 35\,\text{k}\Omega$, the maximum dynamic range for $V_2$ is:
\[
V_2 = \pm 12.75\,\text{V}
\]
This is suitable for typical supply voltages of up to $15\,\text{V}$.

\subsection{Lab Instructions and Oscilloscope Usage}

\subsubsection{Lab Tasks}
\begin{enumerate}
    \item Derive the expression for the output.
    \item Determine the number of quadrants.
    \item Select appropriate values for resistances and capacitances.
    \item Connect the circuit and vary $V_1$ and $V_2$.
    \item Measure and plot $V_{out}$ versus $V_1$ and $V_2$.
    \item Analyze the results and compare them with theoretical predictions.
\end{enumerate}

\subsubsection{Oscilloscope Considerations}

When using oscilloscopes with this circuit:
\begin{itemize}
    \item Do not directly connect the oscilloscope probes between the collectors of Q1 and Q2 as it may cause grounding issues.
    \item Use capacitive coupling or isolated oscilloscope channels to avoid short circuits.
    \item Ensure that the oscilloscope grounds are properly isolated to prevent interference with the circuit operation.
\end{itemize}

\subsection{Transistor Selection and Biasing}

For small-signal transistors (e.g., BC548), typical collector currents are chosen to be around $1\,\text{mA}$ for optimal gain. The biasing resistors $R_B1$ and $R_B2$ are selected based on the desired input impedance and ensuring sufficient base current for proper transistor operation. Following the THB (Thévenin's) rule, resistances up to $50\,\text{k}\Omega$ are suitable, balancing input impedance and bias current requirements.

\subsection{Conclusion}

Today's focus was on enhancing the practical implementation of the Gilbert multiplier by incorporating output referencing and improving dynamic range through resistor $R_1$. This balance between linearity and gain is crucial for real-world applications. In the next class, we will analyze the predistortion circuit in detail.

Any questions about the biasing or circuit configuration? If not, we will proceed to the lab exercises as outlined.
\begin{center}\rule{0.5\linewidth}{0.5pt}\end{center}

\section*{Lecture 16: 14/11/2024}

\subsection{Introduction}

In this section, we analyze the limited dynamic range of the input voltage \( V_1 \) in relation to the currents \( I_1 \), \( I_2 \), \( I_3 \), and \( I_4 \). This analysis focuses on demonstrating the small dynamic range of \( V_1 \) using the general differential amplifier equation. Plotting \( V_1 \) against \( I_1 \) and \( I_2 \) reveals that clipping occurs at \( 4V_T \), where \( V_T \) is the thermal voltage (\( V_T \approx 26\,\text{mV} \) at room temperature).

\subsection{Differential Amplifier Dynamics}

Consider the following equations governing the currents in a general differential amplifier:

\begin{equation}
I_1 = \frac{I_{EE}}{1 + e^{V_1/V_T}}
\end{equation}

\begin{equation}
I_2 = \frac{I_{EE}}{1 + e^{-V_1/V_T}}
\end{equation}

Here, \( V_1 \) is the input voltage, and \( I_{EE} \) is the total emitter current. At \( V_1 = V_T \), substituting into equation (1) yields:

\begin{equation}
I_1 = 0.737\,I_{EE}
\end{equation}

Since \( I_1 + I_2 = I_{EE} \), we find:

\begin{equation}
I_2 = I_{EE} - I_1 = 0.263\,I_{EE}
\end{equation}

Plotting these relationships shows that the currents \( I_1 \) and \( I_2 \) approach saturation at \( \pm 4V_T \), illustrating the limited dynamic range of \( V_1 \).

\subsection{Transconductance (\( g_m \)) Assumptions}

In our initial analysis, we assumed that the transconductance (\( g_m \)) of the transistors is constant. However, \( g_m \) is actually the slope of the \( I-V \) curves and varies with operating conditions. Therefore, a more accurate analysis is necessary to account for the non-constant \( g_m \).

\subsection{Accurate Analysis of the Gilbert Multiplier Cell}

To achieve a more precise understanding, we perform an accurate analysis of the Gilbert multiplier cell by relaxing the assumption of constant \( g_m \). This approach involves deriving the transfer function without presuming a fixed transconductance value.

\subsection{Dynamic Range Limitations}

The total current \( I \) is conserved, meaning that \( I_1 + I_2 = I \). This conservation sets the dynamic range of \( V_1 \) to \( \pm 4V_T \). Beyond this range, the collector current \( I_C \) clips, preventing the circuit from functioning effectively as a multiplier since the voltage cannot vary further.

\subsection{Enhancing the Linear Range with Preconditioning}

To extend the linear range of the input signal \( V_1 \) beyond \( 4V_T \), a preconditioning circuit is employed. This circuit is designed based on the inverse transfer function of the voltage-to-current relationship at the inputs of transistors \( Q_1 \), \( Q_2 \), \( Q_3 \), and \( Q_4 \). By doing so, the linearity of the multiplier is improved, allowing for larger input voltages without clipping.

\subsection{Circuit Configuration}

We analyze the basic Gilbert multiplier cell, which comprises transistors \( Q_3 \), \( Q_4 \), \( Q_5 \), and \( Q_6 \) arranged as parallel differential amplifiers, along with transistors \( Q_1 \) and \( Q_2 \). The collector current of \( Q_1 \) is denoted as \( I_{C1} \).

\subsection{Applying Current-Voltage Relationships}

Using the Ebers-Moll model, we derive the following relationships for the transistors:

\begin{equation}
I_1 = \frac{I_{EE}}{1 + e^{V_{B1}/V_T}}
\end{equation}

\begin{equation}
I_2 = \frac{I_{EE}}{1 + e^{-V_{B2}/V_T}}
\end{equation}

Assuming matched transistors, the equations simplify to:

\begin{equation}
\frac{I_1}{2} = e^{V_1/V_T}
\end{equation}

Since \( I_1 + I_2 = I_{EE} \), we solve for \( I_1 \) and \( I_2 \) accordingly.

\subsection*{Differential Output Current}

The differential output current \( \Delta I \) is given by:

\begin{equation}
\Delta I = I_3 + I_5 - I_4 - I_6
\end{equation}

Simplifying, we have:

\begin{equation}
\Delta I = I_3 - I_6 - (I_4 - I_5)
\end{equation}

Substituting the expressions for \( I_3 \), \( I_4 \), \( I_5 \), and \( I_6 \) from the Ebers-Moll relations, and introducing normalized variables \( X = V_1/V_T \) and \( Y = V_2/V_T \), we simplify the expression further.

\subsection{Utilizing Hyperbolic Functions}

We express the differential current in terms of hyperbolic functions:

\begin{equation}
\Delta I = I_{EE} \tanh\left(\frac{Y}{2}\right) \tanh\left(\frac{X}{2}\right)
\end{equation}

The output voltage \( V_{\text{out}} \) is then:

\begin{equation}
V_{\text{out}} = -R_L \Delta I = -R_L I_{EE} \tanh\left(\frac{Y}{2}\right) \tanh\left(\frac{X}{2}\right)
\end{equation}

This accurately relates the input voltages \( V_1 \) and \( V_2 \) to the output voltage \( V_{\text{out}} \) of the Gilbert multiplier.

\subsection{Conclusion}

The accurate expression for the input-output relationship of the Gilbert multiplier reveals its non-linear behavior. This understanding is crucial for designing the preconditioning circuit, which aims to linearize the multiplier's response for enhanced performance.

In the next lecture, we will delve deeper into equation (25) to analyze different operating cases and explore the design of the preconditioning circuit based on the derived transfer function.
\begin{center}\rule{0.5\linewidth}{0.5pt}\end{center}

\section*{Lecture 17: 18/11/2024}

\subsection{Introduction}
In the previous class, we derived the input-output relationship of the Gilbert's multiplier cell using an accurate analysis. The result obtained is:
\[
V_{out} = -R_L I \tanh\left(\frac{V_2}{2V_T}\right) \tanh\left(\frac{V_1}{2V_T}\right)
\]
This equation accounts for various values of transconductance \( G_M \) and bias currents \( I_B1 \) and \( I_B2 \). Therefore, Equation (25) represents the input-output relationship of a Gilbert's multiplier cell.

\subsection{Applications of Equation (25)}
Equation (25) has three practical applications, categorized based on the magnitude of \( V_1 \). Recall that \( V_T \) is only 26 mV.

\subsubsection{Case 1: \( \frac{V_1}{2V_T} < 1 \) and \( \frac{V_2}{2V_T} < 1 \)}
In this scenario, we can interchange the hyperbolic tangent functions:
\[
\tanh\left(\frac{V_1}{2V_T}\right) \tanh\left(\frac{V_2}{2V_T}\right) = \tanh\left(\frac{V_2}{2V_T}\right) \tanh\left(\frac{V_1}{2V_T}\right)
\]
Given that \( \frac{V_1}{V_T} < 50 \) mV, this condition is not very useful for practical multipliers, such as those used in communication receivers and transceivers, where higher voltages like 5V or 10V are common.

\subsubsection{Case 2: \( V_1 \) and \( V_2 \) Greater Than \( V_T \)}
When both \( V_1 \) and \( V_2 \) are significantly greater than \( V_T \), the output is given by Equation (25) without simplification. To ensure the multiplier's output remains a product of \( V_1 \) and \( V_2 \), two schemes can be employed:
\begin{enumerate}
    \item Use limited generation resistance in the lower differential amplifier to increase its linear output.
    \item Introduce a nonlinearity that pre-distorts the signals \( V_1 \) and \( V_2 \) to compensate for the hyperbolic tangent transfer function.
\end{enumerate}

\subsection{Pre-Distortion Circuit}
In most practical multipliers, the second scheme is commonly used. We introduce a pre-distortion circuit to compensate for the hyperbolic tangent transfer function. The pre-distortion circuit involves:
\begin{itemize}
    \item Applying a hyperbolic inverse tangent function to \( V_1 \), resulting in \( V'_1 \), which is then applied to the top differential amplifiers.
    \item Applying another hyperbolic inverse tangent function to \( V_2 \), resulting in \( V'_2 \), which is directly applied to the bottom transistors (\( Q_5 \) and \( Q_6 \)).
\end{itemize}
This configuration is widely adopted by IC manufacturers and is known as a pre-distortion or preconditioning circuit. Below is the analysis of such a circuit to demonstrate that it produces a hyperbolic inverse transfer function, allowing it to be connected to the standard Gilbert cell and achieving the complete transfer function of the Gilbert multiplier.

\subsection{Analysis of the Pre-Distortion Circuit}
To extend the dynamic range of the signals \( V_1 \) and \( V_2 \) to \( \pm10 \) volts, we implement the following preconditioning circuit:
\begin{figure}[h]
    \centering
\includegraphics[width=\linewidth]{Precond.png}
    \caption{Preconditioning Circuit}
    \label{fig:preconditioning}
\end{figure}
This circuit consists of a normal differential amplifier with two transistors as loads. By connecting the collector to the base, the circuit simplifies to a diode-connected transistor, as shown in Figure \ref{fig:preconditioning}. 

\subsubsection{Deriving the Transfer Function}
The first step is to derive the current-to-voltage relationships for the differential amplifier connected in this manner. Let \( R_E \) be the emitter resistance.

Applying Kirchhoff's Voltage Law (KVL) to the input loop of transistors \( Q_1 \) and \( Q_2 \), we obtain:
\[
V_1 = I_1 R_E - I_2 R_E
\]
\[
I_1 + I_2 = I_{total}
\]
Solving these simultaneous equations, we find:
\[
I_1 = \frac{I_{total}}{2} + \frac{V_1}{2R_E}
\]
\[
I_2 = \frac{I_{total}}{2} - \frac{V_1}{2R_E}
\]

The output voltage \( V_{out} \) is given by:
\[
V_{out} = V_{CC} - V_{D1} - V_{D2} = V_D2 - V_D1
\]
Using the diode current equations:
\[
I_{D1} = I_{S1} e^{\frac{V_{D1}}{V_T}}, \quad I_{D2} = I_{S2} e^{\frac{V_{D2}}{V_T}}
\]
Assuming \( I_{S1} = I_{S2} \), we have:
\[
\tanh^{-1}\left(\frac{V_{out}}{2V_T}\right) = \frac{V_{out}}{2V_T}
\]
Thus, the transfer function becomes approximately:
\[
V_{out} \approx 2V_T \tanh^{-1}\left(\frac{K V_1}{I_{total}}\right)
\]
where \( K = \frac{1}{2R_E} \).

\subsection{Overall Response of the Gilbert Multiplier with Preconditioning}
By applying the pre-distortion circuits to both \( V_1 \) and \( V_2 \), the Gilbert multiplier cell can now handle a much larger dynamic range for \( V_1 \) and \( V_2 \), allowing inputs up to \( \pm10 \) volts.

\subsection{Applications}
Analog multipliers, such as the Gilbert multiplier, have numerous applications including:
\begin{itemize}
    \item \textbf{Analog Computations}: Used in analog computers for performing multiplication operations.
    \item \textbf{Frequency Translation}: Essential in modulators, demodulators, and product detectors, particularly in FM detection and TV engineering for color signal processing.
    \item \textbf{Modems}: Utilized in modulation and demodulation schemes to conserve frequency spectrum.
    \item \textbf{Analog Division}: Circuits designed to perform division of two voltages.
    \item \textbf{Square Root Calculations}: Analog circuits that compute the square root of an input voltage.
\end{itemize}

\subsection{Practical Circuits: MC1496 Multiplier}
An example of a practical multiplier is the MC1496, a linear four-quadrant multiplier. Important considerations while using this IC include:
\begin{itemize}
    \item Pin configurations for differential amplifiers and preconditioning circuits.
    \item Ensuring proper biasing and stable current sources.
    \item Utilizing external resistors and capacitors for filtering and biasing.
\end{itemize}
\subsubsection{Pin Configuration and Resistors}
For the MC1496:
\begin{itemize}
    \item Pins 2 and 14 are connected to \( V_{CC} \) through 3k resistors.
    \item Pins 5 and 6 are connected with 27k resistors.
    \item External resistors \( R_X \) are connected to pins 10 and 11 for setting gain.
    \item Pin 1 connects to the preconditioning circuit.
\end{itemize}

\subsection{Current Source Analysis}
In the preconditioning circuit, the constant current source voltage is maintained for temperature stability using diode-connected transistors. For example, in a circuit with \( Q_9 \) and \( Q_{10} \):
\[
I = \frac{V - V_{BE}}{R}
\]
Where \( V_{BE} \) is typically 0.6V for low currents. Ensuring identical currents through matched transistors maintains symmetry and stability in the multiplier circuit.

\subsection{Exam Questions and Practical Exercises}
Students are encouraged to analyze the current through the preconditioning circuits and the Gilbert multiplier cell. Practical exercises include:
\begin{itemize}
    \item Calculating the emitter currents in the multiplier and preconditioning transistors.
    \item Designing resistor values for desired current sourcing.
    \item Implementing and testing analog multiplication and division circuits.
\end{itemize}

\subsection{Conclusion}
The Gilbert multiplier cell, enhanced with preconditioning circuits, offers a robust solution for analog multiplication with a wide dynamic range. Its applications span various fields, including communications, signal processing, and analog computing. Understanding the underlying principles and practical implementations is crucial for leveraging its full potential in electronic circuit design.

\subsection{Practical Circuit Implementation}
Refer to the MC1496 datasheet for detailed pin configurations and application circuits. Ensure proper resistor values and biasing to maintain optimal performance. Pay attention to pin connections for differential amplifiers and preconditioning circuits to achieve the desired multiplication characteristics.

\subsection{Questions and Discussion}
Students are encouraged to ask questions regarding the analysis and implementation of the Gilbert multiplier and preconditioning circuits. Practical understanding is essential for exam success and real-world applications.
\begin{center}\rule{0.5\linewidth}{0.5pt}\end{center}
\section{Lecture 18: 21/11/2024}

\subsection{Introduction}
DC voltage regulators are essential components in modern electrical and electronic equipment that require specific, stable voltages for proper operation. This lecture introduces the fundamental concepts, types, and applications of DC voltage regulators.

\subsection{Student Assignment}
\textbf{Research Task:} In your free time, please:
\begin{itemize}
    \item Use Google to search for IC-based multipliers and their manufacturers
    \item Study configurations from companies like Motorola 
    \item Examine their internal circuitry
    \item Analyze bias circuits and application circuits
    \item Document typical quiescent currents in properly biased circuits
\end{itemize}

\subsection{Key Applications}
DC voltage regulators find widespread use across various sectors:

\subsubsection{Industrial Applications}
\begin{itemize}
    \item Electroplating processes
    \item Arc welding (47-67V, 100-250A)
    \begin{itemize}
        \item Stick welding for construction
        \item MIG (Metal Inert Gas) welding for industrial processes
        \item TIG welding for specialized materials like stainless steel
    \end{itemize}
    \item Chemical processes and electrolysis
\end{itemize}

\subsubsection{Electronic Equipment}
\begin{itemize}
    \item Computer power supplies ($\pm12$V, $+5$V, $+3.3$V)
    \item Laboratory instruments (e.g., oscilloscopes requiring 200-2500V)
    \item Laboratory power supplies ($\pm15$V, $+5$V)
\end{itemize}

\subsubsection{Domestic Applications}
\begin{itemize}
    \item Television sets
    \item DVD players
    \item Laptop chargers
    \item Mobile phone chargers (typically 5V)
\end{itemize}

\subsection{Design Considerations}
Key factors in designing DC voltage regulators:
\begin{enumerate}
    \item Feasibility and practicality
    \item Efficiency
    \item Mechanical construction
    \item Speed of response
    \item Design cost
    \item Protection circuit requirements
\end{enumerate}

\subsection{Theoretical Concepts}

\subsubsection{Classification of Regulators}
\begin{enumerate}
    \item Linear Regulators
    \begin{itemize}
        \item Series regulators
        \item Shunt regulators
    \end{itemize}
    
    \item Switching Regulators
    \begin{itemize}
        \item Higher efficiency
        \item Smaller size
        \item Multiple output capabilities
    \end{itemize}
\end{enumerate}

\subsubsection{Essential Components}
A typical DC voltage regulator consists of:
\begin{enumerate}
    \item Voltage reference element ($V_{ref}$)
    \item Sampling elements
    \item Error amplifier
    \item Power control element
\end{enumerate}

\subsection{Mathematical Framework}

\subsubsection{Reference Circuit Analysis}
For a Zener diode reference circuit:

Maximum power dissipation:
\[ P_{Z_{max}} = I_{Z_{max}} \cdot V_Z \]

Limiting resistor calculation:
\[ R > \frac{V_{in} - V_Z}{I_{Z_{max}}} \]

Including dynamic resistance ($R_Z$):
\[ I_Z = \frac{V_{in} - V_Z}{R_Z + R} \]
\[ V_{ref} = V_Z + I_Z \cdot R_Z \]

\subsubsection{Sampling Circuit Analysis}
Feedback voltage:
\[ V_{feedback} = V_{out} \cdot \frac{R_2}{R_1 + R_2} = \beta V_{out} \]

Where sampling factor $\beta$:
\[ \beta = \frac{R_2}{R_1 + R_2} \]

\subsubsection{Error Amplifier Output}
\[ V_{error} = A_d(V_{ref} - \beta V_{out}) \]

Where:
\begin{itemize}
    \item $A_d$ is the differential gain
    \item $V_{ref}$ is the reference voltage
    \item $\beta V_{out}$ is the sampled output voltage
\end{itemize}

\subsection{Practical Applications}
Modern applications include:
\begin{itemize}
    \item Laptop and phone chargers using switching regulators
    \item Computer power supplies with multiple outputs
    \item Industrial power supplies
    \item Automotive electronics
\end{itemize}

\subsection{Practice Problems}
\textbf{Design Exercise:} Given a Zener diode with:
\begin{itemize}
    \item $V_Z = 3.3$V
    \item $I_{Z_{max}} = 0.1$A
    \item $P_{Z_{max}} = 0.33$W
    \item Supply voltage = 10V
\end{itemize}

Calculate:
\begin{enumerate}
    \item Minimum value for limiting resistor R
    \item Power dissipation in both resistor and Zener
    \item Recommend a practical resistance value considering efficiency
\end{enumerate}
\onecolumn
\chapter{Past Paper Questions}
\section*{Sample I\footnotetext{Sample I: 18th February 2021}}

\begin{enumerate}
\item 
    \begin{enumerate}[(a)]
        \item Briefly explain why instrumentation amplifiers are unsuitable for amplifying signals due to their limitations. Hence, state clearly the principle used to eliminate the limitations of the former amplifier. \hfill{(5 marks)}
        
        \item List the cause of non-quantifiable errors in the series-shunt switch configuration. Hence, state why it offers better performance than the shunt switch. \hfill{(8 marks)}
        
        \item The chopper-stabilized amplifier system shown in \textbf{Figure 1} has an input of \textbf{5 mV DC}. The operational amplifier (op-amp) used has a drift voltage of \textbf{0.02 mV} when referred to its input, and this drift is assumed to be sinusoidal at a frequency of \textbf{0.002 Hz}.

For the circuit, determine the maximum percentage error of the amplified signal at the output of the system.

The FET parameters are given as:
\begin{itemize}
    \item \( R_{\text{ds\_ON}} = \textbf{25} \, \Omega \)
    \item \( R_{\text{ds\_OFF}} = \textbf{10^{10}} \, \Omega \)
\end{itemize}

The chopping frequency is \textbf{1 kHz} (square wave).

The op-amp may be assumed ideal, apart from suffering from offsets. \hfill{(12 marks)}
    \end{enumerate}

\begin{figure}[h!]
    \centering
    \includegraphics[width=0.85\textwidth]{PPNo.jpg}
    \caption{Chopper-stabilized Amplifier System}
\end{figure}

\begin{center}\rule{0.5\linewidth}{0.5pt}\end{center}

\item 
    \begin{enumerate}[(a)]
        \item Sketch a circuit capable of computing the logarithm of an analog voltage and does not suffer from the effects of reverse saturation currents of the transistors. Verify this with the aid of analysis. List some of its limitations and possible ways of improving its performance. \hfill{(4 marks)}

        \item Obtain the transfer function of the Gilbert multiplier cell of \textbf{Figure 2}. Hence, show that it acts as a four-quadrant multiplier. List the limitations of this basic cell, and state how production IC multipliers solve these limitations. Obtain the quiescent currents in the transistors and evaluate the input resistance seen by signals \( V_x \) and \( V_y \). \hfill{(10 marks)}
    \end{enumerate}

\begin{figure}[h!]
    \centering
    \includegraphics[width=0.36\textwidth]{PPN1.jpg}
    \caption{Gilbert Multiplier Cell}
\end{figure}

\begin{center}\rule{0.5\linewidth}{0.5pt}\end{center}

\item
    \begin{enumerate}[(a)]
        \item List the important characteristics of the \textbf{series} and \textbf{switched-mode} regulators. Hence, state which applications each is suited for, giving reasons. \hfill{(8 marks)}

        \item The regulator of \textbf{Figure 3} is required to supply a nominal load of \textbf{5A} from an input that varies from \textbf{20V} to \textbf{40V}. Evaluate the output voltage, given that the op-amp has an open-loop gain of \textbf{10,000}, and the transistors have a \( V_{\text{be}} \) of \textbf{0.6V}. State the function of Q3. \hfill{(4 marks)}
        
        \item Given that the output transistor Q1 has \( h_{\text{fe}} = \textbf{100} \), design requirements for the driver transistors Q2 and Q3 if the op-amp has a maximum output drive capability of \textbf{5mA}. Calculate the dissipation requirements of the Zener and transistors. What is the minimum efficiency of the regulator at the maximum load current? \hfill{(8 marks)}
    \end{enumerate}

\begin{figure}[h!]
    \centering
    \includegraphics[width=0.6\textwidth]{PPN2.jpg}
    \caption{Voltage Regulator Circuit}
\end{figure}

\begin{center}\rule{0.5\linewidth}{0.5pt}\end{center}

\item
    \begin{enumerate}[(a)]
        \item Explain the principle of operation of the circuit in \textbf{Figure 4}. What are the functions of diodes, inductors, and capacitors? State the criteria used to determine their selection. \hfill{(5 marks)}

        \item Of the schemes used to implement closed-loop supply, state which is best for a laptop power supply, giving reasons. \hfill{(5 marks)}

        \item In a step-down switching regulator of \textbf{Figure 4}, the control unit operates at \textbf{5 kHz}. The circuit converts \textbf{32 V} into \textbf{5 V} at a nominal current of \textbf{8 A} and a peak ripple voltage of \textbf{100 mV}. The peak-to-peak ripple current of the inductor is \textbf{1.2 A}. Assuming ideal devices, determine the values of inductor \( L \) and capacitance \( C \). \hfill{(10 marks)}
    \end{enumerate}

\begin{figure}[h!]
    \centering
    \includegraphics[width=0.6\textwidth]{PPN3.jpg}
    \caption{Step-down Switching Regulator}
\end{figure}

\begin{center}\rule{0.5\linewidth}{0.5pt}\end{center}

\section*{Sample II\footnotetext{Sample II: 27/11/2017}}

\begin{enumerate}
\item 
    \begin{enumerate}[(a)]
        \item List the limitations of a low noise instrumentation amplifier hence explain the need for a chopper stabilized amplifier. Explain its principle of operation. \hfill{(5 marks)}

        \item With aid of analysis evaluate the ON and OFF error of the series-shunt chopper. Comment on the errors due to switching spikes as compared to the series chopper. \hfill{(8 marks)}

        \item The chopper stabilized amplifier system shown in \textbf{Figure 1} has an input of \textbf{10mV DC}. The op-amp used has a drift voltage of \textbf{0.01mV} when referred to its input and assume sinusoidal at \textbf{0.01 Hz}. For the circuit determine the maximum percentage error of the amplified signal at the output of the system. The FET parameters are given as:
\begin{itemize}
    \item \( R_{\text{ds\_ON}} = \textbf{25} \, \Omega \)
    \item \( R_{\text{ds\_OFF}} = \textbf{10^{10}} \, \Omega \)
\end{itemize}
 \hfill{(7 marks)}
    \end{enumerate}

\begin{figure}[h!]
    \centering
    \includegraphics[width=0.7\textwidth]{PPM1.jpg}
    \caption{Chopper Stabilized Amplifier System}
\end{figure}
\begin{center}\rule{0.5\linewidth}{0.5pt}\end{center} 
\item
    \begin{enumerate}[(a)]
        \item List the advantages of variable transconductance scheme of implementation of multipliers hence explain how you would design multipliers to work at speeds of \textbf{2GHz}. \hfill{(4 marks)}

        \item Obtain the transfer function of the Gilbert cell only in circuit of \textbf{Figure 2}. \hfill{(8 marks)}

        \item Evaluate the transfer function of pre-conditioning circuit only of \textbf{Figure 2} hence determine the overall transfer function of the complete circuit. What are the functions of the preconditioning circuit? \hfill{(8 marks)}
    \end{enumerate}

\begin{figure}[h!]
    \centering
    \includegraphics[width=0.7\textwidth]{PPMo.png}
    \caption{Gilbert Cell Circuit}
\end{figure}
\begin{center}\rule{0.5\linewidth}{0.5pt}\end{center} 
\item
    \begin{enumerate}[(a)]
        \item Sketch the blocks that constitute a series closed loop regulator clearly explaining their functions, hence explain the principle of operation of the series regulator. List the advantages and disadvantages of the series regulator over the shunt regulator. \hfill{(8 marks)}

        \item The regulator of \textbf{Figure 3} is required to supply a nominal load of \textbf{5A} from an input that varies from \textbf{20-45 V}. Evaluate the output voltage given that the op-amp has an open loop gain of \textbf{50000} and the transistors have a \( V_{\text{be}} \) active of \textbf{0.6V}. \hfill{(5 marks)}

        \item Given that transistors have \( h_{\text{fe}}=\textbf{100} \), design requirements of both the transistor if the op-amp has a maximum output drive capability of \textbf{15mA}. Calculate the dissipation requirements of the zener and transistors. What is the minimum efficiency of the regulator? State any assumptions used. Draw a modified circuit with short protection at \textbf{5 amps} \hfill{(7 marks)}
    \end{enumerate}

\begin{figure}[h!]
    \centering
    \includegraphics[width=0.4\textwidth]{PPM2.jpg}
    \caption{Series Regulator Circuit}
\end{figure}
\begin{center}\rule{0.5\linewidth}{0.5pt}\end{center} 
\item
    \begin{enumerate}[(a)]
        \item List the important functional blocks that constitute a switching regulator hence clearly state their functions hence its principle of operation. \hfill{(5 marks)}

        \item Of the schemes used to implement closed loop DC power supplies state the scheme best suited for the desktop computer power supply giving four reasons \hfill{(5 marks)}

        \item In a step-down switching regulator of \textbf{Figure 4} has a control unit that operates at \textbf{15kHz}. The circuit converts \textbf{40 V} into \textbf{7.33V} at a nominal current of \textbf{4A} and a peak ripple voltage of \textbf{200mV}. The peak-to-peak ripple current of the inductor is \textbf{1.2 A}. Assuming ideal devices determine the values of inductor \( L \) and capacitance \( C \). \hfill{(10 marks)}
    \end{enumerate}

\begin{figure}[h!]
    \centering
    \includegraphics[width=0.4\textwidth]{PPM3.jpg}
    \caption{Step-down Switching Regulator}
\end{figure}
\begin{center}\rule{0.5\linewidth}{0.5pt}\end{center} 
\end{enumerate}

\end{document}